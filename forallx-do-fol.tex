%!TEX root = forallxdo.tex

\part{Die Logik erster Ordnung}
\label{ch.FOL}
\addtocontents{toc}{\protect\mbox{}\protect\hrulefill\par}
\chapter{Bausteine der LEO}\label{s:FOLBuildingBlocks}

\section{Die Notwendigkeit, Sätze zu zerlegen}
Lasst uns das folgende Argument betrachten, welches im Deutschen offenbar gültig ist:
\begin{earg}
\label{willard1}
\item[] Willard ist ein Logiker.
\item[] Alle Logiker*innen tragen komische Hüte. 
\item[\therefore] Willard trägt einen komischen Hut.
\end{earg}
Um es in der WFL zu symbolisieren, könnten wir einen Symbolisierungsschlüssel anbieten:
\begin{ekey}
\item[L] Willard ist ein Logiker.
\item[A] Alle Logiker*innen tragen komische Hüte.
\item[F] Willard trägt einen komischen Hut.
\end{ekey}
Das Argument sieht dann so aus:
$$L, A \therefore F$$
Aber der Wahrheitstabellentest wird nun zeigen, dass dieses Argument \emph{ungültig} ist. Was ist schief gelaufen?

Das Problem ist nicht, dass wir beim Symbolisieren des Arguments einen Fehler gemacht haben. Unsere Symbolisierung ist die beste Symbolisierung, die wir \emph{in der WFL} geben können. Das Problem liegt bei der WFL selbst. Bei `Alle Logiker*innen tragen komische Hüte' geht es sowohl um Logiker*innen als auch um das Tragen von komischen Hüten. Wenn wir diese Struktur in unserer Symbolisierung nicht beibehalten, verlieren wir den Verbindung zwischen dem Logikerdasein Willards und seinem komischen Hut.

Die Basiselemente der WFL sind Satzbuchstaben, und die WFL kann diese nicht weiter auseinandernehmen. Um Argumente wie das vorhergehende zu symbolisieren, müssen wir eine neue formale Sprache entwickeln, die es uns erlaubt, \emph{das Atom zu spalten}. Wir werden diese Sprache die \emph{Logik erster Ordnung} oder \emph{LEO} nennen. 

Die Einzelheiten der LEO werden wir im Laufe dieses Kapitels erläutern, aber hier sind drei Begriffe die wir für das Spalten der einfachen Sätze, die wir in der WFL mit Satzbuchstaben symbolisieren, nutzen werden.

Zuerst haben wir \emph{Namen}. In der LEO symbolisieren wir diese mit kursiven Kleinbuchstaben. Zum Beispiel könnte `$b$' für Bertie stehen, oder `$w$' für Willard.

Zweitens haben wir \emph{Prädikate}. Deutsche Prädikate sind Ausdrücke wie `\blank\ ist ein Hund' oder `\blank\ ist ein/e Logiker*in'. Dies sind an sich keine vollständigen Sätze. Um einen vollständigen Satz zu bilden, müssen wir die Lücke im Prädikat ausfüllen. Wir müssen etwas sagen wie `Bertie ist ein Hund' oder `Willard ist ein Logiker'. In der LEO symbolisieren wir Prädikate mit kursiven Gro{\ss}buchstaben. Zum Beispiel könnten wir das LEO-Prädikat `$H$' das deutsche Prädikat `\blank\ ist ein Hund' symbolisieren lassen. Dann wird der Ausdruck `$\atom{H}{b}$' in der LEO ein Satz sein, der den deutschen Satz `Bertie ist ein Hund' symbolisiert. Ebenso könnten wir das FOL-Prädikat `$L$' das deutsche Prädikat `\blank\ ist ein/e Logiker*in' symbolisieren lassen. Dann wird der Ausdruck `$\atom{L}{w}$' den deutschen Satz `Willard ist ein Logiker' symbolisieren.

Drittens haben wir \emph{Quantoren}. Zum Beispiel wird `$\exists$' grob ausdrücken: `Es gibt zumindest einen \ldots'. Wir könnten also den deutschen Satz `Es gibt einen Hund' mit dem LEO-Satz `$\exists x\, \atom{H}{x}$' symbolisieren, den wir als `Es gibt zumindest ein Ding, $x$, und $x$ ist ein Hund'.

\section{Namen}
Im Deutschen ist ein \emph{singulärer Term} ein Wort oder eine Phrase, das/die sich auf eine \emph{bestimmte} Person, einen \emph{bestimmten} Ort oder eine \emph{bestimmte} Sache bezieht. Das Wort `Hund' ist kein singulärer Term, weil es sehr viele Hunde gibt, auf die sich `Hund' bezieht. Der Ausdruck `Bertie' hingegen ist ein singulärer Term, weil er sich auf einen \emph{bestimmten} Terrier bezieht. Ebenso ist der Ausdruck `Philips Hund Bertie' ein singulärer Term, weil er sich auf einen \emph{bestimmten} Terrier bezieht. 

\emph{Eigennamen} sind eine besonders wichtige Art von singulären Termen. Dies sind Ausdrücke, die auf Individuen verweisen, ohne sie weiter zu beschreiben. Der Name `Emerson' ist ein Eigenname, und der Name allein sagt noch nichts über Emerson aus. Natürlich werden einige Namen traditionell an Jungen und andere an Mädchen vergeben. Wenn `Hilary' als singulärer Term verwendet wird, könnte man vermuten, dass er sich auf eine Frau bezieht. Damit kann man aber auch falsch liegen. Tatsächlich weist der Name nicht einmal notwendigerweise darauf hin, dass das Individuum, auf das verwiesen wird, überhaupt eine Person ist: Hilary könnte auch eine Giraffe sein. 

In der LEO sind unsere \define{Namen} kursive Kleinbuchstaben `$a$' bis `$r$'. Wir können Subskripte hinzufügen, wenn wir einen Buchstaben mehr als einmal verwenden wollen. Hier sind also einige singulärer Terme der LEO:
	$$a,b,c,\ldots, r, a_1, f_{32}, j_{390}, m_{12}$$
Diese sollten wie die Eigennamen im Deutschen verstanden werden, aber mit einem Unterschied. `Tim Button' ist ein Eigenname, aber es gibt mehrere Personen mit diesem Namen. (Gleicherma{\ss}en gibt es mindestens zwei Personen mit dem Namen `P.D.\ Magnus'.) Wir leben mit dieser Art von Zweideutigkeit im Deutschen, da das Deutsche zulässt, dass der Kontext dafür zuständig ist, dass sich `Tim Button' auf einen Autor dieses Buches bezieht und nicht auf ein anderes Individuum, das ebenfalls `Tim Button' hei{\ss}t. In der LEO allerdings tolerieren wir eine solche Zweideutigkeit nicht. Jeder Name muss auf \emph{genau} ein Individuum verweisen. (Zwei verschiedene Namen können jedoch auf dasselbe Individuum verweisen).

\newglossaryentry{name}{
  name = Name,
  description = {Ein Symbol der LEO, genutzt, um auf ein Individuum in der \gls{Domäne} zu verweisen}
  }

Wie in der WFL stellen wir Symbolisierungsschlüssel zur Verfügung. Diese geben vorübergehend an, worauf ein Name verweist. Zum Beispiel:
	\begin{ekey}
		\item[e] Elsa
		\item[g] Gregor
		\item[m] Marybeth
	\end{ekey}

\section{Prädikate}
Die einfachsten Prädikate sind Eigenschaften von Individuen. Es sind Dinge, die man von einem Objekt aussagen kann; Dinge, die auf ein Objekt zutreffen oder nicht. Hier sind einige Beispiele deutscher Prädikate:
	\begin{quote}
		\blank\ ist ein Hund\\
		\blank\ ist ein Mitglied von Monty Python\\
		Ein Klavier fiel auf \blank
	\end{quote}
Im Allgemeinen kann man sich Prädikate als Dinge vorstellen, die sich mit singulären Termen zu Sätzen verbinden. Umgekehrt können Sie mit Sätzen beginnen und aus ihnen Prädikate bilden, indem sie singuläre Terme entfernen. Betrachten Sie z.B.\@ den Satz: `Vinnie hat sich das Familienauto von Nunzio geliehen'. Durch das Entfernen eines singulären Terms können wir eines von drei verschiedenen Prädikaten erhalten:
	\begin{quote}
		\blank\ hat das Familienauto von Nunzio geliehen\\
		Vinnie hat \blank\ von Nunzio geliehen\\
		Vinnie hat das Familienauto von \blank\ geliehen
	\end{quote}
In der LEO sind \define{Prädikate} kursive Gro{\ss}buchstaben $A$ bis $Z$, mit oder ohne Subskripten. So könnten wir einen Symbolisierungsschlüssel für Prädikate schreiben:
	\begin{ekey}
		\item[\atom{A}{x}] \gap{x} ist verärgert
		\item[\atom{G}{x}] \gap{x} ist glücklich
		\item[\atom{G_1}{x,y}] \gap{x} ist so gro{\ss} wie oder grö{\ss}er als \gap{y}
		\item[\atom{Z}{x,y}] \gap{x} ist so zäh oder zäher wie \gap{y}
		\item[\atom{Z_2}{x,y,z}] \gap{y} befindet sich zwischen \gap{x} und \gap{z}
	\end{ekey}
        (Warum die Subskripte zu den Lücken? Darauf kommen wir in \S\ref{s:MultipleGenerality} zurück).

%In diesen Beispielen, sehen Sie Prädikate verschiedener Art. Die ersten zwei sind \emph{einstellige} Prädikate, die zweiten zwei \emph{zweistellige} und das letzte ist ein \emph{dreigliedriges} Prädikat. Die Zahl der Glieder eines Prädikats bestimmt, wie viele singuläre Terme, Vorkommnisse eines singulären Terms, oder Quantoren notwendig sind um aus einem Prädikat einen Satz zu formen. Generell gilt: für jedes Glied brauchen wir ein Vorkommnis eines singulären Terms oder einen Quantor, der dieses Glied ausfüllt.

\newglossaryentry{predicate}{
  name = Prädikat,
  description = {Ein Symbol der LEO, genutzt, um eine Eigenschaft oder Beziehung auszudrücken}
}

Wenn wir unsere beiden Symbolisierungsschlüssel kombinieren, können wir nun damit beginnen, einige deutsche Sätze zu symbolisieren, die diese Namen und Prädikate verwenden. Nehmen wir zum Beispiel die deutschen Sätze:
	\begin{earg}
		\item[\ex{terms1}] Elsa ist verärgert.
		\item[\ex{terms2a}] Gregor und Marybeth sind verärgert.
		\item[\ex{terms2}] Wenn Elsa verärgert ist, dann sind auch Gregor und Marybeth verärgert.
	\end{earg}
Satz \ref{terms1} ist einfach: wir symbolisieren ihn als `$\atom{A}{e}$'.

Satz \ref{terms2a} ist eine Konjunktion zweier einfacher Sätze. Die einfachen Sätze können durch `$\atom{A}{g}$' und `$\atom{A}{m}$' symbolisiert werden. Dann bedienen wir uns unserer Ressourcen aus der WFL und symbolisieren den ganzen Satz durch `$\atom{A}{g} \eand \atom{A}{m}$'. Dies veranschaulicht einen wichtigen Punkt: Die LEO beinhaltet alle wahrheitsfunktionalen Junktoren der WFL. Sie ist eine \emph{Erweiterung} der WFL. 

Satz~\ref{terms2} ist ein Konditional, dessen Antezedens Satz~\ref{terms1} und dessen Konsequens Satz~\ref{terms2a} ist. Daher können wir diesen Satz mit `$\atom{A}{e} \eif (\atom{A}{g} \eand \atom{A}{m})$' symbolisieren.

\section{Quantoren}
Wir sind jetzt bereit, Quantoren einzuführen. Betrachten Sie diese Sätze:
	\begin{earg}
		\item[\ex{q.a}] Alle sind glücklich.
%		\item[\ex{q.ac}] Alle sind zumindest so zäh wie Elsa.
		\item[\ex{q.e}] Jemand ist verärgert.
	\end{earg}
Es könnte verlockend sein, den Satz \ref{q.a} als `$\atom{H}{e} \eand \atom{H}{g} \eand \atom{H}{m}$' zu symbolisieren. Doch dies würde nur sagen, dass Elsa, Gregor und Marybeth glücklich sind. Wir wollen aber sagen, dass alle glücklich sind, auch diejenigen, die keinen Namen haben. Um dies zu erreichen, führen wir das Symbol `$\forall$' ein. Dieses Symbol nennen wir den \define{Universalquantor}.

\newglossaryentry{universal quantifier}{
  name = Universalquantor,
  description = {Das Symbol $\forall$ der LEO, genutzt, um die Allgemeinheit zu symbolisieren; $\forall x\, \atom{F}{x}$ ist wahr genau dann, wenn jedes Element der Domäne~$F$ ist}
}

Auf einen Quantor muss immer eine \define{Variable} folgen. In der LEO sind Variablen kursive Kleinbuchstaben `$s$' bis `$z$', mit oder ohne Subskripte. Wir könnten also den Satz \ref{q.a} als `$\forall x\, \atom{G}{x}$' symbolisieren. Die Variable `$x$' dient als eine Art Platzhalter. Der Ausdruck `$\forall x$' bedeutet intuitiv, dass Sie jedes Individuum auswählen und als `$x$' eintragen können. Das nachfolgende `$\atom{G}{x}$' zeigt an, dass das Ding, das Sie ausgewählt haben, glücklich ist. 

\newglossaryentry{variable}{
  name = Variable,
  description = {Ein Symbol der LEO, welches nach Quantoren und als Platzhalter in einfachen Sätzen genutzt wird; kursive Kleinbuchstaben zwischen $s$ und~$z$}
}

Es sei darauf hingewiesen, dass es keinen besonderen Grund gibt, `$x$' anstelle einer anderen Variable zu verwenden. Die Sätze `$\forall x\, \atom{G}{x}$', `$\forall y\, \atom{G}{y}$', `$\forall z\, \atom{G}{z}$' und `$\forall x_5 \atom{G}{x_5}$' verwenden verschiedene Variablen, aber sie sind alle äquivalent.

Um den Satz \ref{q.e} zu symbolisieren, führen wir ein weiteres neues Symbol ein: den \define{Existenzquantor}, `$\exists$'. Wie der Universalquantor erfordert auch der Existenzquantor eine Variable. Der Satz \ref{q.e} kann durch `$\exists x\,\atom{A}{x}$' symbolisiert werden. Während `$\forall x\,\atom{A}{x}$' als `für alle $x$, $x$ ist verärgert', wird `$\exists x\,\atom{A}{x}$' als `es gibt zumindest ein Ding, $x$, und $x$ ist verärgert'. Auch hier gilt: die Variable ist eine Art Platzhalter. Wir hätten Satz~\ref{q.e}) auch als `$\exists z\, \atom{A}{z}$', `$\exists w_{256}\, \atom{A}{w_{256}}$', usw.\@ symbolisieren können.

\newglossaryentry{existential quantifier}{
  name = Existenzquantor,
  description = {Das Symbol $\exists$ der LEO, genutzt, um die Existenz zu symbolisieren; $\exists x\, \atom{F}{x}$ ist wahr genau dann, wenn zumindest ein Element der Domäne~$F$ ist}
}

Einige weitere Beispiele werden helfen. Betrachten Sie diese weiteren Sätze:
	\begin{earg}
		\item[\ex{q.ne}] Niemand ist verärgert.
		\item[\ex{q.en}] Es gibt jemanden, der nicht glücklich ist.
		\item[\ex{q.na}] Nicht alle sind glücklich.
	\end{earg}
Satz \ref{q.ne} kann wie folgt umschrieben werden: `Es ist nicht der Fall, dass jemand verärgert ist'. Wir können ihn mithilfe der Negation und eines Existenzquantors symbolisieren: `$\enot \exists x\, \atom{A}{x}$'. Satz \ref{q.ne} könnte aber auch wie folgt umgeschrieben werden: `Alle sind nicht verärgert'. Wenn er so verstanden wird, dann kann dieser Satz mithilfe der Negation und eines Universalquantors symbolisiert werden: `$\forall x\, \enot \atom{A}{x}$'. Beide Alternativen sind akzeptable Symbolisierungen. Es wird sich in der Tat herausstellen, dass im Allgemeinen $\forall x\, \enot \metav{A}$ äquivalent zu $\enot \exists x\, \metav{A}$ ist. (Beachten Sie, dass wir hier `$\metav{A}$' als Metavariable verwenden). Einen Satz auf die eine Art und Weise zu symbolisieren und nicht auf die andere, mag in manchen Kontexten `natürlicher' erscheinen, aber es ist nicht viel mehr als eine Frage des Geschmacks.

Der Satz \ref{q.en} wird am natürlichsten als `Es gibt zumindest etwas~,$x$, und $x$ ist nicht glücklich'. Dies wird dann als `$\exists x\, \enot \atom{G}{x}$' symbolisiert. Natürlich hätten wir auch `$\enot \forall x\, \atom{G}{x}$' schreiben können, was wir so verstehen würden: `Es ist nicht der Fall, dass alle glücklich sind'. Auch das wäre eine vollkommen angemessene Symbolisierung des Satzes \ref{q.na}.

\section{Domänen}
Angesichts des Symbolisierungsschlüssels, den wir verwendet haben, symbolisiert `$\forall x\,\atom{G}{x}$' `Alle sind glücklich'. Wer ist mit diesem \emph{alle} gemeint? Wenn wir im Deutschen Sätze wie diesen verwenden, meinen wir für gewöhnlich nicht jeden, der jetzt auf der Erde lebt. Mit Sicherheit meinen wir nicht jeden, der jemals gelebt hat oder jemals leben wird. Gewöhnlich meinen wir etwas Bescheideneres: jeden, der jetzt im Gebäude ist, jeden, der dieses Modul belegt, oder ähnliches.

Um diese Mehrdeutigkeit zu beseitigen, müssen wir eine \define{Domäne} angeben. Die Domäne ist die Menge von Dingen, über die wir sprechen. Wenn wir also über Menschen in Dortmund sprechen wollen, dann definieren wir die Domäne als Menschen in Dortmund. Wir schreiben dies an den Anfang des Symbolisierungsschlüssels, etwa so:
	\begin{ekey}
		\item[\text{Domäne}] Menschen in Dortmund
	\end{ekey}
Die Quantoren \emph{überspannen} diese Domäne. Angesichts dieser Domäne ist `$\forall x$' ungefähr so zu lesen wie `Jeder Mensch in Dortmund ist so, dass \ldots' und `$\exists x$' ist ungefähr so zu lesen wie `Zumindest ein Mensch in Dortmund ist so, dass \ldots'. 

\newglossaryentry{domain}{
  name = Domäne,
  description = {Die Mengen an Dingen, die wir in einer Symbolisierung der LEO voraussetzen oder die in einer \gls{Interpretation} in die Spannweite der Quantoren fällt}
}

In der LEO muss die Domäne immer mindestens ein Ding enthalten. Darüber hinaus können wir im Deutschen legitimerweise aus `Gregor ist verärgert' `Jemand ist verärgert' herleiten. In der LEO wollen wir also ebenso in der Lage sein, `$\exists x\, \atom{A}{x}$' von `$\atom{A}{g}$' herzuleiten. Wir werden also darauf bestehen, dass jeder Name auf genau ein Ding in der Domäne verweist. Wenn wir Menschen von au{\ss}erhalb Dortmunds benennen wollen, dann müssen wir diese Menschen in die Domäne aufnehmen. 
	\factoidbox{
		Eine Domäne hat \emph{mindestens} ein Element. Jeder Name muss auf \emph{genau} ein Element der Domäne verweisen, wobei auf ein Element der Domäne von einem Namen, mehreren Namen oder keinem Namen verwiesen werden kann.
	}

Selbst die Zulassung einer Domäne mit nur einem Element kann zu merkwürdigen Ergebnissen führen. Nehmen wir an, wir haben dies als einen Symbolisierungsschlüssel:
\begin{ekey}
\item[\text{Domäne}] der Eiffelturm
\item[\atom{P}{x}] \gap{x} ist in Paris.
\end{ekey}
Der Satz $\forall x\,\atom{P}{x}$ könnte im Deutschen als `Alles ist in Paris' paraphrasiert werden. Doch das wäre irreführend. Es bedeutet, dass alles \emph{in der Domäne} in Paris ist. Diese Domäne enthält nur den Eiffelturm, so dass mit diesem Symbolisierungsschlüssel $\forall x\,\atom{P}{x}$ nur bedeutet, dass der Eiffelturm in Paris ist.

\subsection{Leere Terme}

In der LEO muss jeder Name auf genau ein Element der Domäne verweisen. Ein Name kann sich nicht auf mehr als ein Ding beziehen - es ist ein \emph{singulärer} Term. Jeder Name muss aber auf \emph{etwas} verweisen. Dies ist mit einem klassischen philosophischen Problem verbunden: dem so genannten Problem der leeren Terme.

Mittelalterliche Philosophen benutzten typischerweise Sätze über die \emph{Chimäre}, um dieses Problem zu veranschaulichen. Die Chimäre ist ein mythologisches Geschöpf; sie existiert nicht wirklich. Betrachten Sie diese beiden Sätze:
\begin{earg}
\item[\ex{chimera1}] Die Chimäre ist verärgert.
\item[\ex{chimera2}] Die Chimäre ist nicht verärgert.
\end{earg}
Es ist verlockend, einen Namen einfach so zu definieren, dass er die gleiche Bedeutung wie `die Chimäre' hat. Der Symbolisierungsschlüssel würde wie folgt aussehen: 
\begin{ekey}
\item[\text{Domäne}] Irdische Kreaturen
\item[\atom{A}{x}] \gap{x} ist verärgert.
\item[c] Die Chimäre
\end{ekey}
Wir könnten dann den Satz \ref{chimera1} als $\atom{A}{c}$ und den Satz \ref{chimera2} als $\enot \atom{A}{c}$ symbolisieren.

Probleme entstehen aber, wenn wir fragen, ob diese Sätze wahr oder falsch sind. Eine Möglichkeit ist zu sagen, dass Satz \ref{chimera1} nicht wahr ist, weil es keine Chimäre gibt. Wenn Satz \ref{chimera1} falsch ist, weil er von einer nicht existierende Sache handelt, dann ist Satz \ref{chimera2} aus dem gleichen Grund auch falsch. Dies würde jedoch bedeuten, dass $\atom{A}{c}$ und $\enot \atom{A}{c}$ beide falsch wären. Angesichts der Wahrheitsbedingungen für die Negation kann dies jedoch nicht der Fall sein.

Was sollen wir tun, angesichts der Tatsache, dass wir nicht sagen können, dass beide Sätze falsch sind? Eine andere Möglichkeit ist zu sagen, dass der Satz \ref{chimera1} \emph{sinnlos} ist, weil er von einer nicht existierenden Sache handelt. Also wäre $\atom{A}{c}$ ein sinnvoller Ausdruck der LEO relativ zu einigen Symbolisierungsschlüsseln, aber nicht relativ zu anderen. Doch dies würde unsere formale Sprache zur Geisel bestimmter Symbolisierungsschlüssel machen. Da wir an der Form der Sätze unserer Sprache interessiert sind, wollen wir die logische Kraft eines Satzes wie $\atom{A}{c}$ unabhängig von einem bestimmten Symbolisierungsschlüssel betrachten. Wenn $\atom{A}{c}$ manchmal sinnvoll und manchmal sinnlos wäre, könnten wir das aber nicht tun.

Das ist das \emph{Problem der leeren Terme}, und wir werden später noch einmal darauf zurückkommen (siehe S.~\pageref{subsec.defdesc}.) Der wichtige Punkt für jetzt ist, dass jeder Name der LEO auf etwas in der Domäne verweisen \emph{muss}, obwohl die Domäne welche Dinge auch immer enthalten kann, die wir wollen. Wenn wir Argumente über mythologische Wesen symbolisieren wollen, dann müssen wir eine Domäne definieren, die sie miteinschlie{\ss}t und laut der sie eben schon existieren. Dies zu tun ist wichtig, wenn wir die Logik von (literarischen) Erzählungen berücksichtigen wollen. Wir können einen Satz wie `Sherlock Holmes lebte in der Baker Street 221B' symbolisieren, indem wir fiktionale Figuren wie Sherlock Holmes in unsere Domäne aufnehmen.

\chapter{Sätze mit einem Quantor}
\label{s:MoreMonadic}

Wir haben jetzt alle Teile der LEO beisammen. Um kompliziertere Sätze zu symbolisieren, muss man nur wissen, wie man Prädikate, Namen, Quantoren und Junktoren kombiniert. 

\section{Gängige Quantorenphrasen}
Betrachten Sie die folgenden Sätze:
	\begin{earg}
		\item[\ex{quan1}] Jede Münze in meiner Tasche ist ein Euro.
		\item[\ex{quan2}] Eine Münze auf dem Tisch ist ein Schilling.
		\item[\ex{quan3}] Nicht alle Münzen auf dem Tisch sind Schillinge.
		\item[\ex{quan4}] Keine der Münzen in meiner Tasche sind Schillinge.
	\end{earg}
Bei der Bereitstellung eines Symbolisierungsschlüssels müssen wir eine Domäne angeben. Da wir über Münzen in meiner Tasche und auf dem Tisch sprechen, muss die Domäne zumindest alle diese Münzen enthalten. Da wir von nichts anderem als von Münzen sprechen, lassen wir die Domäne alle Münzen sein. Da es sich nicht um bestimmte Münzen handelt, brauchen wir uns auch nicht mit Namen zu befassen. Hier ist also unser Schlüssel:
	\begin{ekey}
		\item[\text{Domäne}] Alle Münzen
		\item[\atom{T}{x}] \gap{x} ist in meiner Tasche % P -> T, T -> T_2, Q -> F, D -> S
		\item[\atom{T_2}{x}] \gap{x} ist auf dem Tisch
		\item[\atom{E}{x}] \gap{x} ist ein Euro
		\item[\atom{S}{x}] \gap{x} ist ein Schilling
	\end{ekey}
Satz \ref{quan1} wird am natürlichsten mittels eines Universalquantors symbolisiert. Der Universalquantor sagt etwas über alles in der Domäne aus, nicht nur über die Münzen in meiner Tasche. Satz \ref{quan1} kann wie folgt umschrieben werden: `Für jede Münze, wenn diese Münze in meiner Tasche ist, dann ist sie ein Euro'. Wir können ihn also als `$\forall x(\atom{T}{x} \eif \atom{E}{x})$'. 

Da es in Satz \ref{quan1} um Münzen geht, die sich sowohl in meiner Tasche befinden, als auch um Schillinge, könnte es verlockend sein, ihn mit einer Konjunktion zu symbolisieren. Der Satz `$\forall x(\atom{T}{x} \eand \atom{E}{x})$' würde jedoch den Satz `Jede Münze ist sowohl ein Euro als auch in meiner Tasche' symbolisieren. Dies bedeutet jedoch offensichtlich etwas ganz anderes als der Satz \ref{quan1}. Daher sagen wir:
	\factoidbox{
		Ein Satz kann als $\forall x (\atom{\metav{F}}{x} \eif \atom{\metav{G}}{x})$ symbolisiert werden, wenn er im Deutschen als `Alle $F$s sind $G$s' oder `Jede/r/s $F$ ist ein/e $G$' paraphrasiert werden kann.
	}

Satz \ref{quan2} wird am natürlichsten mit einem Existenzquantor symbolisiert.
er kann als `es gibt eine Münze, die auf dem Tisch liegt und ein Schilling ist' umschrieben werden. Also können wir ihn als `$\exists x(\atom{T}{x} \eand \atom{D}{x})$' symbolisieren.

Mit dem Universalquantor nutzten wir ein Konditional, mit dem Existenzquantor hingegen eine Konjunktion. Nehmen Sie an, wir hätten stattdessen `$\exists x(\atom{T}{x} \eif \atom{D}{x})$' geschrieben. Dies bedeutet, dass es ein Objekt in der Domäne gibt, welches `$(\atom{T}{x} \eif \atom{D}{x})$' erfüllt. Aber in der WFL ist $\metav{A} \eif \metav{B}$ äquivalent zu $\enot\metav{A} \eor \metav{B}$. Diese Äquivalenz gilt auch in der LEO. Also ist `$\exists x(\atom{T}{x} \eif \atom{D}{x})$' wahr, wenn ein Objekt in der Domäne ist, welches `$(\enot \atom{T}{x} \eor \atom{D}{x})$' erfüllt. Und das bedeutet: `$\exists x (\atom{T}{x} \eif \atom{D}{x})$' ist wahr, wenn irgendeine Münze \emph{entweder} nicht auf dem Tisch ist \emph{oder} ein Schilling ist. Doch es gibt sehr viele Münzen, die nicht auf dem Tisch sind. Daher ist es \emph{sehr leicht} für `$\exists x(\atom{T}{x} \eif \atom{D}{x})$' wahr zu sein. Ein Konditional ist normalerweise der beste Junktor für den Universalquantor, aber ein Konditional im Geltungsbereich eines Existenzquantors sagt etwas aus, das oftmals zu schwach ist. Als allgemeine Faustregel gilt, dass Sie das Konditional nicht im Geltungsbereich eines Existenzquantors nutzen sollten, au{\ss}er Sie sind sich sicher, dass Sie es brauchen.
	\factoidbox{
		Ein Satz kann als $\exists x (\atom{\metav{F}}{x} \eand \atom{\metav{G}}{x})$ symbolisiert werden, wenn er im Deutschen als `Ein/e $F$ ist $G$' umschrieben werden kann.
	}		

Satz \ref{quan3} kann als `Es ist nicht der Fall, dass jede Münze auf dem Tisch ein Schilling ist' paraphrasiert werden. Also können wir ihn als `$\enot \forall x(\atom{T}{x} \eif \atom{D}{x})$' symbolisieren. Sie können \ref{quan3} auch als `Eine Münze auf dem Tisch ist kein Schilling' umschreiben. Sie würden ihn dann als `$\exists x(\atom{T}{x} \eand \enot \atom{D}{x})$' symbolisieren. Obwohl es Ihnen vielleicht noch nicht klar ist, sind diese zwei Symbolisierungen äquivalent. (Dies ist der Fall, weil $\enot\forall x\,\metav{A}$ und $\exists x\enot\metav{A}$, sowie $\enot(\metav{A}\eif\metav{B})$ und $\metav{A}\eand\enot\metav{B}$ äquivalent sind.)

Satz \ref{quan4} kann als `Es ist nicht der Fall, dass ein Schilling in meiner Tasche ist' umschrieben werden. Dies kann wiederum als `$\enot\exists x(\atom{P}{x} \eand \atom{D}{x})$' symbolisiert werden. Den Satz können wir auch als `Alles in meiner Tasche ist kein Schilling' paraphrasieren und dann mittels `$\forall x(\atom{P}{x} \eif \enot \atom{D}{x})$' symbolisiert werden. Auch hier gilt, dass die zwei Symbolisierungen äquivalent sind; beide sind korrekte Symbolisierungen von \ref{quan4}.

\factoidbox{
	Ein Satz, der im Deutschen als `Kein $F$ ist $G$' paraphrasiert werden kann, kann als $\enot\exists x (\atom{\metav{F}}{x} \eand \atom{\metav{G}}{x})$ und als $\forall x (\atom{\metav{F}}{x} \eif \enot\atom{\metav{G}}{x})$ symbolisiert werden.
}

Zuletzt wenden wir uns dem Wort `nur' zu, z.B.\@ in:
\begin{earg}
	\item[\ex{quan5}] Nur Schillinge sind auf dem Tisch.
\end{earg}
Wie sollen wir dies symbolisieren? Eine gute Strategie ist, uns zu überlegen, unter welchen Umständen der Satz falsch ist. Wenn wir \ref{quan5} nutzen, schlie{\ss}en wir all jene Fälle aus, in denen etwas auf dem Tisch ist, das kein Schilling ist. Also können wir den Satz genau so symbolisieren, wie wir auch `Keine Dinge, die keine Schillinge sind, sind auf dem Tisch' symbolisieren können. Die möglichen Symbolisierungen sind also: `$\enot\exists x(\atom{T}{x} \eand \enot \atom{D}{x})$' oder `$\forall x(\atom{T}{x} \eif \enot\enot \atom{D}{x})$'. Da sich doppelte Negationen aufheben, ist die zweite Option äquivalent zu `$\forall x(\atom{T}{x} \eif \atom{D}{x})$'. D.h.\@ `Nur Schillinge sind am Tisch' und `Alles auf dem Tisch ist ein Schilling' erhalten die gleiche Symbolisierung.

\factoidbox{
	Ein Satz der im Deutschen als `Nur $F$s sind $G$s' paraphrasiert werden kann, kann also $\enot\exists x (\atom{\metav{G}}{x} \eand \enot\atom{\metav{F}}{x})$ oder $\forall x (\atom{\metav{G}}{x} \eif \atom{\metav{F}}{x})$ symbolisiert werden.
}

\section{Leere Prädikate}

In \S\ref{s:FOLBuildingBlocks} betonten wir, dass ein Name auf genau ein Objekt in der Domäne verweisen muss. Im Gegensatz dazu muss ein Prädikat auf nichts in der Domäne zutreffen. Ein Prädikat, das auf nichts in der Domäne zutrifft, nenn wir ein \define{leeres Prädikat}.

\newglossaryentry{empty predicate}{
  name = {leeres Prädikat},
  description = {Ein \gls{Prädikat}, das auf kein Objekt in der \gls{Domäne} zutrifft}
}

Nehmen Sie an, wir wollen diese beiden Sätze symbolisieren:
	\begin{earg}
		\item[\ex{monkey1}] Jede Maus versteht die deutsche Sprache.
		\item[\ex{monkey2}] Eine Maus versteht die deutsche Sprache.
	\end{earg}
Auf folgende Weise können wir den Symbolisierungsschlüssel für diese Sätze aufschreiben:
	\begin{ekey}
		\item[\text{Domäne}] Tiere
		\item[\atom{M}{x}] \gap{x} ist eine Maus.
		\item[\atom{S}{x}] \gap{x} versteht die deutsche Sprache.
	\end{ekey}
Satz \ref{monkey1} kann nun als `$\forall x(\atom{M}{x} \eif \atom{S}{x})$' symbolisiert werden und Satz \ref{monkey2} als `$\exists x(\atom{M}{x} \eand \atom{S}{x})$'.

Es ist verlockend zu sagen, dass \ref{monkey1} \ref{monkey2} zur Folge hat. Wir könnten also denken, dass es unmöglich ist, dass jede Maus die deutsche Sprache versteht, wenn es keine Maus gibt, die das tut. Aber das wäre ein Fehler. Der Satz `$\forall x(\atom{M}{x} \eif \atom{S}{x})$'  kann wahr sein, auch wenn der Satz `$\exists x(\atom{M}{x} \eand \atom{S}{x})$' falsch ist.

Wie kann das sein? Die Antwort erhalten wir, indem wir uns vor Augen führen, was passieren würde, \emph{wenn es keine Mäuse geben würde}. In dem Fall wäre `$\forall x(\atom{M}{x} \eif \atom{S}{x})$' \emph{nichtssagenderweise} wahr: nehmen Sie jede Maus, die Sie wollen---sie versteht die deutsche Sprache! Aber wenn es keine Mäuse (in der Domäne) geben würde, dann wäre `$\exists x(\atom{M}{x} \eand \atom{S}{x})$' falsch.

Hier ist ein weiteres Beispiel. Lasst uns den obigen Symbolisierungsschlüssel erweitern:
	\begin{ekey}
		\item[\atom{R}{x}] \gap{x} ist ein Rotor
	\end{ekey}
Nun, betrachten wir `$\forall x(\atom{R}{x} \eif \atom{M}{x})$'. Dieser Satz symbolisiert `Jeder Rotor ist eine Maus'. Er ist wahr, unserem Symbolisierungsschlüssel nach, obwohl wir nicht sagen wollen, dass es eine ganze Reihe von Rotormäusen gibt. Denn `$\forall x(\atom{R}{x} \eif \atom{M}{x})$' ist wahr genau dann, wenn jedes Objekt in der Domäne, das ein Rotor ist, auch eine Maus ist. Doch weil die Domäne nur \emph{Tiere} beinhaltet, gibt es keine Rotoren in der Domäne. Daher ist der Satz \emph{nichtssagenderweise} wahr. 

Wenn Sie den Satz `Jeder Rotor ist eine Maus' tatsächlich symbolisieren würden, dann würden Sie wahrscheinlich Maschinen zur Domäne hinzufügen. Denn das Prädikat `$R$' wäre dann nicht leer und `$\forall x(\atom{R}{x} \eif \atom{M}{x})$' wäre falsch.
	\factoidbox{
		Wenn $\metav{F}$ ein leeres Prädikat ist, dann ist jeder Satz $\forall x (\atom{\metav{F}}{x} \eif \ldots)$ nichtssagenderweise wahr.
	}

\section{Domäne wählen}
Die angemessene Symbolisierung eines deutschen Satzes in der LEO hängt vom Symbolisierungsschlüssel ab. Die Wahl eines Schlüssels kann schwierig sein. Nehmen Sie an, wir wollen den folgenden Satz symbolisieren:
	\begin{earg}
		\item[\ex{pickdomainrose}] Jede Rose hat einen Dorn.
	\end{earg}
Wir könnten diesen Symbolisierungsschlüssel anbieten:
	\begin{ekey}
		\item[\atom{R}{x}] \gap{x} ist eine Rose
		\item[\atom{D}{x}] \gap{x} hat einen Dorn % T -> D
	\end{ekey}
Es ist verlockend zu sagen, dass wir \ref{pickdomainrose} als `$\forall x(\atom{R}{x} \eif \atom{D}{x})$' symbolisieren sollen, aber noch haben wir keine Domäne gewählt. Wenn die Domäne alle Rosen enthält, dann wäre dies eine gute Symbolisierung. Aber wenn die Domäne nur \emph{Objekte in meiner Küche} enthält, dann würde `$\forall x(\atom{R}{x} \eif \atom{D}{x})$' nur aussagen, dass jede Rose \emph{in meiner Küche} einen Dorn hat. Wenn es nun aber keine Rosen in meiner Küche gibt, dann ist dieser Satz nichtssagenderweise wahr. Das wollen wir nicht. Um \ref{pickdomainrose} adäquat zu symbolisieren, müssen wir alle Rosen in der Domäne einschlie{\ss}en. An dieser Stelle haben wir die Wahl zwischen zwei Optionen. 

Erstens können wir die Domäne so beschränken, dass sie \emph{nur} Rosen beinhaltet. Dann können wir den Satz \ref{pickdomainrose} als `$\forall x\,\atom{D}{x}$' symbolisieren. Diese Symbolisierung ist wahr genau dann, wenn jedes Objekt in der Domäne einen Dorn hat; weil die Domäne nur Rosen beinhaltet, ist dies wahr genau dann, wenn jede Rose einen Dorn hat. Indem wir die Domäne beschränken, können wir unseren deutschen Satz mit einem sehr kurzen und einfachen Satz der LEO symbolisieren. Dieser Ansatz kann uns also Arbeit ersparen, wenn alle Sätze, die wir symbolisieren wollen, nur von Rosen handeln.

Zweitens können wir die Domäne auch andere Objekte als Rosen beinhalten lassen: Rhododendren, Ratten, Gewehre, was auch immer. Diesen Ansatz müssen wir wählen, wenn wir mit Hilfe des gleichen Symbolisierungsschlüssels auch weitere Sätze symbolisieren wollen:
	\begin{earg}
		\item[\ex{pickdomaincowboy}] Jeder Cowboy singt ein trauriges Lied.
	\end{earg}
Unsere Domäne muss nun sowohl alle Rosen (für Satz \ref{pickdomainrose}) und alle Cowboys (für Satz \ref{pickdomaincowboy}) beinhalten. Also könnten wir den folgenden Symbolisierungsschlüssel anbieten:
	\begin{ekey}
		\item[\text{Domäne}] Personen und Pflanzen
		\item[\atom{C}{x}] \gap{x} ist ein Cowboy
		\item[\atom{S}{x}] \gap{x} singt ein trauriges Lied
		\item[\atom{R}{x}] \gap{x} ist eine Rose
		\item[\atom{D}{x}] \gap{x} hat einen Dorn % T -> D
	\end{ekey}
Nun werden wir den Satz \ref{pickdomainrose} mit `$\forall x (\atom{R}{x} \eif \atom{D}{x})$' symbolisieren, da `$\forall x\, \atom{D}{x}$' den Satz `Jede Person und Pflanze hat einen Dorn' symbolisieren würde. Ähnlicherweise werden wir den Satz \ref{pickdomaincowboy} als `$\forall x (\atom{C}{x} \eif \atom{S}{x})$' symbolisieren. 

Im Allgemeinen kann der Universalquantor verwendet werden, um den deutschen Ausdruck `Jedermann' zu symbolisieren, wenn die Domäne nur Personen enthält. Wenn es in der Domäne Menschen und andere Dinge gibt, dann muss `jedermann' als `jede Person' behandelt werden.
%Den Satz finde ich schwer verständlich, aber einen besseren habe ich auch nicht.


\section{Der Nutzen von Paraphrasen}
Wenn Sie deutsche Sätze in der LEO symbolisieren, ist es wichtig, die Struktur der Sätze zu verstehen, die Sie symbolisieren möchten. Was zählt ist die abschlie{\ss}ende Symbolisierung in der LEO. Manchmal werden Sie in der Lage sein, von einem deutschen Satz direkt zu einem Satz der LEO überzugehen. Andere Male hilft es, den deutschen Satz ein oder mehrere Male zu paraphrasieren. Jede nachfolgende Paraphrase sollte vom ursprünglichen Satz näher an etwas herankommen, das Sie leicht direkt in der LEO symbolisieren können.

Für die nächsten Beispiele werden wir diesen Symbolisierungsschlüssel verwenden:
	\begin{ekey}
		\item[\text{Domäne}] Personen
		\item[\atom{B}{x}] \gap{x} ist ein Bassist.
		\item[\atom{R}{x}] \gap{x} ist ein Rockstar.
		\item[k] Kim Deal
	\end{ekey}
Betrachten wir nun diese Sätze:
	\begin{earg}
		\item[\ex{pronoun1}] Wenn Kim Deal eine Bassistin ist, dann ist sie ein Rockstar.
		\item[\ex{pronoun2}] Wenn eine Person eine Bassistin ist, dann ist sie ein Rockstar.
	\end{earg}
Die Konsequenten in \ref{pronoun1} und \ref{pronoun2} beinhalten die gleichen Worte (`$\ldots$ ist sie ein Rockstar'), aber sie haben sehr verschiedene Bedeutungen. Um dies klarzustellen, hilft es die ursprünglichen Sätze umzuschreiben, insbesondere die Pronomina.

Satz \ref{pronoun1} kann als `Wenn Kim Deal eine Bassistin ist, dann ist \emph{Kim Deal} ein Rockstar'. Dies kann als `$\atom{B}{k} \eif \atom{R}{k}$' symbolisiert werden.

Satz \ref{pronoun2} muss anders umschrieben werden `Wenn eine Person eine Bassistin ist, dann ist \emph{diese Person} ein Rockstar'. Dieser Satz beschäftigt sich nicht mit einer bestimmten Person, also brauchen wir eine Variable. In einem Zwischenschritt können wir den Satz als `Für jede Person x gilt: wenn x eine Bassistin ist, dann ist x ein Rockstar'. Dies kann nun als `$\forall x (\atom{B}{x} \eif \atom{R}{x})$' symbolisiert werden. Und das ist der gleiche Satz, den wir auch benutzt hätten, um `Jede Bassist*in ist ein Rockstar' darzustellen. Das ist klarerweise wahr genau dann, wenn Satz \ref{pronoun2} wahr ist; so wie erhofft.

Betrachten Sie nun ein paar weitere Sätze:
	\begin{earg}
		\item[\ex{anyone1}] Wenn jemand ein/e Bassist*in ist, dann ist Kim Deal ein Rockstar.
		\item[\ex{anyone2}] Wenn jemand ein/e Bassist*in ist, dann ist er/sie ein Rockstar.
	\end{earg}
Hier kommen in das Antezedens von \ref{anyone1} und \ref{anyone2} die gleichen Worte vor (`Wenn jemand eine Bassist*in ist$\ldots$'). Aber es ist nicht einfach zu wissen, wie diese beiden Vorkommnisse zu symbolisieren sind. Auch hier kommt uns die Paraphrase zu Hilfe. 

Satz \ref{anyone1} kann als `Wenn es eine Bassist*in gibt, dann ist Kim Deal ein Rockstar' dargestellt werden. Nun ist klar, dass der Satz ein Konditional ist, dessen Antezedens einen Quantor enthält. Also können wir den ganzen Satz mit einem Konditional als Hauptjunktor symbolisieren: `$\exists x \atom{B}{x} \eif \atom{R}{k}$'.

Satz \ref{anyone2} kann als `Für alle Personen $x$ gilt: wenn $x$ ein/e Bassist*in ist, dann ist $x$ ein Rockstar'. Oder, in schönerem Deutsch: `Alle Bassist*innen sind Rockstars'. Dies ist am besten als `$\forall x(\atom{B}{x} \eif \atom{R}{x})$' zu symbolisieren, genau wie Satz \ref{pronoun2}.

Das Prinzip hier ist, dass das deutsche Wort `jemand' typischerweise mit Hilfe eines Quantors symbolisiert werden sollte. Wenn es Ihnen schwer fällt, zu entscheiden, ob Sie einen Existenz- oder Universalquantor verwenden sollen, versuchen Sie, den Satz mit einem deutschen Satz zu paraphrasieren, der `jemand' durch ein anderes Wort ersetzt.

\section{Quantoren und Geltungsbereich}
Betrachten wir die folgenden Sätze:
	\begin{earg}
		\item[\ex{qscope1}] Wenn alle Bassist*innen sind, dann ist Lars ein Bassist
		\item[\ex{qscope2}] Alle sind so, dass, wenn sie Bassist*innen sind, auch Lars ein Bassist ist.
	\end{earg}
Um diese Sätze zu symbolisieren, fügen wir einen Namen zu unserem Symbolisierungsschlüssel hinzu:
	\begin{ekey}
		\item[l] Lars
	\end{ekey}
Satz \ref{qscope1} ist ein Konditional, dessen Antezedens `Alle sind Bassist*innen' ist. Also symbolisieren wir diesen Satz als `$\forall x\, \atom{B}{x} \eif \atom{B}{l}$'. Dieser Satz ist \emph{notwendigerweise} wahr: wenn \emph{alle} Bassist*innen sind, dann können wir irgendeine Person nehmen---zum Beispiel Lars---und sie wir ein/e Bassist*in sein. 

Satz \ref{qscope2} hingegen ist besser umschrieben als `Jede Person $x$ ist so, dass gilt: wenn $x$ ein/e Bassist*in ist, dann ist Lars ein Bassist'. Das symbolisieren wir als `$\forall x (\atom{B}{x} \eif \atom{B}{l})$'. Dieser Satz kann falsch sein; Kim Deal ist eine Bassist*in. Also ist `$\atom{B}{k}$' wahr. Nehmen wir nun an, dass Lars kein Bassist ist (er ist Schlagzeuger). Dann ist `$\atom{B}{l}$' falsch. Dementsprechend ist nun `$\atom{B}{k} \eif \atom{B}{l}$' falsch und somit auch `$\forall x (\atom{B}{x} \eif \atom{B}{l})$'. 

Kurz gesagt: `$\forall x \atom{B}{x} \eif \atom{B}{l}$' und `$\forall x (\atom{B}{x} \eif \atom{B}{l})$' sind sehr unterschiedliche Sätze. Der Unterschied zwischen ihnen beruht in einem Unterschied der \emph{Geltungsbereiche} ihrer Quantoren. Der Geltungsbereich eines Quantors ähnelt dem der Negation, welchen wir uns in unserer Diskussion der WFL angesehen hatten.

Im Satz `$\enot \atom{B}{k} \eif \atom{B}{l}$' ist der Geltungsbereich von `$\enot$' nur das Antezedens des Konditionals. Wir sagen so etwas wie: wenn `$\atom{B}{k}$' falsch ist, dann ist `$\atom{B}{l}$' wahr. Ähnlich verhält sich der Satz `$\forall x \atom{B}{x} \eif \atom{B}{l}$'. Auch hier ist der Geltungsbereich von `$\forall x$' nur das Antezedens des Konditionals. Wir sagen so etwas wie: wenn `$\atom{B}{x}$' auf \emph{alle} zutrifft, dann ist auch `$\atom{B}{l}$' wahr. 

Die Moral hier ist einfach: wenn Sie Konditionale verwenden, stellen Sie sicher, dass sie den Geltungsbereich ihrer Junktoren und Quantoren richtig zugeordnet haben. 

\subsection{Mehrdeutige Prädikate}

Wenden wir uns dem folgenden Satz zu:
\begin{earg}
\item[\ex{surgeon1}] Adina ist eine erfahrene Chirurgin.
\end{earg}
Die Domäne besteht aus Personen; $\atom{K}{x}$ symbolisiert `$x$ ist ein/e erfahrene/r Chirurg*in'; $a$ verweist auf Adina. Satz \ref{surgeon1} symbolisieren wir dann als $\atom{K}{a}$.

Was, wenn wir stattdessen das folgende Argument symbolisieren wollen?
\begin{quote}
Das Krankenhaus stellt nur erfahrene Chirurg*innen ein. Alle Chirurg*innen sind gierig. Billy ist ein Chirurg, aber er ist nicht erfahren. Deshalb ist Billy gierig, aber das Krankenhaus wird ihn nicht einstellen.
\end{quote}
Wir müssen hier zwischen \emph{erfahrenen Chirurg*innen} und \emph{Chirurg*innen} unterscheiden. Das tun wir im folgenden Symbolisierungsschlüssel:
\begin{ekey}
\item[\text{Domäne}] Personen
\item[\atom{G}{x}] \gap{x} ist gierig.
\item[\atom{K}{x}] Das Krankenhaus stellt \gap{x} ein. % H -> K
\item[\atom{C}{x}] \gap{x} ist ein/e Chirurg*in. % R -> C
\item[\atom{E}{x}] \gap{x} ist erfahren. % K -> E
\item[b] Billy
\end{ekey}

Dieser Schlüssel erlaubt uns das Argument wie folgt zu symbolisieren:
\begin{earg}
\label{surgeon2}
\item[] $\forall x\bigl[\enot (\atom{C}{x} \eand \atom{E}{x}) \eif \enot \atom{K}{x}\bigr]$
\item[] $\forall x(\atom{C}{x} \eif \atom{G}{x})$
\item[] $\atom{C}{b} \eand \enot \atom{E}{b}$
\item[\therefore] $\atom{G}{b} \eand \enot \atom{K}{b}$
\end{earg}

So weit, so gut. Doch das folgende Beispiel bringt einige Komplikationen zu Tage:
\begin{quote}
\label{surgeon3}
Carola ist eine erfahrene Chirurgin und eine Tennisspielerin. Also ist Carola eine erfahrene Tennisspielerin.
\end{quote}
Wenn wir mit dem Symbolisierungsschlüssel beginnen, den wir für das vorangegangene Argument verwendet haben, könnten wir einfach ein Prädikat ($\atom{T}{x}$ für `$x$ ist ein/e Tennisspieler*in') und einen Namen ($c$ für Carola) hinzufügen. Dann symbolisieren wir das Argument so:
\begin{earg}
\item[] $(\atom{C}{c} \eand \atom{E}{c}) \eand \atom{T}{c}$
\item[\therefore] $\atom{T}{c} \eand \atom{E}{c}$
\end{earg}
Aber diese Symbolisierung ist eine Katastrophe! Das ursprüngliche deutsche Argument ist \emph{ungültig}, während die Symbolisierung ein \emph{gültiges} Argument der LEO ist. Das Problem ist, dass es einen Unterschied zwischen chirurgischer Erfahrung und Tenniserfahrung gibt: man kann erfahren \emph{als} Chirurg*in sein, ohne erfahren \emph{als} Tennisspieler*in zu sein. Um das Argument korrekt zu symbolisieren, brauchen wir zwei Prädikate, welches jeweils von einer bestimmten Art der Erfahrung handeln. Wenn wir $\atom{E_1}{x}$ für `$x$ ist erfahren als Chirurg*in' und $\atom{K_2}{x}$ für `$x$ ist erfahren als Tennisspieler*in' verwenden, dann können wir das Argument so symbolisieren:
\begin{earg}
\label{surgeon3correct}
\item[] $(\atom{C}{c} \eand \atom{E_1}{c}) \eand \atom{T}{c}$
\item[\therefore] $\atom{T}{c} \eand \atom{E_2}{c}$
\end{earg}
Wie das Argument der deutschen Sprache ist auch dieses Argument ungültig.

Man beachte hier, dass keine spezielle logische Verbindung zwischen $\atom{E_1}{c}$ und $\atom{R}{c}$ besteht. Als Symbole der LEO könnten sie beliebige einstellige Prädikate beinhalten. Im Deutschen dagegen gibt es eine Verbindung zwischen `ist ein/e Chirurg*in' und `ist ein/e erfahrene/r Chirurg*in': Jede erfahrene Chirurgin ist eine Chirurgin. Um diese Verbindung erfassen, symbolisieren wir `Carola ist eine erfahrene Chirurgin' als $\atom{C}{c} \eand \atom{E_1}{c}$. Zu Deutsch: `Carola ist eine Chirurgin und ist erfahren als Chirurgin.'

Die Moral dieser Beispiele ist, dass man sich davor hüten muss, Prädikate auf mehrdeutige Weise zu symbolisieren. Ähnliche Probleme können bei Prädikaten wie \emph{gut}, \emph{schlecht}, \emph{gro{\ss}} und \emph{klein} auftreten. So wie erfahrene Chirurg*innen und erfahrene Tennisspieler*innen unterschiedliche Erfahrungen haben, so sind gro{\ss}e Hunde, gro{\ss}e Mäuse und gro{\ss}e Probleme auf unterschiedliche Weise gro{\ss}.

Reicht es aus, \emph{ein} Prädikat zu haben, das `ist ein/e erfahrene/r Chirurg*in' symbolisiert, anstatt zweier Prädikate `ist erfahren' und `ist ein/e Chirurgin'? Manchmal. Wie der Satz \ref{surgeon1} zeigt, brauchen wir manchmal nicht zwischen erfahrenen Chirurg*innen und anderen Chirurg*innen zu unterscheiden. In anderen Fällen aber müssen wir diese Unterscheidung sehr wohl treffen. 

Ähnliches gilt auch für andere Prädikate. Wir müssen nicht immer zwischen verschiedenen Weisen, gut, schlecht oder gro{\ss} zu sein, unterscheiden. Wenn Sie ein Argument symbolisieren, bei dem es nur um Hunde geht, dann ist es in Ordnung, ein Prädikat zu definieren, das `ist gro{\ss}' symbolisiert. Wenn ihre Domäne jedoch Hunde und Mäuse umfasst, ist es wahrscheinlich am besten, das Prädikat feinkörniger zu definieren, sodass es `ist gro{\ss} für einen Hund' symbolisiert.

\practiceproblems
\problempart
\label{pr.BarbaraEtc}
Hier sind die syllogistischen Figuren, die von Aristoteles und seinen Nachfolgern identifiziert wurden, zusammen mit ihren mittelalterlichen Namen:
\begin{earg}
	\item \textbf{Barbara.} Alle Gs sind F. All Hs sind G. Also:  Alle Hs sind F.
	\item \textbf{Celarent.} Kein G ist F. Alle Hs sind G. Also: Kein H ist F
	\item \textbf{Ferio.} Kein G ist F. Ein H ist G. Also: Ein H ist kein F.
	\item \textbf{Darii.} Alle Gs sind F. Ein H ist G. Also: Ein H ist F.
	\item \textbf{Camestres.} Alle Fs sind G. Kein H ist G. Also: Keine Hs sind F.
	\item \textbf{Cesare.} Kein F ist G. Alle Hs sind G. Also: Kein H ist F.
	\item \textbf{Baroko.} Alle Fs sind G. Ein H ist kein G. Also: Ein H ist kein F.
	\item \textbf{Festino.} Kein F ist G. Ein H ist G. Also: Ein H ist kein F.
	\item \textbf{Datisi.} Alle Gs sind F. Ein G ist H. Also: Ein H ist F.
	\item \textbf{Disamis.} Ein G ist F. Alle Gs sind H. Also: Ein H ist F.
	\item \textbf{Ferison.} Kein G ist F. Ein G ist H. Also: Ein H ist kein F.
	\item \textbf{Bokardo.} Ein G ist kein F. Alle Gs sind H. Also: Ein H ist kein F.
	\item \textbf{Camenes.} Alle Fs sind G. Kein G ist H. Also: Kein H ist F.
	\item \textbf{Dimaris.} Ein F ist G. Alle Gs sind H. Also: Ein H ist F.
	\item \textbf{Fresison.} Kein F ist G. Ein G ist H. Also: Ein H ist kein F.
\end{earg}
Symbolisieren Sie jedes dieser Argumente in der LEO.

\

\problempart
\label{pr.FOLvegetarians}
Unter Verwendung des folgenden Symbolisierungsschlüssels:
\begin{ekey}
\item[\text{Domäne}] Personen
\item[\atom{K}{x}] \gap{x} kennt den Code
\item[\atom{S}{x}] \gap{x} ist ein Spion
\item[\atom{V}{x}] \gap{x} ist Vegetarier
%\item[\atom{T}{x,y}] \gap{x} trusts \gap{y}.
\item[h] Hofthor
\item[i] Ingmar
\end{ekey}
symbolisieren Sie die folgenden Sätze in der LEO:
\begin{earg}
\item Weder Hofthor noch Ingmar ist ein Vegetarier.
\item Kein Spion kennt den Code.
\item Niemand kennt den Code, es sei denn, Ingmar tut es.
\item Hofthor ist ein Spion, aber kein Vegetarier ist ein Spion.
\end{earg}
\solutions
\problempart\label{pr.FOLalligators}
Unter Verwendung des folgenden Symbolisierungsschlüssels:
\begin{ekey}
\item[\text{Domäne}] Tiere
\item[\atom{A}{x}] \gap{x} ist ein Alligator.
\item[\atom{M}{x}] \gap{x} ist eine Maus.
\item[\atom{R}{x}] \gap{x} ist ein Reptil.
\item[\atom{Z}{x}] \gap{x} lebt im Zoo.
\item[a] Amos
\item[b] Bouncer
\item[c] Cleo
\end{ekey}
symbolisieren Sie die folgenden Sätze in der LEO:
\begin{earg}
\item Amos, Bouncer und Cleo leben im Zoo. 
\item Bouncer ist ein Reptil, aber kein Alligator. 
\item Ein Reptil lebt im Zoo. 
\item Alle Alligatoren sind Reptilien.
\item Jedes Tier, das im Zoo lebt, ist entweder eine Maus oder ein Alligator. 
\item Es gibt Reptilien, die keine Alligatoren sind.
\item Wenn ein Tier ein Reptil ist, dann ist Amos eines.
\item Wenn ein Tier ein Alligator ist, dann ist es ein Reptil.
\end{earg}

\problempart
\label{pr.FOLarguments}
Für jedes der folgenden Argumente, schreiben Sie einen Symbolisierungsschlüssel und symbolisieren Sie das Argument.
\begin{earg}
\item Willard ist ein Logiker. Alle Logiker*innen tragen komische Hüte. Also trägt Willard komische Hüte.
\item Nichts auf meinem Schreibtisch entgeht meiner Aufmerksamkeit. Auf meinem Schreibtisch steht ein Computer. Es gibt also einen Computer, der meiner Aufmerksamkeit nicht entgeht.
\item Alle meine Träume sind schwarz-wei{\ss}. Alte Fernsehsendungen sind in schwarz-wei{\ss}. Deshalb sind einige meiner Träume alte Fernsehsendungen.
\item Weder Holmes noch Watson waren in Australien. Ein Mensch kann ein Känguru nur sehen, wenn er in Australien oder in einem Zoo ist. Watson hat zwar kein Känguru gesehen, aber Holmes schon. Deshalb war Holmes in einem Zoo.
\item Niemand erwartet die Spanische Inquisition. Niemand kennt das Leid, das ich gesehen habe. Deshalb kennt jeder, der die Spanische Inquisition erwartet, das Leid, das ich gesehen habe.
\item Alle Babys sind unlogisch. Niemand, der unlogisch ist, kann mit einem Krokodil umgehen. Berthold ist ein Baby. Deshalb kann Berthold nicht mit einem Krokodil umgehen.
%Wie bitte? :)
\end{earg}


\chapter{Mehrfache Allgemeinheit}\label{s:MultipleGenerality}
Bislang haben wir nur Sätze betrachtet, die einstellige Prädikate und einen Quantor erfordern. Die volle Leistungsfähigkeit der LEO fördern wir aber erst zu Tage, wenn wir beginnen, mehrstellige Prädikate und mehrere Quantoren zu verwenden. 

\section{Mehrstellige Prädikate}
Alle Prädikate, die wir bisher betrachtet haben, betrafen Eigenschaften, die Objekte alleine haben können. Diese Prädikate haben \emph{eine} Lücke, und um einen Satz zu bilden, müssen diese Lücke füllen. Es sind \define{einstellige} Prädikate.

Andere Prädikate betreffen jedoch die \emph{Beziehung} zwischen zwei Dingen. Hier sind einige Beispiele für Beziehungsprädikate im Deutschen:
	\begin{quote}
		\blank\ liebt \blank\\
		\blank\ ist links von \blank\\
		\blank\ ist verschuldet bei \blank
	\end{quote}
Dies sind \define{zweistellige} Prädikate. Sie haben zwei Lücken, und um einen Satz zu bilden, müssen wir diese Lücken füllen. Umgekehrt: wenn wir mit einem deutschen Satz mit mehreren singulären Termen anfangen, können wir zwei dieser Begriffe entfernen, um ein zweistelliges Prädikat zu erhalten. Betrachten wir den Satz `Vinnie hat das Familienauto von Nunzio geliehen'. Indem wir zwei singuläre Terme entfernen, können wir drei verschiedene zweistellige Prädikate erhalten:
	\begin{quote}
		Vinnie hat \blank\ von \blank geliehen\\
		\blank\ hat das Familienauto von \blank geliehen\\
		\blank\ hat \blank\ von Nunzio geliehen
	\end{quote}
und indem wir alle drei singuläre Terme entfernen, kriegen wir ein \define{dreistelliges} Prädikat:
	\begin{quote}
		\blank\ hat \blank\ von \blank geliehen
	\end{quote}
In der Tat gibt es keine prinzipielle Begrenzung der Zahl der Stellen, die ein Prädikat haben kann: wir können generell von \define{$n$stelligen} Prädikaten sprechen.

\section{Beachten Sie die Lücke(n)!}

Wir haben das Symbol `\blank' genutzt um die Lücke zu symbolisieren, die wir durch das Entfernen eines singulärer Terms aus einem Satz erhalten. Wie Frege aber schon im 19ten Jahrhundert betonte, müssen wir zwischen verschiedenen Lücken differenzieren. Um einen Satz zu erhalten, können wir Lücken mit den eben entfernten singulären Termen füllen, aber wir können sie ebenso mit anderen singulären Termen füllen, oder in einer anderen Reihenfolge. Die folgenden drei verschiedenen Sätze erhalten wir, indem wir die Lücken in `\blank\ liebt \blank{}' auf verschiedene Weisen füllen; aber jeder diese Sätze hat eine bestimmte Bedeutung:
\begin{earg}
	\item[\ex{terms3}] Karl liebt Imre.
	\item[\ex{terms3b}] Imre liebt Karl.
	\item[\ex{terms3a}] Karl liebt Karl.
\end{earg}
Wir müssen die Lücken in unseren Prädikaten im Auge behalten, sodass wir nachvollziehen können, wie wir sie füllen. Um das zu tun, ordnen wir ihnen Variablen zu. Nehmen wir an, wir wollen die obigen Sätze symbolisieren. Dann könnten wir mit dem folgenden Symbolisierungsschlüssel beginnen: 
	\begin{ekey}
		\item[\text{Domäne}] Personen
		\item[i] Imre
		\item[k] Karl
		\item[\atom{L}{x,y}] \gap{x} liebt \gap{y}
	\end{ekey}
Satz \ref{terms3} wird nun als `$\atom{L}{k,i}$' symbolisiert; Satz \ref{terms3b} als `$\atom{L}{i,k}$'; und Satz \ref{terms3a} als `$\atom{L}{k,k}$'. Hier sind ein paar weitere Sätze, die wir mit diesem Symbolisierungsschlüssel symbolisieren können:
\begin{earg}
	\item[\ex{terms4}] Imre liebt sich selbst.
	\item[\ex{terms5}] Karl liebt Imre, aber nicht umgekehrt.
	\item[\ex{terms6}] Karl wird von Imre geliebt.
\end{earg}
Satz \ref{terms4} kann `Imre liebt Imre' umschrieben, und daher als `$\atom{L}{i,i}$' symbolisiert, werden. Satz \ref{terms5} ist eine Konjunktion. Wir können diese als `Karl liebt Imre und Imre liebt Karl nicht' paraphrasieren, und somit als `$\atom{L}{k,i} \eand \enot \atom{L}{i,k}$' symbolisieren. Satz \ref{terms6} kann als `Imre liebt Karl' umschrieben, und dementsprechend als `$\atom{L}{i,k}$' symbolisiert werden. Im letzten Fall haben wir natürlich einen Unterschied im Fokus zwischen Aktiv und Passiv; aber die Wahrheitsbedingungen des Aktivs und des Passivs scheinen sich nicht zu unterscheiden.

Die Beziehung zwischen `Imre liebt Karl' und `Karl wird von Imre geliebt' hebt etwas Wichtiges hervor. Um das zu sehen, nehmen Sie an, wir fügen einen weiteren Eintrag zu unserem Symbolisierungsschlüssel hinzu:
\begin{ekey}
	\item[\atom{M}{x,y}] \gap{y} liebt \gap{x}
\end{ekey}
Der Eintrag für `$M$' nutzt genau das gleiche deutsche Wort---`liebt'---wie der Eintrag für `$L$'. \emph{Aber die Lücken wurden vertauscht!} (Schauen Sie auf die Subskripte.) Und das \emph{macht einen unterschied}.

Zur Erklärung: ein Satz wie `$\atom{L}{k,i}$' sagt uns, dass wir den \emph{ersten} Namen (`$k$') und seinen Wert (Karl) mit der `$x$'-Lücke verbinden, den \emph{zweiten} Namen aber (`$i$') und seinen Wert (Imre) mit der `$y$'-Lücke verbinden, und so \emph{Karl liebt Imre} erhalten. Der Satz `$\atom{M}{i,k}$' sagt uns hingegen, dass wir die `$x$'-Lücke mit dem Wert des \emph{ersten} Namens (`$i$', Imre) füllen, die `$y$'-Lücke mit dem Wert des \emph{zweiten} Namens (`$k$', Karl) füllen, und so \emph{Imre liebt Karl} erhalten. 

`$\atom{L}{i,k}$' und `$\atom{M}{k,i}$' symbolisieren also beide `Imre liebt Karl', wohingegen `$\atom{L}{k,i}$' und `$\atom{M}{i,k}$' `Karl liebt Imre' symbolisieren. Weil Liebe unerwidert bleiben kann, unterscheiden sich diese Sätze in ihren Wahrheitsbedingungen. 

Ein weiteres Beispiel ist vielleicht hilfreich. Fügen wir das folgende Prädikat zu unserem Symbolisierungsschlüssel hinzu:
\begin{ekey}
	\item[\atom{P}{x,y}] \gap{x} zieht \gap{x} \gap{y} vor
\end{ekey}
Nun symbolisiert `$\atom{P}{i,k}$' `Imre zieht Imre Karl vor' und `$\atom{P}{k,i}$' `Karl zieht Karl Imre vor'. Beachten Sie, das wir das gleiche Resultat auch mit dem folgenden Prädikat hätten erreichen können:
\begin{ekey}
	\item[\atom{P}{x,y}] \gap{x} zieht sich selbst \gap{y} vor
\end{ekey}
Die Lehre hier ist: \emph{wenn Sie sich mit mehrstelligen Prädikaten befassen, dann achten Sie sorgfältig auf die Reihenfolge der Lücken.} 


\section{Die Reihenfolge der Quantoren}\label{ss:OrderQuant}
Betrachten Sie den Satz: `Alle lieben jemanden'. Dieser Satz ist potenziell zweideutig. Er könnte eine der folgenden Bedeutungen haben:
	\begin{earg}
		\item[\ex{lovecycle}] Für jede Person $x$ gilt: $x$ liebt zumindest eine Person.
		\item[\ex{loveconverge}] Es gibt zumindest eine bestimmte Person, die von jeder Person geliebt wird.
	\end{earg}
Satz \ref{lovecycle} wird als `$\forall x \exists y\, \atom{L}{x,y}$' symbolisiert und würde auf ein Dreiecksverhältnis zutreffen. Nehmen wir beispielsweise an, dass die Domäne auf Imre, Juan und Karl beschränkt ist. Nehmen wir auch an, dass Karl Imre liebt, aber nicht Juan; dass Imre Juan liebt, aber nicht Karl; und, dass Juan Karl liebt, aber nicht Imre. Dann ist Satz \ref{lovecycle} wahr. 

Satz \ref{loveconverge} wird als `$\exists y \forall x\, \atom{L}{x,y}$' symbolisiert. Er ist in der gerade beschriebenen Situation \emph{nicht} wahr. Nehmen wir wieder an, dass unsere Domäne auf Imre, Juan und Karl beschränkt ist. Dann müssen Juan, Imre und Karl allesamt zumindest ein Liebesobjekt teilen, damit der Satz wahr ist. 

Dieses Beispiel veranschaulicht den Punkt, dass die Reihenfolge der Quantoren wichtig ist. Quantoren zu vertauschen ist ein \emph{Fehlschluss}. Hier ist ein Beispiel dieses Fehlschlusses, der manchmal in der Philosophie vorkommt:
	\begin{earg}
		\item[] Für jede Person gilt: es gibt zumindest eine Wahrheit, die sie nicht wissen kann. \hfill ($\forall \exists$)
		\item[\therefore] Es gibt zumindest eine Wahrheit, die keine Person wissen kann. \hfill ($\exists \forall$)
	\end{earg}
Diese Argumentstruktur ist offensichtlich ungültig. Vergleiche:
	\begin{earg}
		\item[] Jede/r hat eine Steuer-ID. \hfill ($\forall \exists$)
		\item[\therefore] Es gibt zumindest eine Steuer-ID, die jede/r hat. \hfill ($\exists \forall$)
	\end{earg}

   
Die Reihenfolge der Quantoren ist auch in mathematischen Definitionen wichtig. Es gibt beispielsweise einen gro{\ss}en Unterschied zwischen der punktweisen und gleichmä{\ss}igen Stetigkeit von Funktionen:
\begin{itemize}
\item Eine Funktion $f$ ist \emph{punktweise stetig} wenn
\[
\forall \epsilon\forall x\forall y\exists \delta(\left|x - y\right| < \delta \to \left|\atom{F}{x} - f(y)\right| < \epsilon)
\]
\item Eine Funktion $f$ ist \emph{gleichmä{\ss}ig stetig} wenn
\[
\forall \epsilon\exists \delta\forall x\forall y(\left|x - y\right| < \delta \to \left|\atom{F}{x} - f(y)\right| < \epsilon)
\]
\end{itemize}

\section{Schritte zur Symbolisierung}
Symbolisierungen in der LEO können etwas kompliziert werden. Wenn Sie also einen Satz symbolisieren wollen, dann sollten sie bestimmten Schritten folgen. Wie immer ist das am Besten anhand eines Beispiels gezeigt. Betrachten Sie den folgenden Symbolisierungsschlüssel: 
\begin{ekey}
\item[\text{Domäne}] Personen und Delfine
\item[\atom{D}{x}] \gap{x} ist ein Delfin
\item[\atom{F}{x,y}] \gap{x} ist ein Freund von \gap{y}
\item[\atom{B}{x,y}] \gap{x} besitzt \gap{y} % O -> E
\item[g] Gerald
\end{ekey}
Nun versuchen wir die folgenden Sätze zu symbolisieren:
\begin{earg}
\item[\ex{dog2}] Gerald besitzt einen Delfin.
\item[\ex{dog3}] Jemand besitzt einen Delfin.
\item[\ex{dog4}] Alle Freude von Gerald besitzen Delfine.
\item[\ex{dog5}] Alle Delfinbesitzer*innen sind Freunde von Delfinbesitzer*innen.
\item[\ex{dog6}] Alle Freunde von Delfinbesitzer*innen besitzen einen Hund eines Freundes.
\end{earg}
Satz \ref{dog2} kann als `Es gibt zumindest einen Delfin, den Gerald besitzt' paraphrasiert werden. Das können wir dann als `$\exists x(\atom{D}{x} \eand \atom{B}{g,x})$' symbolisieren.

Satz \ref{dog3} kann als `Es gibt zumindest etwas, $y$, das einen Delfin besitzt'. Wenn wir den ersten Teil symbolisieren, erhalten wir `$\exists y(y\text{ besitzt einen Delfin})$'. Das Fragment, das uns übrig bleibt ist `$y$ besitzt einen Delfin' und ähnelt Satz \ref{dog2}, bis auf die Tatsache, dass es nicht von Gerald handelt. Also können wir Satz \ref{dog3} mittels:
$$\exists y \exists x(\atom{D}{x} \eand \atom{B}{y,x})$$
symbolisieren. Hier sollten wir kurz innehalten. Als Zwischenschritt unseres letzte Symbolisierung, schrieben wir `$\exists y(y\text{ besitzt einen Delfin})$'. Das ist \emph{weder} ein Satz der LEO \emph{noch} ein Satz der deutschen Sprache: das nutzt Symbole der LEO (`$\exists$', `$y$') und Teile des Deutschen (`besitzt einen Delfin'). Es ist nur ein \emph{Zwischenschritt} auf dem Weg zu einer vollständigen Symbolisierung des deutschen Satzes. Sie sollten es als ein bisschen grobe Ausarbeitung betrachten, auf einer Stufe mit den Kritzeleien, die Sie vielleicht geistesabwesend am Seitenrand dieses Buches zeichnen, während Sie sich auf ein Problem konzentrieren.  

Satz \ref{dog4} kann als `Jedes $x$, das ein Freund Geralds ist, besitzt einen Delfin' umschrieben werden. Mit unserer Zwischenschrittstrategie schreiben wir nun:
$$\forall x \bigl[\atom{F}{x,g} \eif x \text{ besitzt einen Delfin}\bigr]$$
Das Fragment, das uns übrig bleibt, ist `$x$ besitzt einen Delfin' und ähnelt Satz \ref{dog2}. Allerdings wäre es ein Fehler, einfach das Folgende zu schreiben:
$$\forall x \bigl[\atom{F}{x,g} \eif \exists x(\atom{D}{x} \eand \atom{B}{x,x})\bigr]$$
Denn hier hätten wir einen \emph{Variablenkonflikt}. Der Geltungsbereich des Universalquantors `$\forall x$' ist das ganze Konditional, also muss das `$x$' in `$\atom{D}{x}$' vom Universalquantor gebunden werden. Aber  `$\atom{D}{x}$' fällt auch in den Geltungsbereich des Existenzquantors `$\exists x$'; und daher sollte `$x$' in `$\atom{D}{x}$' vom Existenzquantor gebunden werden. Nun macht sich Verwirrung breit: über welches `$x$' reden wir? Der Satz ist im Besten Fall mehrdeutig (andernfalls Unsinn). Logiker*innen aber hassen Mehrdeutigkeit. Eine einzige Variable kann nicht von zwei Quantoren zur gleichen Zeit in Anspruch genommen werden.

Um mit unserer Symbolisierung fortzufahren, müssen wir eine andere Variable für unseren Existenzquantor wählen. Das, was wir wollen, ist so etwas wie:
$$\forall x\bigl[\atom{F}{x,g} \eif\exists z(\atom{D}{z} \eand \atom{B}{x,z})\bigr]$$
Das ist eine adäquate Symbolisierung von \ref{dog4}.

Satz \ref{dog5} kann als `Für jedes $x$, das einen Delfin besitzt, gibt es zumindest einen Delfinbesitzer, mit dem $x$ befreundet ist'. Unter Anwendung unserer Zwischenschritttaktik wird dies zu: 
$$\forall x\bigl[\mbox{$x$ besitzt einen Delfin}\eif\exists y(\mbox{$y$ besitzt einen Delfin}\eand \atom{F}{x,y})\bigr]$$
Wenn wir die Symbolisierung abschlie{\ss}en, erhalten wir:
$$\forall x\bigl[\exists z(\atom{D}{z} \eand \atom{B}{x,z})\eif\exists y\bigl(\exists z(\atom{D}{z} \eand \atom{B}{y,z})\eand \atom{F}{x,y}\bigr)\bigr]$$
Hier haben wir denselben Buchstaben `$z$' sowohl im Antezedens als auch im Konsequens des Konditionals verwendet, obwohl diese zwei Vorkommnisse von zwei unterschiedlichen Quantoren gebunden werden. Das passt: Es gibt hier keinen Konflikt, weil klar ist, welcher Quantor mit welcher Variable verbunden ist. Wir können die Geltungsbereiche der Quantoren so darstellen:
$$\overbrace{\forall x\bigl[\overbrace{\exists z(\atom{D}{z} \eand \atom{B}{x,z})}^{\text{1ster `}\exists z\text{'}}\eif \overbrace{\exists y(\overbrace{\exists z(\atom{D}{z} \eand \atom{B}{y,z})}^{\text{2ter `}\exists z\text{'}}\eand \atom{F}{x,y})\bigr]}^{\text{`}\exists y\text{'}}}^{\text{`}\forall x\text{'}}$$
Dies zeigt, dass keine der Variablen von zwei Quantoren zur gleichen Zeit in Anspruch genommen wird.

Satz \ref{dog6} ist der bis dato schwierigste. Zuerst umschreiben wir ihn als `Jedes $x$, das ein Freund eines Delfinbesitzers ist, besitzt einen Delfin, den auch ein Freund von $x$ besitzt. Unter Anwendung unserer Zwischenschritttaktik wird dies:
\begin{multline*}
\forall x\bigl[x\text{ ist ein Freund eines Delfinbesitzers}\eif {}\\
x\text{ besitzt einen Delfin, den auch ein Freund von $x$ besitzt}\bigr]
\end{multline*}
Wir können das weiter aufteilen:
\begin{multline*}
	\forall x\bigl[\exists y(\atom{F}{x,y} \eand y\text{ besitzt einen Delfin})\eif {}\\
\exists y(\atom{D}{y} \eand \atom{B}{x,y} \eand y\text{ wird von einem Freund von $x$ besessen})\bigr]
\end{multline*}
Und weiter 
\begin{multline*}
\forall x\bigl[\exists y(\atom{F}{x,y} \eand \exists z(\atom{D}{z} \eand \atom{B}{y,z})) \eif {}\\
\exists y(\atom{D}{y} \eand \atom{B}{x,y} \eand \exists z(\atom{F}{z,x} \eand \atom{B}{z,y}))\bigr]
\end{multline*}
Und dann sind wir auch schon fertig.

\section{Unterdrückte Quantoren}\label{ss:SuppQuant}

Logik kann oft helfen, die Bedeutungen deutscher Ausdrücke klar zu stellen; insbesondere, wenn Quantoren implizit belassen werden oder ihre Reihenfolge mehrdeutig oder unklar bleibt. Die Klarheit des Ausdrucks und des Denkens, die die LEO ermöglicht, kann Ihnen einen erheblichen Vorteil in der Argumentation verschaffen. Dies illustriert die folgende Passage der britischen Philosophin Mary Astell (1666--1731), in der sie gegen ihren Zeitgenossen, den Theologen William Nicholls, argumentiert. In Discourse IV: The Duty of Wives to their Husbands in seinem \textit{The Duty of Inferiors towards their Superiors, in Five Practical Discourses} (London 1701), argumentierte Nicholls, das Frauen Männern natürlich unterlegen sind. Im Vorwort zur dritten Edition Ihres Buchs \emph{Some Reflections upon Marriage, Occasion'd by the Duke and Duchess of Mazarine's Case} antwortet Astell wie folgt:
\begin{quotation}
'Tis true, thro' Want of Learning, and of that Superior Genius which Men as Men lay claim to, she [Astell] was ignorant of the \textit{Natural Inferiority} of our Sex, which our Masters lay down as a Self-Evident and Fundamental Truth. She saw nothing in the Reason of Things, to make this either a Principle or a Conclusion, but much to the contrary; it being Sedition at least, if not Treason to assert it in this Reign. 

For if by the Natural Superiority of their Sex, they mean that \textit{every} Man is by Nature superior to \textit{every} Woman, which is the obvious meaning, and that which must be stuck to if they would speak Sense, it wou'd be a Sin in \textit{any} Woman to have Dominion over \textit{any} Man, and the greatest Queen ought not to command but to obey her Footman, because no Municipal Laws can supersede or change the Law of Nature; so that if the Dominion of the Men be such, the \textit{Salique Law,}\footnote{The Salique law was
the common law of France which prohibited the crown be passed on to female heirs.} as unjust as \textit{English Men} have ever thought it, ought to take place over all the Earth, and the most glorious
Reigns in the \textit{English, Danish, Castilian}, and other Annals, were wicked Violations of the Law of Nature!

If they mean that \textit{some} Men are superior to \textit{some} Women this is no great Discovery; had they turn'd the Tables they might have seen that \textit{some} Women are Superior to \textit{some} Men. Or had they been pleased to remember their Oaths of Allegiance and Supremacy, they might have known that \textit{One} Woman is superior to \textit{All} the Men in these Nations, or else they have sworn to very little purpose.\footnote{In 1706, England was ruled by Queen Anne.} And it must not be suppos'd, that their Reason and Religion wou'd suffer them to take Oaths, contrary to the Laws of Nature and Reason of things.\footnote{Mary Astell, \textit{Reflections upon Marriage}, 1706 Preface, iii--iv, and Mary Astell,
  \textit{Political Writings}, ed. Patricia Springborg, Cambridge University Press, 1996, 9--10.}
\end{quotation}
Wir können die verschiedenen Interpretationen von Nicholls Aussage, die Astell unterscheidet, wie folgt auflisten: Er meinte entweder, dass jeder Mann jeder Frau überlegen ist, d.h.\@
\[
\forall x(\atom{M}{x} \eif \forall y(\atom{F}{y} \eif \atom{Ü}{x,y}))
\]
oder, dass einige Männer einigen Frauen überlegen sind, d.h.\@
\[
\exists x(\atom{M}{x} \eand \exists y(\atom{F}{y} \eand \atom{Ü}{x,y})).
\]
Letzteres ist zwar wahr, aber das gleiche gilt auch für:
\[
\exists y(\atom{F}{y} \eand \exists x(\atom{M}{x} \eand \atom{Ü}{y,x})).
\]
(einige Frauen sind einigen Männer überlegen), sodass es ``no great discovery'' wäre.  In der Tat, weil die Queen all ihren Untertanen überlegen ist, ist es sogar wahr, dass
\[
\exists y(\atom{F}{y} \land \forall x(\atom{M}{x} \eif \atom{Ü}{y,x})).
\]
Aber das ist inkonsistent mit dem ``obvious meaning'' von Nicholls Aussage, der ersten Interpretation. Was Nicholls sagt, ist also Hochverrat an der Königin!

\practiceproblems
\solutions
\problempart
Unter Verwendung dieses Symbolisierungsschlüssels:
\begin{ekey}
\item[\text{Domäne}] Tiere
\item[\atom{A}{x}] \gap{x} ist ein Alligator
\item[\atom{M}{x}] \gap{x} ist eine Maus
\item[\atom{R}{x}] \gap{x} ist ein Reptil
\item[\atom{Z}{x}] \gap{x} lebt im Zoo
\item[\atom{L}{x,y}] \gap{x} liebt \gap{y}
\item[a] Amos
\item[b] Bouncer
\item[c] Cleo
\end{ekey}
symbolisieren Sie die folgenden Sätze in der LEO:
\begin{earg}
\item Wenn Cleo Bouncer liebt, dann ist Bouncer eine Maus. 
\item Wenn Bouncer und Cleo Alligatoren sind, dann liebt Amos beide.
\item Cleo liebt ein Reptil.
\item Bouncer liebt alle Mäuse, die im Zoo leben.
\item Alle Mäuse, die Amos liebt, lieben ihn gleicherma{\ss}en.
\item Jede Maus, die von Cleo geliebt wird, wird auch von Amos geliebt.
\item Es gibt eine Maus, die Bouncer liebt, aber leider erwidert Bouncer diese Liebe nicht.
\end{earg}

\problempart 
Unter Verwendung dieses Symbolisierungsschlüssels:
\begin{ekey}
\item[\text{Domäne}] Tiere
\item[\atom{T}{x}] \gap{x} ist eine Taube
\item[\atom{S}{x}] \gap{x} mag Samuraifilme
\item[\atom{G}{x,y}] \gap{x} ist grö{\ss}er als \gap{y}
\item[r] Rave
\item[h] Shane
\item[d] Daisy
\end{ekey}
symbolisieren Sie die folgenden Sätze in der LEO:
\begin{earg}
\item Rave ist eine Taube, die Samuraifilme mag.
\item Rave, Shane und Daisy sind Tauben.
\item Shane ist grö{\ss}er als Rave und Daisy ist grö{\ss}er als Shane.
\item Alle Tauben mögen Samuraifilme.
\item Nur Tauben mögen Samuraifilme.
\item Es gibt eine Taube, die grö{\ss}er als Shane ist.
\item Wenn es eine Taube gibt, die grö{\ss}er als Daisy ist, dann gibt es eine Taube, die grö{\ss}er als Shane ist.
\item Kein Tier, das Samuraifilme mag, ist grö{\ss}er als Shane.
\item Keine Taube ist grö{\ss}er als Daisy.
\item Jedes Tier, das Samuraifilme nicht mag, ist grö{\ss}er als Rave.
\item Es gibt ein Tier, das kleiner als Rave und grö{\ss}er als Shane ist.
\item Es gibt keine Taube, die kleiner als Rave und grö{\ss}er als Shane ist.
\item Keine Taube ist grö{\ss}er als sie selbst.
\item Jede Taube ist grö{\ss}er als eine Taube.
\item Es gibt ein Tier, das kleiner als jede Taube ist.
\item Wenn es ein Tier gibt, das grö{\ss}er als jede Taube ist, dann mag dieses Tier Samuraifilme nicht.
\end{earg}

\problempart
\label{pr.QLcandies}
Unter Verwendung dieses Symbolisierungsschlüssels:
\begin{ekey}
\item[\text{Domäne}] Sü{\ss}igkeiten
\item[\atom{S}{x}] \gap{x} beinhaltet Schokolade.
\item[\atom{M}{x}] \gap{x} beinhaltet Marzipan.
\item[\atom{Z}{x}] \gap{x} beinhaltet Zucker.
\item[\atom{V}{x}] Boris hat \gap{x} versucht.
\item[\atom{B}{x,y}] \gap{x} ist besser als \gap{y}.
\end{ekey}
symbolisieren Sie die folgenden Sätze in der LEO:
\begin{earg}
\item Boris hat noch nie Sü{\ss}igkeiten probiert.
\item Marzipan beinhaltet immer Zucker.
\item Einige Sü{\ss}igkeiten sind zuckerfrei.
\item Die besten Sü{\ss}igkeiten beinhalten Schokolade.
\item Keine Sü{\ss}igkeit ist besser als sie selbst.
\item Boris hat noch nie zuckerfreie Schokolade probiert.
\item Boris hat Marzipan und Schokolade probiert, aber nie beides in einer Sü{\ss}igkeit.
\item Jede Sü{\ss}igkeit mit Schokolade ist besser als alle Sü{\ss}igkeiten ohne.
\item Jede Sü{\ss}igkeit mit Schokolade und Marzipan ist besser als alle Sü{\ss}igkeiten ohne beide.
\end{earg}

\problempart
Unter Verwendung dieses Symbolisierungsschlüssels:
\begin{ekey}
\item[\text{Domäne}] Personen und Gerichte bei einem Potluck
\item[\atom{A}{x}] \gap{x} ist alle.
\item[\atom{T}{x}] \gap{x} steht am Tisch.
\item[\atom{E}{x}] \gap{x} ist Essen.
\item[\atom{P}{x}] \gap{x} ist eine Person.
\item[\atom{M}{x,y}] \gap{x} mag \gap{y}.
\item[e] Eli
\item[f] Francesca
\item[g] die Guacamole
\end{ekey}
symbolisieren Sie die folgenden Sätze in der LEO:
\begin{earg}
\item Alles Essen steht am Tisch.
\item Wenn die Guacamole nicht alle ist, dann steht sie am Tisch.
\item Alle mögen Guacamole.
\item Wenn jemand die Guacamole mag, dann ist das Eli.
\item Francesca mag nur jene Gerichte, die schon alle sind.
\item Francesca mag niemanden, und niemand mag Francesca.
\item Eli mag jede/n, der/die die Guacamole mag.
\item Eli mag alle, die die Personen mögen, die sie auch mag.
\item Wenn schon eine Person am Tisch steht, dann ist alles Essen schon alle.
\end{earg}

\chapter{Identität}
\label{sec.identity}

Betrachten wir den folgenden Satz:
\begin{earg}
\item[\ex{else1}] Pavel schuldet allen Geld.
\end{earg}
die Domäne sind Personen; damit können wir `allen' mittels eines Universalquantors allein symbolisieren. Anhand des Symbolisierungsschlüssels:
	\begin{ekey}
		\item[\atom{S}{x,y}] \gap{x} schuldet \gap{y} Geld
		\item[p] Pavel
	\end{ekey}
können wir nun Satz \ref{else1} als `$\forall x\, \atom{S}{p,x}$' symbolisieren. Aber diese Symbolisierung hat eine komische Folge. Sie bedingt, dass Pavel jedem Element der Domäne Geld schuldet (was auch immer die Domäne beinhaltet). Pavel beinhaltet die Domäne auf jeden Fall. Also besagt unsere Symbolisierung, dass Pavel sich selbst Geld schuldet. Das wollten wir wahrscheinlich nicht aussagen. Eher wollten wir so was sagen wie:
	\begin{earg}
		\item[\ex{else1b}] Pavel schuldet allen \emph{anderen} Geld.
		\item[\ex{else1c}] Pavel schuldet allen, \emph{au{\ss}er Pavel}, Geld.
	\end{earg}
Aber die kursiv geschriebenen Ausdrücke können wir noch nicht symbolisieren. Um dies zu tun, fügen wir nun ein neues Symbol zur LEO hinzu.

Das Symbol `$=$' ist ein zweistelliges Prädikat. Weil es eine besondere Bedeutung hat, werden wir es aber anders aufschreiben: wir stellen es \emph{zwischen} zwei Begriffe, und nicht vor sie. (Diese Praxis ist in der Mathematik Usus; erinnere dich an eine mathematische Gleichung wie $\frac{1}{2} = 0.5$.) Die besondere Bedeutung genie{\ss}t `$=$', da wir für dieses Symbol \emph{immer} den gleichen Symbolisierungsschlüssel nutzen: 
	\begin{ekey}
		\item[x=y] \gap{x} ist identisch mit \gap{y}
	\end{ekey}
Dass zwei Objekte identisch sind, hei{\ss}t nicht \emph{nur}, dass sie ununterscheidbar sind, oder, dass die gleichen Dinge auf sie zutreffen. Es hei{\ss}t, dass diese Objekte \emph{dasselbe} Objekt sind.

Um zu sehen, wie wir unser neues Symbol nutzen können, lasst uns den folgenden Satz symbolisieren:
\begin{earg}
\item[\ex{else2}] Pavel ist Mister Checkov.
\end{earg}
Hierzu fügen wir das hier zu unserem Symbolisierungsschlüssel hinzu:
	\begin{ekey}
		\item[c] Mister Checkov
	\end{ekey}
Nun kann Satz \ref{else2} als `$p=c$' symbolisiert werden. Das sagt uns, dass `$p$' und `$c$' auf das gleiche Objekt verweisen.

Wir können nun auch Sätze wie \ref{else1b}--\ref{else1c} symbolisieren. All diese Sätze können als `Allen, die nicht Pavel sind, schuldet Pavel Geld' umschrieben werden. Weiter paraphrasierend erhalten wir: `Für jedes $x$, wenn $x$ nicht Pavel ist, dann schuldet Pavel $x$ Geld'. Endlich können wir dann den Satz in der LEO symbolisieren, nämlich als `$\forall x (\enot x = p \eif \atom{S}{p,x})$'.

Dieser Satz beinhaltet den Ausdruck `$\enot x = p$'. Das mag zwar komisch aussehen, da das Symbol, welches der Negation `$\enot$' folgt, eine Variable und kein Prädikat ist, aber das ist kein Problem. Hier verneinen wir den ganzen Ausdruck `$x = p$'. 

% Make Ausdruck -> Formel?

Zusätzlich zu Sätzen, die Worte wie `anders', `au{\ss}er' usw.\@ verwenden, wird das Identitätssymbol uns beim Symbolisieren von Sätzen, die Worte wie `neben' oder `nur' beinhalten. Hier sind ein paar Beispiele:

\begin{earg}
\item[\ex{else3}] Niemand neben Pavel schuldet Hikaru Geld.
\item[\ex{else4}] Nur Pavel schuldet Hikaru Geld.
\end{earg}
Satz \ref{else3} kann als `Niemand, der nicht Pavel ist, schuldet Hikaru Geld'. Sagen wir, dass `$h$' auf Hikaru verweist. Dann kann der Satz als `$\enot\exists x(\enot x = p \eand \atom{S}{x,h})$' symbolisiert werden. Gleichfalls kann der Satz \ref{else3} als `für jedes $x$, wenn $x$ Hikaru Geld schuldet, dann ist $x$ Pavel'. Dann kann er als `$\forall x (\atom{S}{x,h} \eif x = p)$' symbolisiert werden.

Satz \ref{else4} kann auf ähnliche Weise behandelt werden, aber hier gilt es eine Feinheit zu beachten. Haben Satz \ref{else3} und Satz \ref{else4} zur Folge, dass Pavel Hikaru Geld schuldet?

\section{Es gibt zumindest \ldots}
Wir können das Identitätsprädikat auch nutzen, um zu sagen, wie viele Objekte einer Art es gibt. Betrachten Sie beispielsweise die folgenden Sätze:
\begin{earg}
\item[\ex{atleast1}] Es gibt zumindest einen Apfel.
\item[\ex{atleast2}] Es gibt zumindest zwei Äpfel.
\item[\ex{atleast3}] Es gibt zumindest drei Äpfel.
\end{earg}
Wir nutzen den folgenden Symbolisierungsschlüssel:
	\begin{ekey}
		\item[\atom{A}{x}] \gap{x} ist ein Apfel.
	\end{ekey}
Für Satz \ref{atleast1} brauchen wir die Identität nicht. Er kann als `$\exists x\, \atom{A}{x}$' symbolisiert werden.

Es ist verlockend, auch \ref{atleast2} ohne Identität zu symbolisieren. Aber betrachten Sie den Satz `$\exists x \exists y(\atom{A}{x} \eand \atom{A}{y})$'. Das besagt ungefähr, dass es zumindest einen Apfel $x$ in der Domäne gibt und zumindest einen Apfel $y$. Da aber nichts ausschlie{\ss}t, dass dies dieselben Äpfel sind, wäre dies wahr, selbst wenn es nur einen Apfel gäbe. Um sicherzustellen, dass wir von \emph{verschiedenen} Äpfeln sprechen, brauchen wir das Identitätsprädikat. Satz \ref{atleast2} muss aussagen, dass die zwei Äpfel, die es zumindest gibt, nicht miteinander identisch sind. Also kann der Satz als `$\exists x \exists y((\atom{A}{x} \eand \atom{A}{y}) \eand \enot x = y)$' symbolisiert werden.

Satz \ref{atleast3} bedingt, dass wir von drei verschiedenen Äpfeln sprechen. Daher brauchen wir nun drei Existenzquantoren und müssen sicherstellen, dass diese jeweils von verschiedenen Äpfeln handeln: 
\[
	\exists x \exists y\exists z[((\atom{A}{x} \eand \atom{A}{y}) \eand \atom{A}{z}) \eand ((\enot x = y \eand \enot y = z) \eand \enot x = z)].
\]
Beachten Sie, dass es \emph{nicht} ausreicht, `$\lnot x = y \land \lnot y = z$' zu nutzen, um zu symbolisieren, dass $x$, $y$ und~$z$ alle voneinander verschieden sind. Denn unser Vorschlag wäre wahr, wenn $x$ und $y$ zwar verschieden sind, aber gilt, dass $x = z$. Allgemein gesprochen: um zu sagen, dass $x_1$, \dots, $x_n$ alle voneinander verschieden sind, brauchen wir eine Konjunktion $\lnot x_i = x_j$ für jedes bis auf Reihenfolge verschiedene Paar $i$ und $j$.
%"\lnot x_1=x_2" ist äquivalent zu "\lnot x_2=x_1", nicht?

\section{Es gibt höchstens \ldots}
Wenden wir uns nun diesen Sätzen zu:
\begin{earg}
	\item[\ex{atmost1}] Es gibt höchstens einen Apfel.
	\item[\ex{atmost2}] Es gibt höchstens zwei Äpfel.
\end{earg}
Satz \ref{atmost1} kann als `Es ist nicht der Fall, dass es zumindest \emph{zwei} Äpfel gibt'. Das ist einfach die Negation von Satz \ref{atleast2}: 
$$\enot \exists x \exists y[(\atom{A}{x} \eand \atom{A}{y}) \eand \enot x = y]$$
Aber Satz \ref{atmost1} können wir auch anders symbolisieren. Der Satz bedeutet, dass Sie, wenn Sie ein Objekt auswählen, das ein Apfel ist, und dann ein weiteres Objekt auswählen, das auch ein Apfel ist, zweimal das gleiche Objekt ausgewählt haben. Wenn wir dies beachten, können wir unseren Satz auch als 
$$\forall x\forall y\bigl[(\atom{A}{x} \eand \atom{A}{y}) \eif x=y\bigr]$$
symbolisieren. Wie wir sehen werden, sind unsere zwei Symbolisierungen äquivalent.

Ähnlicherweise kann auch Satz \ref{atmost2} auf zwei Weisen behandelt werden. Er kann als `Es ist nicht der Fall, dass es zumindest \emph{drei} Äpfel gibt'. Daher können wir:
$$\enot \exists x \exists y\exists z\bigl[((\atom{A}{x} \eand \atom{A}{y}) \eand \atom{A}{z}) \eand ((\enot x = y \eand \enot x = z) \eand \enot y = z)\bigr]$$
anbieten. Andererseits können wir den Satz auch so verstehen, dass er sagt, dass Sie, wenn Sie drei Äpfel auswählen, zumindest einen Apfel mehrmals ausgewählt haben. Das ergibt dann:
$$\forall x\forall y\forall z\bigl[((\atom{A}{x} \eand \atom{A}{y}) \eand \atom{A}{z}) \eif ((x=y \eor x=z) \eor y=z)\bigr]$$


\section{Es gibt genau \ldots}\label{sec:exactlyn}
Als letztes schauen wir uns genau benannte Mengen an: 
\begin{earg}
\item[\ex{exactly1}] Es gibt genau einen Apfel.
\item[\ex{exactly2}] Es gibt genau zwei Äpfel.
\item[\ex{exactly3}] Es gibt genau drei Äpfel.
\end{earg}
Satz \ref{exactly1} kann als `Es gibt \emph{zumindest} und \emph{höchstens} einen Apfel' umschrieben werden. Das ist einfach nur die Konjunktion der Sätze \ref{atleast1} und \ref{atmost1}. Also erhalten wir:
$$\exists x \atom{A}{x} \eand \forall x\forall y\bigl[(\atom{A}{x} \eand \atom{A}{y}) \eif x=y\bigr]$$
Aber es ist vielleicht einfacher, Satz \ref{exactly1} als `Etwas, $x$, ist ein Apfel und alles, was ein Apfel ist, ist einfach nur $x$'. So umschrieben, symbolisieren wir Satz \ref{exactly1} als: 
\[
	\exists x\bigl[\atom{A}{x} \eand \forall y(\atom{A}{y} \eif x= y)\bigr]
\]

Ähnlicherweise kann Satz \ref{exactly2} als `Es gibt \emph{zumindest} und \emph{höchstens} zwei Äpfel'. Als könnten wir die folgende Symbolisierung anbieten:
\begin{multline*}
  \exists x \exists y((\atom{A}{x} \eand \atom{A}{y}) \eand \enot x = y) \eand {}\\
  \forall x\forall y\forall z\bigl[((\atom{A}{x} \eand \atom{A}{y}) \eand \atom{A}{z}) \eif ((x=y \eor x=z) \eor y=z)\bigr]
\end{multline*}
Effizienter ist es allerdings, wenn wir den Satz als `Es gibt zumindest zwei Äpfel und jeder Apfel ist einer dieser zwei Äpfel' symbolisieren. Daraus ergibt sich dann:
$$\exists x\exists y\bigl[((\atom{A}{x} \eand \atom{A}{y}) \eand \enot x = y) \eand \forall z(\atom{A}{z} \eif ( x = z \eor y = z))\bigr]$$

Zuletzt betrachten Sie zwei weitere Sätze:
\begin{earg}
\item[\ex{exactly2things}] Es gibt genau zwei Dinge.
\item[\ex{exactly2objects}] Es gibt genau zwei Objekte.
\end{earg}
Hier ist es verlockend, ein Prädikat zu unserem Symbolisierungsschlüssel hinzuzufügen, um die deutschen Prädikate `\blank\ ist ein Ding' oder `\blank\ ist ein Objekt' zu symbolisieren, aber das ist nicht notwendig. Worte wie `Ding' und `Objekt' trennen die Spreu nicht vom Weizen: sie treffen nichtssagenderweise auf alles in unserer Domäne zu. Also können wir auf zwei Weisen symbolisieren:
	\begin{align*}
		\exists x \exists y \enot x = y & \eand  \enot \exists x \exists y \exists z ((\enot x = y \eand \enot y = z) \eand \enot x = z) \\
		\exists x \exists y \bigl[\enot x = y & \eand \forall z(x = z \eor y = z)\bigr]
	\end{align*}

\practiceproblems

\problempart Erklären Sie, wieso
	\begin{ebullet}
		\item   `$\exists x \forall y(\atom{A}{y} \eiff x = y)$' eine gute Symbolisierung von `Es gibt genau einen Apfel' ist.
		\item `$\exists x \exists y \bigl[\enot x = y \eand \forall z(\atom{A}{z} \eiff (x= z \eor y = z))\bigr]$' eine gute Symbolisierung von `Es gibt genau zwei Äpfel' ist.
	\end{ebullet}		


\chapter{Sätze der LEO}\label{s:FOLSentences}
Wir wissen nun wie wir deutsche Sätze in der LEO symbolisieren können. Nun definieren wir, was den ein Satz der LEO eigentlich ist.

\section{Ausdrücke}
Es gibt sechs verschiedene Arten von Symbolen in der LEO:

\begin{description}
\item[Prädikate] $A,B,C,\ldots,Z$ oder 
mit Subskripten, falls nötig: $A_1, B_1,Z_1,A_2,A_{25},J_{375},\ldots$
\item[Namen] $a,b,c,\ldots, r$ oder
mit Subskripten, falls nötig $a_1, b_{224}, h_7, m_{32},\ldots$
\item[Variablen] $s, t, u, v, w, x, y, z$ oder
mit Subskripten, falls notwendig $x_1, y_1, z_1, x_2,\ldots$
\item[Junktoren]  $\enot,\eand,\eor,\eif,\eiff$
\item[Klammern] ( , )
\item[Quantoren]  $\forall, \exists$
\end{description}
Wir definieren einen \define{Ausdruck der LEO} als eine beliebige Reihe von Symbolen der LEO. Welche Symbole auch immer Sie aufschreiben, in welcher Anordnung auch immer, Sie kriegen einen Ausdruck der LEO heraus.

\section{Terme und Formeln}
\label{s:TermsFormulas}

In \S\ref{s:TFLSentences}, gingen wir direkt von der Auflistung des Vokabulars der WFL zur Definition eines Satzes der WFL über. In der LEO, müssen wir einen Zwischenschritt tätigen: nämlich via den Begriff einer \define{Formel}. Die Idee ist, dass eine Formel ein Satz ist oder ein Ausdruck, aus dem wir einen Satz bilden können, indem wir einen Quantor vorne anstellen. Lasst uns diese Idee etwas erläutern.

Wir beginnen, indem wir den Begriff eines Terms definieren: 
	\factoidbox{
		Ein \define{Term} ist ein beliebiger Name oder eine beliebige Variable. }
Hier sind also ein paar Terme:
	$$a, b, x, x_1 x_2, y, y_{254}, z$$
Nun definieren wir den Begriff einer atomaren Formel:
	\factoidbox{
		\begin{enumerate}
		\item Jeder Satzbuchstabe ist eine atomare Formel.
		\item Wenn $\metav{R}$ ein $n$-stelliges Prädikat ist und $\metav{t}_1, \metav{t}_2, \ldots, \metav{t}_n$ Terme sind, dann ist $\atom{\metav{R}}{\metav{t}_1, \metav{t}_2, \ldots, \metav{t}_n}$ eine atomare Formel.
		\item Wenn $\metav{t}_1$ und $\metav{t}_2$ Terme sind, dann ist $\metav{t}_1 = \metav{t}_2$ eine atomare Formel.
		\item Nichts anderes ist eine atomare Formel.
		\end{enumerate}
	}
Da wir Satzbuchstaben auch als atomare Formeln zählen, gilt jeder Satz der WFL als eine Formel der LEO (auch wenn manche natürlich keine atomaren Formeln sind).

\newglossaryentry{term}{
  name = Term,
  description = {Entweder ein \gls{Name} oder eine \gls{Variable}}
}

\newglossaryentry{formula}{
  name = Formel,
  description = {Ein Ausdruck der LEO, den induktiven Regeln in \S\ref{s:TermsFormulas} entsprechend gebildet}
}

Das Verwenden der kalligraphischen Lettern folgt den Konventionen in \S\ref{s:UseMention}. `$\metav{R}$' ist also kein Prädikat der LEO. Stattdessen ist es ein Symbol unserer Metasprache (erweitertes Deutsch), das wir verwenden, um über beliebige Prädikate der LEO zu sprechen. Ähnlicherweise ist `$\metav{t}_1$' kein Term der LEO, sondern ein Symbol der Metasprache, das wir nutzen, um über beliebige Terme der LEO zu sprechen. 

Wenn wir uns `$F$' als ein einstelliges Prädikat, `$G$' ein dreistelliges und `$S$' ein sechsstelliges Prädikat vorstellen, dann sind die folgenden Ausdrücke atomare Formeln der LEO:
	\begin{align*}
		& D &&  \atom{F}{a}\\
		& x = a && \atom{G}{x,a,y}\\
		& a = b && \atom{G}{a,a,a}\\
		& \atom{F}{x} && \atom{S}{x_1, x_2, a, b, y, x_1}\\
	\end{align*}
Nun, da wir wissen was atomare Formeln sind, können wir Klauseln anbieten, um beliebige Formeln zu definieren. Die ersten paar Bedingungen sind fast die gleichen wie die für die WFL.
	\factoidbox{
	\begin{enumerate}
		\item Jede atomare Formel ist eine Formel. 
		\item Wenn \metav{A} eine Formel ist, dann ist $\enot\metav{A}$ eine Formel.
		\item Wenn \metav{A} und \metav{B} Formeln sind, dann ist $(\metav{A}\eand\metav{B})$ eine Formel.
		\item Wenn \metav{A} und \metav{B} Formeln sind, dann ist $(\metav{A}\eor\metav{B})$ eine Formel.
		\item Wenn \metav{A} und \metav{B} Formeln sind, dann ist $(\metav{A}\eif\metav{B})$ eine Formel.
		\item Wenn \metav{A} und \metav{B} Formeln sind, dann ist $(\metav{A}\eiff\metav{B})$ eine Formel.
		\item Wenn \metav{A} eine Formel ist und \metav{x} eine Variable, dann ist $\forall\metav{x}\,\metav{A}$ eine Formel.
		\item Wenn \metav{A} eine Formel ist und \metav{x} eine Variable, dann ist $\exists\metav{x}\,\metav{A}$ eine Formel.
		\item Nichts anderes ist eine Formel.
	\end {enumerate}
	}
Wenn wir uns `$F$' wieder als einstelliges, `$G$' als dreistelliges und `$S$' als sechsstelliges Prädikat vorstellen, sind hier ein paar Formeln, die wir mit unserer Definition bauen können:
	\begin{align*}
		& \atom{F}{x}\\
		& \atom{G}{a,y,z}\\
		& \atom{S}{y,z,y,a,y,x}\\
		(\atom{G}{a,y,z} \eif {}& \atom{S}{y,z,y,a,y,x})\\
		\forall z (\atom{G}{a,y,z} \eif {}& \atom{S}{y,z,y,a,y,x})\\
		\atom{F}{x} \eand \forall z (\atom{G}{a,y,z} \eif {}& \atom{S}{y,z,y,a,y,x})\\
		\exists y (\atom{F}{x} \eand \forall z (\atom{G}{a,y,z} \eif {}& \atom{S}{y,z,y,a,y,x}))\\
		\forall x \exists y (\atom{F}{x} \eand \forall z (\atom{G}{a,y,z} \eif {}& \atom{S}{y,z,y,a,y,x}))
	\end{align*}
%Der folgende Ausdruck ist allerdings \emph{keine} Formel:
%$$\forall x \exists x\, \atom{G}{x,x,x}$$
%`$\atom{G}{x,x,x}$' ist natürlich eine Formel und `$\exists x\, \atom{G}{x,x,x}$' ebenso. Der Unterschied zwischen diesen zwei Formeln ist, dass $x$ in der ersten Formel ungebunden vorkommt--$x$ wird hier nicht von einem Quantor in Anspruch genommen; in der zweiten Formel hingegen kommt $x$ gebunden vor--$x$ wird hier vom Existenzquantor in Anspruch genommen. Da $x$ nun aber schon gebunden ist, können wir `$\forall x$' nicht noch an unsere zweite Formel andocken. Denn das verletzt Bedingung 7 unserer induktiven Definition: in `$\exists x\, \atom{G}{x,x,x}$' kommt $x$ nicht ungebunden vor. Wir definieren, was eine gebundene oder ungebundene Variable ist im nächsten Abschnitt.
%Ist das bewusst rauskommentiert? 

Nun definieren wir den Geltungsbereich eines logischen Operators (Quantors oder Junktors). Hier folgen wir der Definition der WFL:
	\factoidbox{
		Der \define{Hauptoperator} in einer Formel ist der Operator, der bei der Konstruktion dieser Formel als Letzter genutzt wurde.		
\bigskip
		Der \define{Geltungsbereich} eines Operators in einer Formel ist jene Teilformel, für den dieser Operator der Hauptoperator ist.
	}
Damit wir können wir den Geltungsbereich der Quantoren im vorhergehenden Beispiel wie folgt darstellen:
$$\overbrace{\forall x \overbrace{\exists y (\atom{F}{x} \eiff \overbrace{\forall z (\atom{G}{a,y,z} \eif \atom{S}{y,z,y,a,y,x})}^{\text{Geltungsbereich von `}\forall z\text{'}}}^{\text{Geltungsbereich von `}\exists y\text{'}})}^{\text{Geltungsbereich von `$\forall x$'}}$$

\newglossaryentry{main logical operator}{
  name = Hauptoperator,
  description = {Der Operator, der bei der Konstruktion eines \gls{Satzes der LEO} oder einer \glx{Formel}  als Letzter genutzt wurde.}
}

\newglossaryentry{scope}{
  name = Geltungsbereich,
  description = {Die Teilformel eines \gls{Satzes der LEO} oder einer \gls{Formel} der LEO, für den der relevante Operator der \gls{Hauptoperator} ist}
}

\section{Sätze und ungebundene Variablen}
In der Logik beschäftigen wir uns normalerweise mit Sätzen, die wahr oder falsch sein können. Aber viele Formeln sind keine Sätze. Betrachten wir hierzu den folgenden Symbolisierungsschlüssel:
	\begin{ekey}
		\item[\text{Domäne}] Personen
		\item[\atom{L}{x,y}] \gap{x} liebt \gap{y}
		\item[b] Boris
	\end{ekey}
Die atomare Formel `$\atom{L}{z,z}$' ist eine Formel, da alle atomaren Formeln Formeln sind. Aber kann sie wahr oder falsch sein? Sie denken sich vielleicht, dass sie wahr ist, wenn die Person, die `$z$' genannt wird, sich selbst liebt, genau so wie `$\atom{L}{b,b}$' wahr ist genau dann, wenn Boris (die Person auf die `$b$' verweist) sich selbst liebst. \emph{Allerdings ist `$z$' eine Variable und benennt kein Objekt.}

Wenn wir der Formel einen Existenzquantor voranstellen, um `$\exists z\atom{L}{z,z}$' zu erhalten, dann wäre dies natürlich wahr genau dann, wenn jemand sich selbst liebt. Gleichfalls gilt: wenn wir der Formal einen Universalquantor voranstellen, um `$\forall z \atom{L}{z,z}$' zu erhalten, dann wäre dies natürlich wahr genau dann, wenn alle sich selbst lieben. Der Punkt hier ist, dass wir eine Quantor brauchen, damit unsere Formel bestimmte Wahrheitsbedingungen hat. 

Lasst uns diese Idee etwas präziser benennen.
	\factoidbox{
		Ein Vorkommnis einer Variable \metav{x} ist \define{gebunden} genau dann, wenn sie im Geltungsbereich von $\forall\metav{x}$ oder $\exists\metav{x}$ liegt. Ein Vorkommnis einer Variable, das nicht gebunden ist, nennen wir \define{ungebunden} oder \define{frei}.}

\newglossaryentry{bound variable}{
  name = gebundene Variable,
  description = {Ein Vorkommnis einer Variable in einer \gls{Formel}, welches im Geltungsbereich eines Quantors gefolgt von der gleichen Variable liegt}
}

\newglossaryentry{free variable}{
  name = ungebundene Variable,
  description = {Ein Vorkommnis einer Variable in einer \gls{Formel}, welche keine \gls{gebundene Variable} ist}
}

        
Betrachten wir die folgende Formel als ein Beispiel:
	$$(\forall x(\atom{E}{x} \eor \atom{D}{y}) \eif \exists z(\atom{E}{x} \eif \atom{L}{z,x}))$$
Der Geltungsbereich des Universalquantors `$\forall x$' ist `$\forall x (\atom{E}{x} \eor \atom{D}{y})$'. Also ist das erste `$x$' vom Universalquantor gebunden. Die zweiten und dritten Vorkommnisse von `$x$' hingegen sind ungebunden. Gleichfalls ist auch `$y$' ungebunden. Der Geltungsbereich des Existenzquantors `$\exists z$' ist schlie{\ss}lich `$(\atom{E}{x} \eif \atom{L}{z,x})$'. Also ist `$z$' gebunden. 

Zuletzt können wir das Folgende zu Sätzen der LEO sagen:	
	\factoidbox{	
		Ein \define{Satz} der LEO ist eine Formel der LEO, welche keine ungebundenen Variablen beinhaltet.
	}

\newglossaryentry{sentence of FOL}{
	name = Satz (der LEO),
	text = Satz der LEO,
	description = {Eine \gls{Formel} der LEO, welche keine \glspl{ungebundene Variable} beinhaltet.}
}

\section{Klammerkonventionen}

Wir werden die gleichen Klammerkonventionen annehmen wie für die WFL (vgl. \S\ref{s:TFLSentences} und \S\ref{s:MoreBracketingConventions}.) Erstens können wir die äu{\ss}ersten Klammern einer Formel weglassen. Zweitens können wir eckige Klammern, `[' und `]', anstatt runder Klammern nutzen, um die Lesbarkeit von Formeln zu verbessern.

Sätze der LEO können recht umständlich werden, wie wir bereits gesehen haben. Daher führen wir auch Konventionen für Konjunktionen und Disjunktionen von mehr als zwei Sätzen ein. Wir legen fest, dass $A_1 \land A_2 \land \dots \land A_n$ und $A_1 \lor A_2 \lor \dots \lor A_n$ wie folgt zu verstehen sind:
\begin{earg}
	\item[] $(\dots(A_1 \land A_2) \land \dots \land A_n)$
	\item[] $(\dots(A_1 \lor A_2) \lor \dots \lor A_n)$
\end{earg}
In der Praxis bedeutet dies, dass Sie bei langen Konjunktionen und Disjunktionen Klammern weglassen dürfen. Aber denken Sie daran, dass Sie (sofern es sich nicht um die äu{\ss}ersten Klammern des Satzes handelt) immer noch die gesamte Konjunktion oder Disjunktion in Klammern einschlie{\ss}en müssen. Au{\ss}erdem dürfen Sie Konjunktionen und Disjunktionen nicht mit von sich selbst verschiedenen Junktoren vermischen. Daher sind die folgenden Ausdrücke nach wie vor nicht erlaubt und wären in diesem Fall mehrdeutig:
\begin{earg}
	\item[] $A \lor B \land C \land D$
	\item[] $B \lor C \eif D$
\end{earg}

\section{Superskript bei Prädikaten}
Oben haben wir gesagt, dass ein $n$-stelliges Prädikat, gefolgt von $n$ Termen, eine atomare Formel ist. Diese Definition hat jedoch ein kleines Problem: Die Symbole, die wir für Prädikate verwenden, geben nicht an, wie viele Stellen ein Prädikat hat. An einigen Stellen in diesem Buch haben wir den Buchstaben `$G$' als einstelliges Prädikat verwendet; an anderen Stellen aber als dreistelliges. Wenn wir also nicht explizit angeben, ob wir `$G$' als einstelliges oder dreistelliges Prädikat verwenden, ist es \emph{unbestimmt}, ob `$\atom{G}{a}$' eine atomare Formel ist.

Es gibt einen einfachen Weg, dies zu vermeiden, den viele Lehrbücher gehen. Anstatt zu sagen, dass unsere Prädikate nur Gro{\ss}buchstaben sind (gegebenenfalls mit numerischen Subskripten), könnten wir sagen, dass sie Gro{\ss}buchstaben \emph{mit numerischen Superskripten} sind (gegebenenfalls mit numerischen Subskripten). Der Zweck der Superskripte wäre, explizit anzuzeigen, wie viele Stellen ein Prädikat hat. Bei diesem Ansatz wäre `$G^1$' ein einstelliges Prädikat und `$G^3$' ein (völlig anderes) dreistelliges Prädikat. Sie müssten in jedem Symbolisierungsschlüssel unterschiedliche Einträge haben. `$\atom{G^1}{a}$' wäre eine atomare Formel, während `$\atom{G^3}{a}$' keine atomare Formel wäre. Ebenso wäre `$\atom{G^3}{a,b,c}$' eine atomare Formel, im Gegensatz zu `$\atom{G^1}{a,b,c}$'. 

Wir \emph{könnten} also all unseren Prädikaten mit Superskripten ausstatten. Dies hätte den Vorteil, dass bestimmte Dinge explizit angezeigt würden. Es hätte jedoch auch den Nachteil, dass unsere Formeln viel schwieriger zu lesen wären; die Superskripte würden uns ablenken. Wir werden uns also nicht die Mühe machen, diese Änderung vorzunehmen. Unsere Prädikate schreiben wir \emph{ohne} Superskripte auf.

Diese Konvention lässt manchmal Mehrdeutigkeit zu. Wenn eine solche Mehrdeutigkeit auftaucht--in der Praxis sehr selten--sollten Sie daher explizit sagen, wie viele Stellen Ihr(e) Prädikat(e) haben. 

\practiceproblems
\problempart
\label{pr.freeFOL}
Bestimmen Sie, welche Variablen gebunden und welche ungebunden sind.
\begin{earg}
\item $\exists x\, \atom{L}{x,y} \eand \forall y\, \atom{L}{y,x}$
\item $\forall x\, \atom{A}{x} \eand \atom{B}{x}$
\item $\forall x (\atom{A}{x} \eand \atom{B}{x}) \eand \forall y(\atom{C}{x} \eand \atom{D}{y})$
\item $\forall x\exists y[\atom{R}{x,y} \eif (\atom{J}{z} \eand \atom{K}{x})] \eor \atom{R}{y,x}$
\item $\forall x_1(\atom{M}{x_2} \eiff \atom{L}{x_2,x_1}) \eand \exists x_2\,\atom{L}{x_3,x_2}$
\end{earg}


\chapter{Bestimmte Beschreibungen}\label{subsec.defdesc}
Betrachten Sie die folgenden Sätze:
	\begin{earg}
		\item[\ex{traitor1}] Nick ist der Verräter.
		\item[\ex{traitor2}] Der Verräter ging nach Herde.
		\item[\ex{traitor3}] Der Verräter ist der Stellvertreter. 
	\end{earg}
Hier haben wir einige \emph{bestimmte Beschreibungen}: sie sollen ein \emph{eindeutiges} Objekt benennen. Sie sind von \emph{unbestimmten} Beschreibungen, z.B.\@ `Nick  ist \emph{ein} Verräter', zu unterscheiden. Ebenfalls sind sie von \emph{generischen Aussagen}, beispielsweise `\emph{Der} Wal ist ein Säugetier', zu unterscheiden (hier ist es unangebracht zu fragen, \emph{welcher} Wal gemeint ist). Die Frage die sich uns stellt ist: wie sollen wir bestimmte Beschreibungen in der LEO behandeln?


\section{Bestimmte Beschreibungen als Terme}
Eine Möglichkeit wäre, neue Namen einzuführen, sobald wir auf bestimmte Beschreibungen sto{\ss}en. Aber das ist wahrscheinlich keine gute Idee. Wir wissen, dass der Verräter--wer auch immer es ist--in der Tat ein Verräter ist. Wir wollen diese Information in unserer Symbolisierung bewahren. Ein neuer Name tut das aber nicht.

Eine zweite Möglichkeit wäre, einen Operator für bestimmte Beschreibungen einzuführen: `$\maththe$'. Die Idee wäre nun, `der/die/das $F$' als `$\maththe x\,\atom{F}{x}$' (gelesen als: `das $x$ für das gilt: $\atom{F}{x}$'). Ausdrücke der Form $\maththe \metav{x}\, \atom{\metav{A}}{\metav{x}}$ würden sich dann so verhalten wie Namen. Für die obigen Sätze könnten wir nun den folgenden Symbolisierungsschlüssel verwenden:
	\begin{ekey}
		\item[\text{Domäne}] Personen
		\item[\atom{V}{x}] \gap{x} ist ein Verräter
		\item[\atom{S}{x}] \gap{x} ist ein Stellvertreter
		\item[\atom{H}{x}] \gap{x} ging nach Herde
		\item[n] Nick
	\end{ekey}
Nun könnten wir \ref{traitor1} als `$n = \maththe x\, \atom{V}{x}$', \ref{traitor2} als `$\atom{H}{\maththe x\,\atom{V}{x}}$' und \ref{traitor3} als `$\maththe x\, \atom{V}{x} = \maththe x\, \atom{S}{x}$' symbolisieren. 

Aber es wäre schön, wenn wir kein weiteres Symbol zur LEO hinzufügen müssten. Lasst uns ausprobieren, ob wir mit den schon vorhandenen Ressourcen auskommen.

\section{Russells Analyse}
Bertrand Russell entwickelte eine Analyse der bestimmten Beschreibungen. Kurz gesagt bemerkte er, dass, wenn wir `der/die/das $F$' in einer bestimmten Beschreibung verwenden, darauf abzielen auf das (in einem bestimmten Kontext) \emph{einzige} Objekt, das $F$ ist, zu verweisen. Russel analysiert bestimmte Beschreibungen also wie folgt:\footnote{Bertrand Russell, `On Denoting', 1905, \emph{Mind 14}, pp.\ 479--93; also Russell, \emph{Introduction to Mathematical Philosophy}, 1919, London: Allen and Unwin, ch.\ 16.}
	\begin{align*}
		\text{der/die/das $F$ ist $G$ \textbf{genau dann, wenn} }&\text{es gibt zumindest ein $F$ \emph{und}}\\
	&\text{es gibt höchstens ein $F$ \emph{und}}\\	
	&\text{jedes $F$ ist $G$}
\end{align*}
Eine wichtige Eigenschaft dieser Analsye ist, das der bestimmte Artikel nicht auf der rechten Seite des Bikonditionals vorkommt. Russell versucht uns ein Verständnis für bestimmte Beschreibungen zur Hand zu reichen, welches dieses nicht schon voraussetzt. 

Nun könnte man befürchten, dass wir sagen können, `der Tisch ist braun', ohne anzudeuten, dass es im Universum nur einen einzigen Tisch gibt. Aber dies ist (noch) kein Gegenbeispiel zu Russells Analyse. Die Domäne wird wahrscheinlich durch den Kontext beschränkt sein (z.B.\@ auf relevante Objekte in meiner Umgebung).

Wenn wir Russells Analyse der bestimmten Beschreibungen akzeptieren, dann können wir Sätze der Form `der/die/das $F$ ist $G$' mittels unserer LEO-Werkzeuge zur Beschreibung von Mengen symbolisieren. Die drei Konjunkte auf der rechten Seite von Russells Analyse können wir wie folgt symbolisieren:
	$$\exists x \atom{F}{x} \eand \forall x \forall y ((\atom{F}{x} \eand \atom{F}{y}) \eif x = y) \eand \forall x (\atom{F}{x} \eif \atom{G}{x})$$
Tatsächlich könnten wir denselben Punkt sogar knackiger ausdrücken, indem wir erkennen, dass die ersten beiden Konjunktionen gerade auf die Behauptung hinauslaufen, dass es \emph{genau} ein $F$ gibt, und erkennen, dass die letzte Konjunktion uns sagt, dass genau dieses Objekt $G$ ist. Äquivalent könnten wir also anbieten:
	$$\exists x \bigl[(\atom{F}{x} \eand \forall y (\atom{F}{y} \eif x = y)) \eand \atom{G}{x}\bigr]$$
Mithilfe dieser Werkzeuge können wir nun Sätze \ref{traitor1}--\ref{traitor3} ohne einen neuen Operator, wie `$\maththe$', symbolisieren. 

Satz \ref{traitor1} würden wir so symbolisieren:
\begin{align*}
\exists x \bigl[\atom{V}{x} \eand \forall y(\atom{V}{y} \eif x = y) & \eand x = n\bigr].
\intertext{Satz \ref{traitor2} ist ebenfalls unproblematisch:} 
\exists x \bigl[\atom{T}{x} \eand \forall y(\atom{T}{y} \eif x = y) & \eand \atom{C}{x}\bigr].
\end{align*}

Satz \ref{traitor3} ist etwas komplizierter, weil er zwei bestimmte Beschreibungen verkoppelt. Aber, wenn wir Russells Analyse nutzen, können wir ihn als `Es gibt genau einen Verräter, $x$, und genau einen Stellvertreter, $y$, und $x = y$' umschreiben. Und dann können wir den Satz wie folgt symbolisieren: 
\begin{multline*}
	\exists x \exists y \bigl(\bigl[\atom{T}{x} \eand 
	\forall z(\atom{T}{z} \eif x = z)\bigr] \eand {}\\
	\bigl[\atom{D}{y} \eand 
	\forall z(\atom{D}{z} \eif y = z)\bigr] \eand x = y\bigr)
\end{multline*}
Beachte hier, dass die Formel `$x = y$' im Geltungsbereich beider Quantoren liegen muss!

\section{Leere bestimmte Beschreibungen}
Eines der guten Merkmale von Russells Analyse ist, dass sie uns erlaubt, mit \emph{leeren} bestimmten Beschreibungen umzugehen.

Frankreich hat zur Zeit keinen König. Wenn wir nun einen Namen, `$k$', einführen würden, um den gegenwärtigen König von Frankreich zu benennen, dann würde alles schief gehen. Denken Sie an \S\ref{s:FOLBuildingBlocks}: Ein Name muss immer irgendein Objekt in der Domäne herausgreifen und was immer wir als unsere Domäne wählen, es wird kein Objekt enthalten, das derzeit König Frankreichs ist. 

Russells Analyse vermeidet dieses Problem. Russell sagt uns, dass wir bestimmte Beschreibungen mit Prädikaten und Quantifikatoren anstelle von Namen behandeln sollen. Da Prädikate leer sein können (siehe \S\ref{s:MoreMonadic}), bedeutet dies, dass jetzt keine Schwierigkeiten auftreten, wenn die bestimmte Beschreibung leer ist--d.h.\@ wenn die Beschreibung auf kein Objekt in der Domäne zutrifft. 

Russells Analyse zeigt in der Tat zwei Möglichkeiten auf, wie eine Verwendung einer bestimmten Beschreibung schief gehen kann. Hier ist ein Beispiel inspiriert von Stephen Neale (1990),\footnote{Neale, \emph{Descriptions}, 1990, Cambridge: MIT Press.} Nehmen Sie an, dass Alex sagt, dass
	\begin{earg}
		\item[\ex{kingdate}] Ich date den derzeitigen König Frankreichs.
	\end{earg}
Wir nutzen den folgenden Symbolisierungsschlüssel:
	\begin{ekey}
		\item[a] Alex
		\item[\atom{K}{x}] \gap{x} ist ein derzeitiger König Frankreichs
		\item[\atom{D}{x,y}] \gap{x} datet \gap{y}
	\end{ekey}
(Der Symbolisierungsschlüssel spricht von \emph{einem} derzeitigen König Frankreichs, nicht \emph{dem} derzeitigen König Frankreichs; d.h.\@ er nutzt eine unbestimmte Beschreibung.) Satz \ref{kingdate} symbolisieren wir nun als `$\exists x \bigl[(\atom{K}{x} \eand \forall y(\atom{K}{y} \eif  x = y)) \eand \atom{D}{a,x}\bigr]$'. Dieser Satz kann auf mindestens zwei Arten falsch sein:
	\begin{earg}
		\item[\ex{outernegation}] Es gibt niemanden, der sowohl derzeitiger König Frankreichs ist und von dem gilt, das Alex ihn datet.
		\item[\ex{innernegation}] Es gibt genau einen derzeitigen König Frankreichs, aber Alex datet ihn nicht
	\end{earg}
Satz \ref{outernegation} können wir als `Es ist nicht der Fall, dass: der derzeitige König Frankreichs und Alex daten'. Er wird dann als `$\enot \exists x\bigl[(\atom{K}{x} \eand \forall y(\atom{K}{y} \eif  x = y)) \eand \atom{D}{a,x} \bigr]$' symbolisiert. Wir nennen dies die \emph{äu{\ss}ere} Negation, da der Geltungsbereich der Negation, der ganze Satz ist. Dieser Satz ist wahr, wenn es keinen derzeitigen König Frankreichs gibt.

Satz \ref{innernegation} können wir als `$\exists x \bigl[(\atom{K}{x} \eand \forall y(\atom{K}{y} \eif x = y)) \eand \enot \atom{D}{a,x}\bigr]$' symbolisieren. Wir nennen dies die \emph{innere} Negation, weil die Negation im Geltungsbereich der bestimmten Beschreibung liegt. Dieser Satz ist wahr genau dann, wenn es einen derzeitigen König Frankreichs gibt, der aber nicht mit Alex datet.

\section{Possessive, `beide', `keines von beiden'}

Wir können Russells Analyse bestimmter Beschreibungen auch verwenden, um singuläre Possessive im Deutschen zu symbolisieren.  Zum Beispiel bedeutet `Schmidts Mörder' so etwas wie `die Person, die Schmidt ermordet hat', d.h.\@ es handelt sich um eine ``getarnte'' bestimmte Beschreibung. In Russells Analyse kann der Satz 
\begin{earg}
	\item[\ex{smithsmurder}] Schmidts Mörder ist verrückt.
\end{earg}
auf drei Weisen falsch sein. Er kann falsch sein, weil die einzige Person, die Schmidt ermordet hat, gar nicht verrückt ist. Er kann aber auch falsch sein, wenn die bestimmte Beschreibung leer ist, nämlich wenn entweder niemand Schmidt ermordet hat (z.B.\@ wenn Schmitt einem Autounfall erlag) oder wenn mehr als eine Person Schmidt ermordet hat.

Um Sätze zu symbolisieren, die singuläre Possessive enthalten, wie z.B.\@ `Schmidts Mörder', sollten Sie diese Sätze zunächst mit einer expliziten, bestimmten Beschreibung paraphrasieren, z.B.\@ `Die Person, die Schmidt ermordet hat, ist verrückt', und sie dann laut Russells Analyse symbolisieren. In unserem Fall würden wir den folgenden Symbolisierungsschlüssel verwenden:
\begin{ekey}
	\item[Domäne] Personen
	\item[\atom{V}{x}] \gap{x} ist verrückt % I -> V
	\item[\atom{M}{x,y}] \gap{x} hat \gap{y} ermordet
	\item[s] Schmidt
\end{ekey}
Nun erhalten wir die folgende Symbolisierung: `$\exists x\bigl[\atom{M}{x,s} \eand \forall y(\atom{M}{y,s} \eif x=y) \eand \atom{V}{x}\bigr]$'.

Russells Analyse können wir auch auf `beide' und `keine/r von beiden' anwenden.  `Beide $F$s sind $G$' besagt, dass es genau zwei $F$s gibt und jedes dieser Objekte ist $G$. `Keines von beiden $F$s ist $G$' besagt, dass es genau zwei $F$s und keines dieser Objekte ist $G$. In der LEO schauen die Symbolisierungen dann so aus:
\begin{earg}
	\item[] $\exists x\exists y\bigl[\atom{F}{x} \eand \atom{F}{y} \eand \enot x=y \eand {}$
	\item[]\qquad$\forall z(\atom{F}{z} \eif (x = z \lor y = z)) \eand \atom{G}{x} \eand \atom{G}{y}\bigr]$
	\item[] $\exists x\exists y\bigl[\atom{F}{x} \eand \atom{F}{y} \eand \enot x=y \eand  {}$
	\item[]\qquad$\forall z(\atom{F}{z} \eif (x = z \lor y = z)) \eand \enot \atom{G}{x} \eand \enot \atom{G}{y}\bigr]$
\end{earg}
Vergleichen Sie diese Symbolisierungen mit unseren Symbolisierungen für `Genau zwei $F$s sind $G$s' aus Abschnitt \ref{sec:exactlyn}, d.h.\@ unsere Symbolisierungen von `Es gibt genau zwei Dinge, die sowohl $F$ und $G$ sind':
\begin{earg}
	\item[] $\exists x\exists y\bigl[(\atom{F}{x} \eand \atom{G}{x}) \eand (\atom{F}{y} \eand \atom{G}{y}) \eand \enot x=y \eand {}$
	\item[]\qquad$\forall z((\atom{F}{z} \eand \atom{G}{z}) \eif (x = z \lor y = z))\bigr]$
\end{earg}
Der Unterschied zwischen diesen Symbolisierungen und der von `beide $F$s sind~$G$s' liegt im Antezedens des Konditionals. Für `genau zwei $F$ sind $G$' verlangen wir nur, dass es keine $F$s gibt, \emph{die auch} $G$s sind, au{\ss}er $x$ und $y$. Im Gegensatz dazu bedingt `beide $F$s sind $G$', dass es, neben $x$ und $y$, keine anderen $F$s gibt, ob sie $G$s sind oder nicht. Mit anderen Worten, `beide $F$s sind $G$' hat zur Folge, dass genau zwei $F$s $G$s sind. Allerdings hat `genau zwei $F$s sind $G$' nicht zur Folge, dass beide $F$s $G$ sind (es könnte ein drittes $F$ geben, das nicht $G$ ist).

\section{Ist Russells Analyse adäquat?}
Wie gut ist Russells Analyse bestimmter Beschreibungen? Diese Frage wird in einer umfangreichen philosophische Literatur behandelt; hier werden wir uns auf zwei Beobachtungen beschränken.

Die eine Sorge konzentriert sich auf Russells Analyse leerer bestimmter Beschreibungen. Wenn es kein $F$ gibt, dann sind nach Russells Analyse sowohl `das $F$ ist $G$' als auch `das $F$ ist kein $G$' falsch. P.F.\@ Strawson hingegen schlug vor, dass solche Sätze nicht als falsch angesehen werden sollten, sondern dass eine ihrer \emph{Voraussetzungen} fehlschlägt und sie daher als \emph{weder} wahr \emph{noch} falsch behandelt werden sollten.\footnote{P.F.\ Strawson, `On Referring', 1950, \emph{Mind 59}, pp.\ 320--34.} 

Wenn wir hier mit Strawson übereinstimmen, müssen wir unsere Logik abändern. Denn in unserer Logik gibt es nur zwei Wahrheitswerte ("wahr" und "falsch") und jedem Satz ist genau einer dieser Wahrheitswerte zugeordnet. 

Aber es gibt Gründe, Strawsons Sorge abzulehnen. Strawson appelliert an unsere sprachlichen Intuitionen, aber es ist nicht klar, ob sie sehr robust sind. Zum Beispiel: Ist es nicht einfach nur \emph{falsch}, dass Alex den gegenwärtigen König von Frankreich datet, wenn es gegenwärtig keinen König von Frankreich gibt?

Keith Donnellan äu{\ss}erte eine zweite Sorge, die wir (sehr grob) nachvollziehen können, wenn wir einen Fall einer Identitätsverwechslung betrachten.\footnote{Keith Donnellan, `Reference and Definite Descriptions', 1966, \emph{Philosophical Review 77}, pp.\ 281--304.} Zwei Männer stehen in der Ecke: ein sehr gro{\ss}er Mann, der etwas trinkt, das wie ein Martini aussieht, und ein sehr kleiner Mann, der etwas trinkt, das wie ein Glas Wasser aussieht. Als er sie sieht, sagt Malika:
	\begin{earg}
		\item[\ex{gindrinker}] Der Martinitrinker ist sehr gro{\ss}!
	\end{earg}
Laut Russells Analyse symbolisieren wir diesen Satz wie folgt:
	\begin{earg}
		\item[\ref{gindrinker}$'$.] Es gibt genau einen Martinitrinker [in der Ecke] und alle Martinitrinker [in der Ecke] sind sehr gro{\ss}.
	\end{earg}
Nehmen wir nun an, dass der sehr gro{\ss}e Mann tatsächlich \emph{Wasser} aus seinem Martini-Glas trinkt, während der sehr kleine Mann Martini aus seinem Wasserglas trinkt. Nach Russells Analyse hat Malika etwas Falsches gesagt, aber wollen wir nicht sagen, dass Malika etwas \emph{Wahres} gesagt hat? 

Wieder könnte man sich fragen, wie klar unsere Intuitionen zu diesem Fall sind. Wir sind uns alle einig, dass Malika die Absicht hatte, einen bestimmten Mann zu benennen und etwas Wahres über ihn zu sagen (dass er sehr gro{\ss} ist). Nach Russells Analyse benannte sie tatsächlich einen anderen Mann (den kleinen) und sagte folglich etwas Falsches über ihn aus. Aber vielleicht brauchen die Befürworter von Russells Analyse nur zu erklären, warum Malika ihre Absicht nicht in die Tat umsetzen konnte, weshalb sie also etwas Falsches gesagt hat. Diese Aufgabe ist aber einfach erledigt: Malika sagte etwas Falsches aus, weil sie falsche Überzeugungen über die Getränke der Männer hatte; wenn diese Überzeugungen wahr gewesen wären, dann hätte sie auch etwas Wahres ausgesagt.\footnote{Bei weitergehendem Interesse dazu empfehle ich Saul Kripke, `Speaker Reference and Semantic Reference', 1977, in French et al (eds.), \emph{Contemporary Perspectives in the Philosophy of Language}, Minneapolis: University of Minnesota Press, pp.\ 6-27.}

Hier noch viel mehr zu sagen würde uns in tiefe philosophische Gewässer führen. Das wäre nicht schlecht, aber es würde uns von unserem unmittelbaren Ziel ablenken, die formale Logik zu erlernen. Deshalb beharren wir erst einmal auf Russells Analyse bestimmter Beschreibungen, wenn es darum geht, Sätze in der LEO zu symbolisieren. Sie ist das Beste, was wir anbieten können, ohne unsere Logik wesentlich abzuändern, und ist als Analyse durchaus vertretbar. 

\practiceproblems

\problempart
Anhand des folgenden Symbolisierungsschlüssels:
\begin{ekey}
\item[\text{Domäne}] Personen
\item[\atom{K}{x}] \gap{x} kennt den Code.
\item[\atom{S}{x}] \gap{x} ist ein Spion.
\item[\atom{V}{x}] \gap{x} ist Vegetarier.
\item[\atom{T}{x,y}] \gap{x} traut \gap{y}.
\item[h] Hofthor
\item[i] Ingmar
\end{ekey}
symbolisieren Sie die folgenden Sätze in der LEO:
\begin{earg}
\item Hofthor traut einem Vegetarier.
\item Alle, die Ingmar trauen, trauen einem Vegetarier.
\item Alle, die Ingmar trauen, trauen jemandem, der einem Vegetarier traut.
\item Nur Ingmar kennt den Code.
\item Ingmar traut Hofthor, aber niemandem sonst.
\item Die Person, die den Code kennt, ist Vegetarierin.
\item Die Person, die den Code kennt, ist keine Spionin.
\end{earg}

\problempart 
Anhand des folgenden Symbolisierungsschlüssels:
\begin{ekey}
\item[\text{Domäne}] Tiere auf der Welt
\item[\atom{B}{x}] \gap{x} steht in Bauer Brauns Feld.
\item[\atom{P_1}{x}] \gap{x} ist ein Pferd.
\item[\atom{P_2}{x}] \gap{x} ist ein Pegasus.
\item[\atom{F}{x}] \gap{x} hat Flügel.
\end{ekey}
symbolisieren Sie die folgenden Sätze in der LEO:
\begin{earg}
\item Es gibt zumindest drei Pferde auf der Welt.
\item Es gibt zumindest drei Tiere auf der Welt.
\item Es steht mehr als ein Pferd in Bauer Brauns Feld.
\item Es stehen drei Pferde in Bauer Brauns Feld.
\item Es gibt genau ein Tier mit Flügeln in Bauer Brauns Feld; alle anderen Tiere sind flügellos.
\item Der Pegasus ist ein geflügeltes Pferd.
\item Das Tier, das in Bauer Brauns Feld steht, ist kein Pferd.
\item Das Pferd, das in Bauer Brauns Feld steht, hat keine Flügel.
\end{earg}

\problempart
In diesem Kapitel haben wir `Nick ist der Verräter' als `$\exists x (\atom{V}{x} \eand \forall y(\atom{V}{y} \eif x = y) \eand x = n)$' symbolisiert. Erkläre wieso die folgenden Alternativen ebenso gute Symbolisierungen sind:
	\begin{ebullet}
		\item $\atom{V}{n} \eand \forall y(\atom{V}{y} \eif n = y)$
		\item $\forall y(\atom{V}{y} \eiff y = n)$
	\end{ebullet}


\chapter{Mehrdeutigkeit}

In Kapitel \ref{s:AbmbiguityTFL} betonten wir, dass deutsche Sätze mehrdeutig sein können und, dass Sätze der WFL \emph{nicht} mehrdeutig sein können. 

Eine wichtige Anwendung dieses Unterschieds besteht darin, dass die strukturelle Mehrdeutigkeit deutscher Sätze oft durch verschiedene Symbolisierungen geradegerückt werden kann. Eine häufige Quelle der Mehrdeutigkeit ist die Mehrdeutigkeit der Geltungsbereiche von Operatoren: hier stellt der deutsche Satz nicht klar, was im Geltungsbereich seiner Operatoren liegt. Mehrere Interpretationen sind daher möglich. In der LEO hingegen hat jeder Junktor und Quantor einen genau festgelegten Geltungsbereich, sodass immer feststeht, ob einer von ihnen im Geltungsbereich eines anderen vorkommt oder nicht.

Betrachten Sie als Beispiel den folgenden Satz:
\begin{earg}
	\item[\ex{onlyamb}] Nur junge Katzen sind verspielt.
\end{earg} 
Unserem bestehenden Schema nach, symbolisieren wir diesen Satz wie folgt:
\begin{earg}
	\item[] $\forall x(\atom{V}{x} \eif (\atom{J}{x} \eand \atom{K}{x}))$ % P -> V , Y -> J , C -> K
\end{earg}
Die Bedeutung dieses LEO Satzes ist ungefähr `Wenn ein Tier verspielt ist, dann ist es eine junge Katze'. (Wir nehmen hier an, dass die Domäne Tiere umfasst.) Aber das entspricht wahrscheinlich nicht der Bedeutung, die wir mit \ref{onlyamb} ausdrücken wollen. Wahrscheinlich wollen wir nur sagen, dass \emph{alte} Katzen nicht verspielt sind. Anders ausgedrückt: wir wollen sagen, dass, wenn etwas eine Katze ist und verspielt ist, dann muss es jung sein. Das würden wir so symbolisieren:
\begin{earg}
	\item[] $\forall x((\atom{K}{x} \eand \atom{V}{x}) \eif \atom{J}{x})$ 
\end{earg}
Unser Satz erlaubt auch noch eine dritte Lesart. Nehmen wir an, wir unterhalten uns über junge Tiere und ihre Eigenschaften. Und nehmen wir an, dass wir sagen wollen, dass unter all den jungen Tieren nur die Katzen verspielt sind. Diese Lesart könnten wir wie folgt symbolisieren:
\begin{earg}
	\item[] $\forall x((\atom{J}{x} \eand \atom{V}{x}) \eif \atom{K}{x})$ 
\end{earg}
Die letzten beiden Lesarten können im Deutschen hervorgehoben werden, indem die Betonung entsprechend gesetzt wird. Um z.B.\@ die letzte Interpretation nahezulegen, würden Sie sagen: `Nur junge \emph{Katzen} sind verspielt', und um die zweite Lesart zu erhalten, würden Sie sagen: `Nur \emph{junge} Katzen sind verspielt'.  Die allererste Lesart kann hingegen durch das Betonen von sowohl `jungen' als auch `Katzen' nahegelegt werden: `Nur \emph{junge Katzen} sind verspielt' (aber nicht alte Katzen oder Hunde jeden Alters).

In Kapitel \ref{ss:OrderQuant} und \ref{ss:SuppQuant} betonten wir, dass die Reihenfolge der Quantoren einen Unterschied macht. Dieser Punkt ist hier wieder relevant, weil im Deutschen die Reihenfolge der Quantoren nicht immer klar ist. Wenn wir sowohl Universal- als auch Existenzquantor in einem Satz verwenden, dann kann es sein, dass die Geltungsbereiche der Quantoren mehrdeutig sind. Betrachten Sie:
\begin{earg}
	\item[\ex{everya}] Alle sahen einen Film.
\end{earg}
Dieser Satz ist mehrdeutig. Einer Lesart nach, bedeutet er, dass es einen Film gab, den alle sahen. Der zweiten Lesart nach, bedeutet der Satz, dass alle einen Film sahen, aber nicht unbedingt den gleichen. Die zwei Lesarten können wir wie folgt symbolisieren: 
\begin{earg}
	\item[] $\exists x(\atom{F}{x} \eand \forall y(\atom{P}{y} \eif \atom{S}{y,x}))$ % M -> F
	\item[] $\forall y(\atom{P}{y} \eif \exists x(\atom{F}{x} \eand \atom{S}{y,x}))$
\end{earg}
Wir nehmen hier an, dass die Domäne (zumindest) Personen und Filme beinhaltet, und, dass der Symbolisierungsschlüssel so aussieht:
\begin{ekey}
	\item[\atom{P}{y}] \gap{y}~ist eine Person, 
	\item[\atom{F}{x}] \gap{x}~ist ein Film
	\item[\atom{S}{y,x}] \gap{y}~sah~\gap{x}. 
\end{ekey}
In der ersten Symbolisierung hat der Existenzquantor einen \emph{weiten Geltungsbereich}, der Universalquantor dagegen hat einen \emph{engen} Geltungsbereich und liegt im Geltungsbereich des Existenzquantors. In der zweiten Symbolisierung läuft es umgekehrt: der Existenzquantor hat hier einen engen und der Universalquantor einen weiten Geltungsbereich.

In Kapitel \ref{subsec.defdesc} trafen wir eine weitere Mehrdeutigkeit an, die auftritt, wenn eine bestimmte Beschreibung mit einer Negation interagiert.  Hier ist Russells eigenes Beispiel:
\begin{earg}
	\item[\ex{thenot}] Der König Frankreichs ist nicht glatzig.
\end{earg}
Wenn die bestimmte Beschreibung weiten Geltungsbereich hat und wir `nicht' als eine `innere' Negation interpretieren, dann interpretieren wir Satz \ref{thenot} so, dass er die Existenz eines einzelnen König Frankreichs bedingt, welcher nicht glatzig ist. Diese Lesart symbolisieren wir als `$\exists x\bigl[\atom{K}{x} \eand \forall y(\atom{K}{y} \eif x=y)) \eand \enot \atom{G}{x}\bigr]$'. Der anderen Lesart nach, negiert `nicht' den ganzen Satz `Der König Frankreichs ist glatzig'. Diese Lesart symbolisieren wir als `$\enot\exists x\bigl[\atom{K}{x} \eand \forall y(\atom{K}{y} \eif x=y)) \eand \atom{G}{x}\bigr]$'. Im ersten Fall hat die bestimmte Beschreibung einen weiten Geltungsbereich, im zweiten einen engen Geltungsbereich.

\practiceproblems
\problempart
Jeder der folgenden Sätze ist mehrdeutig. Stellen Sie einen Symbolisierungsschlüssel zur Verfügung und symbolisieren Sie alle Lesarten.
\begin{earg}
	\item Niemand mag einen Kater.
	\item Holmes fand nur rotes Haar am Tatort.
	\item Schmidts Mörder wurde nicht verhaftet.
\end{earg}

\problempart
Russel gab in seinem Artikel `On Denoting', das folgende Beispiel:
\begin{quote}
	I have heard of a touchy owner of a yacht to whom a guest, on first seeing it, remarked, `I thought your yacht was larger than it is'; and the owner replied, `No, my yacht is not larger than it is'. 
\end{quote}
Erklären Sie, was hier vonstattengeht.
