%!TEX root = forallxdo.tex

\pagestyle{empty}

\vspace*{80pt}

\begin{raggedleft}
\fontsize{30pt}{24pt}\sffamily
\selectfont
  \textbf{forall 
  {\fontsize{37pt}{24pt}\selectfont\rmfamily\textit{x}}: 
  Dortmund}

\medskip\fontsize{18pt}{20pt}\selectfont

\textbf{Eine Einführung in\\ die formale Logik}

\vfill
\fontsize{12pt}{16pt}\selectfont \textit{Von } \textbf{P.~D. Magnus}\\
\textbf{Tim Button}\\
\textbf{Aaron Thomas-Bolduc}\\ 
\textbf{Richard Zach}\\
\textit{mit Ergänzungen von}\\
\textbf{J.~Robert Loftis}\\
\textbf{Robert Trueman}\\
\textit{überarbeitet und übersetzt von}\\
\textbf{Simon Wimmer}


\vfill
\textbf{\forallxversion}\par
\end{raggedleft}


\newpage


\noindent\small%
Dieses Buch basiert auf \href{http://forallx.openlogicproject.org/}{\forallx: \emph{Calgary}}, von Aaron Thomas-Bolduc \& \href{https://richardzach.org/}{Richard Zach} (University of Calgary), verwendet unter einer \href{https://creativecommons.org/licenses/by/4.0/}{CC BY 4.0} Lizenz, welches auf \href{http://www.homepages.ucl.ac.uk/~uctytbu/OERs.html}{\forallx: \emph{Cambridge}} basiert, von \href{http://nottub.com/}{Tim Button} (University College London), verwendet unter einer \href{https://creativecommons.org/licenses/by/4.0/}{CC BY 4.0} Lizenz, welches wiederum auf \href{https://www.fecundity.com/logic/}{\forallx} basiert, von \href{https://www.fecundity.com/job/}{P.D.\ Magnus} (University at Albany, State University of New York), verwendet unter einer \href{https://creativecommons.org/licenses/by/4.0/}{CC BY 4.0} Lizenz.
\forallx: \emph{Calgary} beinhaltet zusätzliches Material aus \forallx{} von P.D. Magnus und \href{http://people.ds.cam.ac.uk/tecb2/metatheory.shtml}{\emph{Metatheory}} von Tim Button, 
verwendet unter einer \href{https://creativecommons.org/licenses/by/4.0/}{CC BY 4.0} Lizenz, 
aus \href{https://github.com/rob-helpy-chalk/openintroduction}{\forallx: \emph{Lorain
County Remix}}, von \href{https://sites.google.com/site/cathalwoods/}{Cathal Woods} und J. Robert Loftis, und aus \href{http://www.rtrueman.com/uploads/7/0/3/2/70324387/modal_logic_primer.pdf}{\emph{A Modal Logic Primer}} von \href{http://www.rtrueman.com/}{Robert Trueman}, verwendet mit Einwilligung. \href{https://github.com/sbwimmer/forallx-do}{\forallx: \emph{Dortmund}} lässt den Teil zur Metatheorie weg.

\bigskip

\noindent \footnotesize Diese Publikation ist unter einer \href{https://creativecommons.org/licenses/by/4.0/}{Creative Commons Namensnennung 4.0} Lizenz lizensiert. 
Sie dürfen das Material in jedwedem Format oder Medium vervielfältigen und weiterverbreiten, das Material remixen, verändern und darauf aufbauen und zwar für beliebige Zwecke, sogar kommerziell, unter den folgenden Bedingungen:
\begin{itemize}
\item Sie müssen angemessene Urheber- und Rechteangaben machen, einen Link zur Lizenz beifügen und angeben, ob Änderungen vorgenommen wurden. Diese Angaben dürfen in jeder angemessenen Art und Weise gemacht werden, allerdings nicht so, dass der Eindruck entsteht, der Lizenzgeber unterstütze gerade Sie oder Ihre Nutzung besonders. 
\item Sie dürfen keine zusätzlichen Klauseln oder technische Verfahren einsetzen, die anderen rechtlich irgendetwas untersagen, was die Lizenz erlaubt. 
\end{itemize}

\vfil\normalsize\noindent
Der \LaTeX{} Quellcode für dieses Buch ist auf \href{https://github.com/sbwimmer/forallx-do}{GitHub} verfügbar. Ein vollständiges PDF wird nach Abschluss der Übersetzung/Überarbeitung auf \href{https://simonwimmer.weebly.com}{simonwimmer.weebly.com} verfügbar gemacht.

\bigskip
