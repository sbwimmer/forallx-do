\chapter{Vorwort}

Wie Sie dem Titel entnehmen können, handelt es sich hier um eine Einführung in die formale Logik. Die formale Logik befasst sich mit einer bestimmten Art von Sprache. Sie ist eine formale Sprache, d.h.\@ ihre Ausdrücke (z.B.\@ Sätze) sind formal definiert. Das macht sie zu einer sehr nützlichen Sprache, um die Sachverhalte, von denen ihre Sätze handeln, sehr genau zu beschreiben. In der formalen Logik ist es unmöglich, mehrdeutig zu sein. Im Zentrum der Arbeit zu formalen Sprachen steht die Folgebeziehung zwischen Sätzen, d.h.\@ welche Sätze aus welchen anderen Sätzen folgen. Die Folgebeziehung ist von zentraler Bedeutung, weil wir durch ein besseres Verständnis dieser Beziehung erkennen können, wann bestimmte Dinge der Fall sein müssen, wenn andere Dinge der Fall sind. Aber die Folgebeziehung ist nicht der einzige wichtige Begriff der formalen Logik. Wir werden noch einige weitere kennen lernen.

Die formale Logik ist ein zentraler Teil der Philosophie, wo die logische Beziehung von Prämissen zu den aus ihnen gewonnenen Schlussfolgerungen wichtig ist. Philosoph*innen untersuchen die Konsequenzen von Definitionen und Annahmen und bewerten sie auf der Grundlage ihrer Konsequenzen. Die formale Logik ist auch in der Mathematik und Informatik von Bedeutung. In der Mathematik werden formale Sprachen verwendet, um nicht ``alltägliche'', sondern mathematische Sachverhalte zu beschreiben. Auch Mathematiker*innen interessieren sich für die Folgen von Definitionen und Annahmen und für sie ist es ebenso wichtig, diese Folgen (die sie `Theoreme' nennen) mit präzisen Methoden zu bestimmen. Die formale Logik stellt solche Methoden zur Verfügung. In der Informatik wird die formale Logik angewandt, um den Zustand und das Verhalten von Rechensystemen, z.B.\@ Schaltkreisen, Programmen, Datenbanken usw.\@, zu beschreiben. Methoden der formalen Logik werden auch verwendet, um die Konsequenzen solcher Beschreibungen festzustellen, z.B.\@ ob eine Schaltung fehlerfrei ist, ob ein Programm das tut, was es tun soll, oder ob eine Datenbank konsistent ist.

Das Lehrbuch ist in acht Teile gegliedert. Teil~\ref{ch.intro} führt auf informelle Weise in das Thema und die Begriffe der Logik ein, noch ohne Bezug auf eine formale Sprache. Die Teile \ref{ch.TFL}--\ref{ch.NDTFL} behandeln wahrheitsfunktionale Sprachen. Darin werden komplexe Sätze aus einfachen Sätzen mittels Junktoren (`oder', `und', `nicht', `wenn\dots, dann\dots') gebildet, die einfache Sätze zu komplizierteren Sätzen zusammenfügen. Wir diskutieren logische Begriffe, wie z.B.\@ die Folgebeziehung auf zwei Arten: semantisch mit der Methode der Wahrheitstabellen (in Teil~\ref{ch.TruthTables}) und beweistheoretisch mit einem System formaler Herleitungen (in Teil~\ref{ch.NDTFL}). Die Teile \ref{ch.FOL}--\ref{ch.NDFOL} führen eine kompliziertere Sprache ein: die Logik erster Ordnung. Sie umfasst neben den Junktoren der wahrheitsfunktionalen Logik auch Namen, Prädikate, Identität und die sogenannten Quantoren. Diese zusätzlichen Elemente erlauben uns, mehr als mittels der wahrheitsfunktionalen Sprache auszusagen. Wir werden untersuchen, wie viel genau man mittels der Logik der ersten Ordnung ausdrücken kann. Auch hier werden wir logische Begriffe semantisch, mit Hilfe von Interpretationen, und beweistheoretisch, mit Hilfe einer komplexeren Version des in Teil~\ref{ch.NDTFL} eingeführten formalen Herleitungssystems definieren. In Teil~\ref{ch.ML} diskutieren wir schlie{\ss}lich die modale Logik, welche die wahrheitsfunktionale Logik durch nicht-wahrheitsfunktionale Operatoren für Möglichkeit und Notwendigkeit ergänzt. 

Im Anhang finden Sie eine Liste, in der wichtige Regeln und Definitionen aufgeführt sind. Zentrale Begriffe sind in einem Glossar am Ende aufgelistet.

Dieses Buch basiert auf einem englisch-sprachigem Text, der ursprünglich von P.D.\@ Magnus geschrieben wurde, von Tim Button überarbeitet und erweitert wurde, und zuletzt auch von Aaron Thomas-Bolduc und Richard Zach in Teilen umgeschrieben und um verschiedene Materialien ergänzt wurde. Spezifisch integrierten Aaron Thomas-Bolduc und Richard Zach Material (hauptsächlich Übungen) von J.\@ Robert Loftis und ergänzten Teil~\ref{ch.ML}, der auf Notizen von Robert Trueman basiert. (Ihr Text enthält auch Material, welches auf zwei Kapiteln aus Tim Buttons offenem Text \emph{Metatheorie} basiert, welches aber nicht in der hier präsentierten Fassung des Buchs vorkommt.) Der resultierende Text steht unter einer Creative Commons Namensnennung 4.0-Lizenz. Ich bedanke mich bei Daniel Foelsch für seine Hilfe beim Korrekturlesen.
