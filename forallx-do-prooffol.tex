%!TEX root = forallxdo.tex
\part{Natürliche Herleitung für die LEO}
\label{ch.NDFOL}
\addtocontents{toc}{\protect\mbox{}\protect\hrulefill\par}

\chapter{Grundregeln für die LEO}\label{s:BasicFOL}

Die LEO nutzt alle Junktoren der WFL. Daher werden Beweise in der LEO alle Grund- und abgeleiteten Regeln der WFL nutzen (siehe Teil \ref{ch.NDTFL}). Ebenso werden wir die beweistheoretischen Begriffe, die wir dort eingeführt hatten auch weiterhin nutzen (insbesondere `$\proves$'). Allerdings brauchen wir auch neue Grundregeln für die Quantoren und das Identitätssymbol.

\section{Universaleliminierung}

Aus der Behauptung, dass alles $F$ ist, können Sie herleiten, dass ein bestimmtes Objekt~$F$ ist. Was auch immer sonst es ist, es ist $F$. Das Folgende ist also in Ordnung:
\begin{fitchproof}
	\hypo{a}{\forall x\,\atom{R}{x,x,d}}
	\have{c}{\atom{R}{a,a,d}} \Ae{a}
\end{fitchproof}
Wir erhielten hier Zeile 2, indem wir den Universalquantor weglie{\ss}en und jede Vorkommnis von `$x$' durch `$a$' ersetzten. Gleicherma{\ss}en ist folgendes erlaubt:
\begin{fitchproof}
	\hypo{a}{\forall x\,\atom{R}{x,x,d}}
	\have{c}{\atom{R}{d,d,d}} \Ae{a}
\end{fitchproof}
Hier erhielten wir Zeile 2, indem wir den Universalquantor weglie{\ss}en und jede Vorkommnis von `$x$' mit `$d$' ersetzten. Wir hätten das gleiche auch mit jedem anderen Namen tun können. 

Diesen Herleitungen entspricht die Universaleliminationsregel ($\forall$E):
\factoidbox{
\begin{fitchproof}
	\have[m]{a}{\forall \metav{x}\,\metav{A}(\ldots \metav{x} \ldots \metav{x}\ldots)}
	\have[\ ]{c}{\metav{A}(\ldots \metav{c} \ldots \metav{c}\ldots)} \Ae{a}
\end{fitchproof}}
Die Notation hier haben wir schon in \S\ref{s:TruthFOL} eingeführt. Der Punkt ist, dass Sie jede \emph{Substitutionsinstanz} einer universalquantifizierten Formel erhalten können: ersetzen Sie einfach jede Vorkommnis der gebundenen Variable mit einem Namen. 

Zu betonen ist, dass Sie (wie bei jeder Eliminationsregel) die $\forall$E-Regel nur dann anwenden können, wenn der Universalquantor der Hauptoperator eines Satzes ist. Daher ist das Folgende \emph{verboten}:
\begin{fitchproof}
	\hypo{a}{\forall x\,\atom{B}{x} \eif \atom{B}{k}}
	\have{c}{\atom{B}{b} \eif \atom{B}{k}}\by{inkorrekte Anwendung von $\forall$E}{a}
\end{fitchproof}
Dies ist unzulässig, da `$\forall x$' nicht der Hauptoperator in Zeile~1 ist (wenn Sie nachvollziehen wollen, warum diese Art von Schlussfolgerung verboten ist, blättern Sie zurück zu \S\ref{s:MoreMonadic}).

\section{Existenzeinführung}
Aus der Behauptung, dass ein bestimmtes Objekt~$F$ ist, können Sie herleiten, dass etwas~$F$ ist. Also ist das folgende zulässig:
\begin{fitchproof}
	\hypo{a}{\atom{R}{a,a,d}}
	\have{b}{\exists x\, \atom{R}{a,a,x}} \Ei{a}
\end{fitchproof}
Hier haben wir den Namen `$d$' durch eine Variable `$x$' ersetzt und diese dann mit einem Existenzquantor gebunden. Gleicherma{\ss}en ist erlaubt:
\begin{fitchproof}
	\hypo{a}{\atom{R}{a,a,d}}
	\have{c}{\exists x\, \atom{R}{x,x,d}} \Ei{a}
\end{fitchproof}
Hier haben wir beide Vorkommnisse des Namens `$a$' durch eine Variable ersetzt und diese Variable dann mit einem Existenzquantor gebunden. Aber wir brauchen nicht beide Vorkommnisse eines Namens durch eine Variable zu ersetzen: Wenn Narziss sich selbst liebt, dann gibt es jemanden, der Narziss liebt. Also lassen wir auch das Folgende zu:
\begin{fitchproof}
	\hypo{a}{\atom{R}{a,a,d}}
	\have{d}{\exists x\, \atom{R}{x,a,d}} \Ei{a}
\end{fitchproof}
Hier haben wir \emph{ein und nur ein} Vorkommnis des Namens `$a$' durch eine Variable ersetzt und diese Variable dann mit einem Existenzquantor gebunden. 

Diese Beobachtungen motivieren unsere Einführungsregel, obwohl wir zu ihrer Erläuterung eine neue Notation einführen müssen. Wo $\metav{A}$ ein Satz ist, der einen Namen $\metav{c}$ beinhaltet, können wir dies betonen, indem wir `$\metav{A}(\ldots \metav{c} \ldots \metav{c}\ldots)$' schreiben. Wir schreiben dann `$\metav{A}(\ldots \metav{x} \ldots \metav{c}\ldots)$', um eine beliebige Formel anzuzeigen, die wir erhalten haben, indem wir \emph{ein oder mehr} Vorkommnisse des Namens \metav{c} mit der Variable \metav{x} ersetzt haben. Diese Notation erlaubt uns, die Existenzeinführungsregel zu formulieren:
\factoidbox{
\begin{fitchproof}
	\have[m]{a}{\metav{A}(\ldots \metav{c} \ldots \metav{c}\ldots)}
	\have[\ ]{c}{\exists \metav{x}\,\metav{A}(\ldots \metav{x} \ldots \metav{c}\ldots)} \Ei{a}
\end{fitchproof}
\metav{x} darf nicht in $\metav{A}(\ldots \metav{c} \ldots \metav{c}\ldots)$ vorkommen.}

Die angeführte Einschränkung garantiert, dass jede Anwendung der Regel einen Satz der LEO ausspuckt. Daher ist das Folgende erlaubt:
\begin{fitchproof}
	\hypo{a}{\atom{R}{a,a,d}}
	\have{d}{\exists x\, \atom{R}{x,a,d}} \Ei{a}
	\have{e}{\exists y \exists x\, \atom{R}{x,y,d}} \Ei{d}
\end{fitchproof}
Aber das hier verboten:
\begin{fitchproof}
	\hypo{a}{\atom{R}{a,a,d}}
	\have{d}{\exists x\, \atom{R}{x,a,d}} \Ei{a}
	\have{e}{\exists x\, \exists x\, \atom{R}{x,x,d}}\by{inkorrekte Anwendung von $\exists$I}{d}
\end{fitchproof}
da der Ausdruck in Zeile~3 widerstreitende Variablen enthält (welcher Existenzquantor bindet welches Vorkommnis von `$x$'?) und somit kein Satz der LEO ist.

\section{Leere Domänen}
Der folgende Beweis kombiniert unsere zwei neuen Regeln:
	\begin{fitchproof}
		\hypo{a}{\forall x\, \atom{F}{x}}
		\have{in}{\atom{F}{a}}\Ae{a}
		\have{e}{\exists x\, \atom{F}{x}}\Ei{in}
	\end{fitchproof}
Könnte dies ein schlechter Beweis sein? Wenn überhaupt etwas existiert, dann können wir sicherlich aus der Tatsache, dass alles~$F$ ist, herleiten, dass etwas~$F$ ist. Aber was ist, wenn überhaupt \emph{nichts} existiert? Dann ist wahr, dass alles~$F$ ist; daraus folgt jedoch nicht, dass etwas~$F$ ist, denn es gibt nichts das $F$ sein könnte. Wenn wir also behaupten, dass `$\exists x\, \atom{F}{x}$' aus `$\forall x\, \atom{F}{x}$' folgt, dann behaupten wir, dass die Logik garantiert, dass es etwas gibt. Ist das akzeptabel?

Tatsächlich haben wir uns bereits darauf festgelegt. In \S\ref{s:FOLBuildingBlocks} haben wir gesagt, dass Domänen der LEO mindestens ein Element haben müssen. Dann definierten wir eine Tautologie als einen Satz, der in jeder Interpretation wahr ist. Da nun `$\exists x\, x=x$' in jeder Interpretation wahr ist, hat unsere Bestimmung zur Folge, dass die Logik garantiert, dass es etwas gibt.

Da es nicht ohne weiteres klar ist, dass die Logik dies garantieren \emph{soll}, könnten wir versuchen, unsere Festlegung zu umgehen. Dieses Manöver ist allerdings kostspielig. Hier sind drei Dinge, die wir sagen wollen:
	\begin{ebullet}
		\item $\forall x\,\atom{F}{x} \proves \atom{F}{a}$: $\forall$E.
		\item $\atom{F}{a} \proves \exists x\,\atom{F}{x}$: $\exists$I.
		\item die Fähigkeit, Beweise durch Copy-and-Paste zu erhalten: Schlie{\ss}lich argumentieren wir, indem wir viele kleine Schritte zu gro{\ss}en Beweisen zusammenfügen.
	\end{ebullet}
Um all diese Dinge zu sagen, müssen wir auch $\forall x\,\atom{F}{x} \proves \exists x\,\atom{F}{x}$ sagen. Wenn die Logik nicht die Existenz einiger Dinge garantieren soll, dann müssen wir also eine der drei obigen Aussagen aufgeben.

Bevor wir anfangen, darüber nachzudenken, welche wir aufgeben wollen, sollten wir uns fragen, ob wir nicht doch sagen sollen, dass die Logik garantiert, dass etwas existiert. Zugegeben, ontologische Debatten darüber, warum es etwas statt nichts gibt, werden dadurch nicht gerade interessanter. Aber normalerweise hat unsere Annahme keine negativen Konsequenzen. Vielleicht sollten wir also einfach unser Herleitungssystem (und die LEO, allgemeiner gesagt) als ein System betrachten, dessen Anwendungsbereich etwas eingeschränkt ist. Wenn wir jemals die Möglichkeit des Nichtsseins zulassen wollen, dann brauchen wir ein komplizierteres Herleitungssystem. Aber solange wir uns damit zufrieden geben, diese Möglichkeit zu ignorieren, ist unser Herleitungssystem in Ordnung. (Genauso wie die Vorschrift, dass jede Domäne mindestens ein Objekt enthalten muss).

\section{Universaleinführung}
Nehmen Sie an, Sie haben zu jedem einzelnen Objekt gezeigt, dass es $F$ ist (und dass es keine anderen Dinge zu berücksichtigen gibt). Dann könnten Sie zu Recht behaupten, dass alles $F$ ist. Das motiviert die folgende Beweisregel. Wenn Sie jede einzelne Substitutionsinstanz von `$\forall x\, \atom{F}{x}$' nachgewiesen haben, dann können Sie `$\forall x\, \atom{F}{x}$' herleiten. 

Leider wäre diese Regel völlig unbrauchbar. Um jede einzelne Substitutionsinstanz zu beweisen, müssten Sie `$\atom{F}{a}$', `$\atom{F}{b}$', \dots, `$\atom{F}{j_2}$', \dots, `$\atom{F}{r_{79002}}$', \dots herleiten. Da es in der LEO prinzipiell unendlich viele Namen gibt, würde dieser Prozess kein Ende finden. Wir können die eben motivierte Regel also niemals anwenden. Wir müssen bei der Aufstellung unserer Regel zur Einführung eines Universalquantors etwas geschickter vorgehen. 

Eine Lösung wird durch Fälle wie:
$$\forall x\,\atom{F}{x} \therefore \forall y\,\atom{F}{y}$$
inspiriert. Dieses Argument ist \emph{klarerweise} gültig. Schlie{\ss}lich ist die alphabetische Variation eine Frage des Geschmacks und hat keine logischen Konsequenzen. Aber wie könnte unser Beweissystem dies widerspiegeln? Nehmen wir an, wir beginnen einen Beweis so:
\begin{fitchproof}
	\hypo{x}{\forall x\, \atom{F}{x}} 
	\have{a}{\atom{F}{a}} \Ae{x}
\end{fitchproof}
Wir haben `$\atom{F}{a}$' bewiesen. Und nichts hindert uns daran, dieselbe Rechtfertigung auch für `$\atom{F}{b}$', `$\atom{F}{c}$', \ldots, `$\atom{F}{j_2}$', \ldots, `$\atom{F}{r_{79002}}$, \dots zu geben (bis unser Leben endet). Und so sehen wir ein, dass wir $F\metav{c}$, für jeden Namen \metav{c} beweisen können. Aber wenn wir das für \emph{jedes} Ding tun können, können wir doch sicherlich sagen, dass `$F$' auf \emph{alles} zutrifft. Dies rechtfertigt den Schluss auf `$\forall y\,\atom{F}{y}$' wie folgt:
\begin{fitchproof}
	\hypo{x}{\forall x\, \atom{F}{x}}
	\have{a}{\atom{F}{a}} \Ae{x}
	\have{y}{\forall y\, \atom{F}{y}} \Ai{a}
\end{fitchproof}
Der wichtige Punkt hier ist, dass `$a$' nur ein \emph{beliebiger} Name war. Es war nichts Besonderes an diesem Namen--wir hätten jeden anderen Namen wählen können--und trotzdem wäre der Beweis korrekt gewesen. Dieser Punkt motiviert die Universaleinführungsregel ($\forall$I):
\factoidbox{
\begin{fitchproof}
	\have[m]{a}{\metav{A}(\ldots \metav{c} \ldots \metav{c}\ldots)}
	\have[\ ]{c}{\forall \metav{x}\,\metav{A}(\ldots \metav{x} \ldots \metav{x}\ldots)} \Ai{a}
\end{fitchproof}
	\metav{c} darf in keiner ungetilgten Annahme vorkommen\\ 
	\metav{x} darf nicht in $\metav{A}(\ldots \metav{c} \ldots \metav{c}\ldots)$ vorkommen}
Ein entscheidender Aspekt dieser Regel ist in der ersten Einschränkung niedergeschrieben. Diese Einschränkung stellt sicher, dass wir immer auf einer ausreichend allgemeinen Ebene argumentieren.

Um zu sehen, wie diese Einschränkung dies garantiert, betrachten Sie:
	\begin{quote}
		Jeder liebt Kylie Minogue; deshalb liebt jeder sich selbst.
	\end{quote}
Wir könnten dieses klar ungültige Argument so symbolisieren:
$$\forall x\,\atom{L}{x,k} \therefore \forall x\,\atom{L}{x,x}$$
Versuchen wir nun, einen Beweis für dieses Argument zu geben:
\begin{fitchproof}
	\hypo{x}{\forall x\, \atom{L}{x,k}}
	\have{a}{\atom{L}{k,k}} \Ae{x}
	\have{y}{\forall x\, \atom{L}{x,x}} \by{inkorrekte Anwendung von $\forall$I}{a}
\end{fitchproof}\noindent
Dies ist nicht erlaubt, weil `$k$' bereits in einer ungetilgten Annahme, nämlich in Zeile 1, vorkommt. Der entscheidende Punkt ist, dass wir, wenn wir irgendwelche Annahmen über das Objekt, mit dem wir arbeiten, gemacht haben, nicht allgemein genug argumentieren, um $\forall$I zu verwenden.

Obwohl der Name in keiner \emph{ungetilgten} Annahme vorkommen darf, darf er in einer \emph{getilgten} Annahme vorkommen. D.h.\@ er darf in einem bereits geschlossenen Unterbeweis vorkommen. Das folgende Beispiel ist zulässig:
\begin{fitchproof}
	\open
		\hypo{f1}{\atom{G}{d}}
		\have{f2}{\atom{G}{d}}\by{R}{f1}
	\close
	\have{ff}{\atom{G}{d} \eif \atom{G}{d}}\ci{f1-f2}
	\have{zz}{\forall z(\atom{G}{z} \eif \atom{G}{z})}\Ai{ff}
\end{fitchproof}
Dieser Beweis zeigt, dass `$\forall z (\atom{G}{z} \eif \atom{G}{z})$' ein \emph{Theorem} ist.

Lasst uns den letzten Punkt noch weiter betonen. Den Konventionen \S\ref{s:MainLogicalOperatorQuantifier}s nach bedingt die Verwendung von $\forall$I, dass wir \emph{jede} Vorkommnis des Namens \metav{c} in $\metav{A}(\ldots \metav{x}\ldots\metav{x}\ldots)$ mit der Variable \metav{x} ersetzen. Wenn wir nur \emph{einige} Namen ersetzen und andere nicht, dann ``beweisen'' wir einige blöde Aussagen. Zum Beispiel:
	\begin{quote}
		Jeder ist so alt wie er selbst; also ist jeder so alt wie Judi Dench.
	\end{quote}
Wir können dieses Argument so symbolisieren:
$$\forall x\,\atom{A}{x,x} \therefore \forall x\,\atom{A}{x,d}$$
Versuchen wir nun einen Beweis zu konstruieren, der diesem blöden Argument entspricht:
\begin{fitchproof}
	\hypo{x}{\forall x\, \atom{A}{x,x}}
	\have{a}{\atom{A}{d,d}}\Ae{x}
	\have{y}{\forall x\, \atom{A}{x,d}}\by{inkorrekte Anwendung von $\forall$I}{a}	
\end{fitchproof}
Glücklicherweise lassen unsere Regeln solche Beweise nicht zu: der versuchte Beweis ist unzulässig, da er nicht \emph{jede} Vorkommnis von `$d$' in Zeile $2$ mit `$x$' ersetzt.

\section{Existenzeliminierung}
Nehmen Sie an, Sie wissen, dass \emph{etwas} $F$ ist. Nur so viel zu wissen, erlaubt uns nicht zu wissen, welches Ding $F$ ist. Daher können wir von `$\exists x\,\atom{F}{x}$' nicht direkt auf `$\atom{F}{a}$', `$\atom{F}{e_{23}}$' oder irgendeine andere Substitutionsinstanz schlie{\ss}en. Können wir sonst was machen?

Nehmen Sie an, Sie wissen, dass etwas $F$ ist und, dass alle $F$s $G$ sind. Nun könnten wir wie folgt argumentieren:
	\begin{quote}
		Da etwas $F$ ist, gibt es ein bestimmtes Ding, das ein $F$ ist. Wir wissen nichts über dieses Ding, au{\ss}er dass es ein $F$ ist, aber der Einfachheit halber nennen wir es `Becky'. Also: Becky ist~$F$. Da alles, was $F$ ist,~$G$ ist, folgt daraus, dass Becky~$G$ ist. Da aber Becky~$G$ ist, folgt daraus, dass etwas~$G$ ist. Und nichts hing davon ab, welches Objekt genau Becky war. Etwas ist also~$G$.
	\end{quote}
Diese Argumentation können wir so in unserem Herleitungssystem einfangen:
\begin{fitchproof}
	\hypo{es}{\exists x\, \atom{F}{x}}
	\hypo{ast}{\forall x(\atom{F}{x} \eif \atom{G}{x})}
	\open
		\hypo{s}{\atom{F}{b}}
		\have{st}{\atom{F}{b} \eif \atom{G}{b}}\Ae{ast}
		\have{t}{\atom{G}{b}} \ce{st, s}
		\have{et1}{\exists x\, \atom{G}{x}}\Ei{t}
	\close
	\have{et2}{\exists x\, \atom{G}{x}}\Ee{es,s-et1}
\end{fitchproof}\noindent
Brechen wir dies herunter: Wir begannen mit unseren Annahmen. In Zeile $3$ führten wir dann eine weitere Annahme ein: `$\atom{F}{b}$'. Das war einfach eine Substitutionsinstanz von `$\exists x\,\atom{F}{x}$'. Mittels dieser Annahme zeigten wir, dass `$\exists x\,\atom{G}{x}$'. Zu beachten ist, dass wir keine  \emph{besonderen} Annahmen über das Objekt, das wir `$b$' nennen, gemacht haben. Wir nahmen \emph{nur} an, dass es `$\atom{F}{x}$' erfüllt. Daher hängt nichts davon ab, welches Objekt es genau ist. Und Zeile $1$ sagt uns, dass \emph{etwas} `$\atom{F}{x}$' erfüllt, also war unsere Argumentation verallgemeinerbar. Wir können die spezifische Annahme `$\atom{F}{b}$' tilgen und auf `$\exists x\,\atom{G}{x}$' schlie{\ss}en.

Hier ist nun die Existenzeliminierungsregel ($\exists$E):
\factoidbox{
\begin{fitchproof}
	\have[m]{a}{\exists \metav{x}\,\metav{A}(\ldots \metav{x} \ldots \metav{x}\ldots)}
	\open	
		\hypo[i]{b}{\metav{A}(\ldots \metav{c} \ldots \metav{c}\ldots)}
		\have[j]{c}{\metav{B}}
	\close
	\have[\ ]{d}{\metav{B}} \Ee{a,b-c}
\end{fitchproof}
\metav{c} darf in keiner Annahme vorkommen, die vor Zeile $i$ ungetilgt ist\\
\metav{c} darf nicht in $\exists \metav{x}\,\metav{A}(\ldots \metav{x} \ldots \metav{x}\ldots)$ vorkommen\\
\metav{c} darf nicht in \metav{B} vorkommen}
Wie bei der Universaleinführung sind die Einschränkungen äu{\ss}erst wichtig. Um zu sehen, warum, betrachten Sie das folgende schreckliche Argument:
	\begin{quote}
		Tim Button ist ein Dozent. Jemand ist kein Dozent. Also ist Tim Button sowohl Dozent als auch nicht.
	\end{quote}
Dieses ungültige Argument können wir wie folgt symbolisieren:
$$\atom{D}{b}, \exists x\, \enot\atom{D}{x} \therefore \atom{D}{b} \eand \enot \atom{D}{b}$$
Versuchen wir nun einen Beweis für dieses Argument zu konstruieren:
\begin{fitchproof}
	\hypo{f}{\atom{D}{b}}
	\hypo{nf}{\exists x\, \enot \atom{D}{x}}	
	\open	
		\hypo{na}{\enot \atom{D}{b}}
		\have{con}{\atom{D}{b} \eand \enot \atom{D}{b}}\ai{f, na}
	\close
	\have{econ1}{\atom{D}{b} \eand \enot \atom{D}{b}}\by{inkorrekte}{}
	\have[\ ]{x}{}\by{Anwendung von $\exists$E }{nf, na-con}
\end{fitchproof}
Die letzte Zeile dieses Beweises ist unzulässig. Der Name, den wir in unserer Substitutionsinstanz für `$\exists x\, \enot \atom{D}{x}$' in Zeile~$3$ nutzten, `$b$', kommt in Zeile $4$ vor. Auch das hier ist schlecht:
\begin{fitchproof}
	\hypo{f}{\atom{D}{b}}
	\hypo{nf}{\exists x\, \enot \atom{D}{x}}	
	\open	
		\hypo{na}{\enot \atom{D}{b}}
		\have{con}{\atom{D}{b} \eand \enot \atom{D}{b}}\ai{f, na}
		\have{con1}{\exists x (\atom{D}{x} \eand \enot \atom{D}{x})}\Ei{con}		
	\close
	\have{econ1}{\exists x (\atom{D}{x} \eand \enot \atom{D}{x})}\by{inkorrekte}{}
	\have[\ ]{x}{}\by{Anwendung von $\exists$E }{nf, na-con1}
\end{fitchproof}
Die letzte Zeile ist immer noch unzulässig. Denn der Name, den wir in unserer Substitutionsinstanz für `$\exists x\, \enot \atom{L}{x}$' nutzten, `$b$', kommt in einer ungetilgten Annahme vor: Zeile~$1$. 

Die Moral ist: \emph{Wenn Sie Informationen aus einem Existenzquantor herausquetschen wollen, wählen Sie einen neuen Namen für Ihre Substitutionsinstanz.} Auf diese Weise garantieren Sie, dass Sie alle Einschränkungen der Regel $\exists$E erfüllen.

\practiceproblems
\problempart
Erklären Sie, wieso die folgenden Beweise \emph{inkorrekt} sind. Geben Sie auch Interpretationen an, die zeigen, dass die den Beweisen zugrundeliegenden Argumente ungültig sind:
\begin{multicols}{2}
	\begin{fitchproof}
		\hypo{Rxx}{\forall x\, \atom{R}{x,x}}
		\have{Raa}{\atom{R}{a,a}}\Ae{Rxx}
		\have{Ray}{\forall y\, \atom{R}{a,y}}\Ai{Raa}
		\have{Rxy}{\forall x\, \forall y\, \atom{R}{x,y}}\Ai{Ray}
	\end{fitchproof}
	\begin{fitchproof}
		\hypo{AE}{\forall x\, \exists y\, \atom{R}{x,y}}
		\have{E}{\exists y\, \atom{R}{a,y}}\Ae{AE}
		\open
			\hypo{ass}{\atom{R}{a,a}}
			\have{Ex}{\exists x\, \atom{R}{x,x}}\Ei{ass}
		\close
		\have{con}{\exists x\, \atom{R}{x,x}}\Ee{E, ass-Ex}
	\end{fitchproof}
\end{multicols}

\problempart 
\label{pr.justifyFOLproof}
Den folgenden drei Beweisen fehlen Zitationen (Regeln und Zeilennummern). Fügen Sie diese hinzu.
\begin{earg}
\item \begin{fitchproof}
\hypo{p1}{\forall x\exists y(\atom{R}{x,y} \eor \atom{R}{y,x})}
\hypo{p2}{\forall x\,\enot \atom{R}{m,x}}
\have{3}{\exists y(\atom{R}{m,y} \eor \atom{R}{y,m})}{}
	\open
		\hypo{a1}{\atom{R}{m,a} \eor \atom{R}{a,m}}
		\have{a2}{\enot \atom{R}{m,a}}{}
		\have{a3}{\atom{R}{a,m}}{}
		\have{a4}{\exists x\, \atom{R}{x,m}}{}
	\close
\have{n}{\exists x\, \atom{R}{x,m}} {}
\end{fitchproof}

\item \begin{fitchproof}
\hypo{1}{\forall x(\exists y\,\atom{L}{x,y} \eif \forall z\,\atom{L}{z,x})}
\hypo{2}{\atom{L}{a,b}}
\have{3}{\exists y\,\atom{L}{a,y} \eif \forall z\atom{L}{z,a}}{}
\have{4}{\exists y\, \atom{L}{a,y}} {}
\have{5}{\forall z\, \atom{L}{z,a}} {}
\have{6}{\atom{L}{c,a}}{}
\have{7}{\exists y\,\atom{L}{c,y} \eif \forall z\,\atom{L}{z,c}}{}
\have{8}{\exists y\, \atom{L}{c,y}}{}
\have{9}{\forall z\, \atom{L}{z,c}}{}
\have{10}{\atom{L}{c,c}}{}
\have{11}{\forall x\, \atom{L}{x,x}}{}
\end{fitchproof}

\item \begin{fitchproof}
\hypo{a}{\forall x(\atom{J}{x} \eif \atom{K}{x})}
\hypo{b}{\exists x\,\forall y\, \atom{L}{x,y}}
\hypo{c}{\forall x\, \atom{J}{x}}
\open
	\hypo{2}{\forall y\, \atom{L}{a,y}}
	\have{3}{\atom{L}{a,a}}{}
	\have{d}{\atom{J}{a}}{}
	\have{e}{\atom{J}{a} \eif \atom{K}{a}}{}
	\have{f}{\atom{K}{a}}{}
	\have{4}{\atom{K}{a} \eand \atom{L}{a,a}}{}
	\have{5}{\exists x(\atom{K}{x} \eand \atom{L}{x,x})}{}
\close
\have{j}{\exists x(\atom{K}{x} \eand \atom{L}{x,x})}{}
\end{fitchproof}
\end{earg}

\problempart
\label{pr.BarbaraEtc.proof1}
In \S\ref{s:MoreMonadic} Übung A, betrachteten wir 15 syllogistische Figuren. Geben Sie Beweise, die diesen Argumentformen entsprechen. Beachten Sie: Die Übung ist \emph{viel} leichter, wenn Sie `Kein F ist G' (beispielsweise) als `$\forall x (\atom{F}{x} \eif \enot \atom{G}{x})$' symbolisieren.

\

\problempart
\label{pr.BarbaraEtc.proof2}
Aristoteles und seine Nachfolger*innen identifizierten weitere syllogistische Formen. Symbolisieren Sie diese in der LEO und geben Sie dementsprechende Beweise.
\begin{earg}
	\item \textbf{Barbari.} Etwas ist H. Alle Gs sind F. Alle Hs sind G. Also: Ein H ist F
	\item \textbf{Celaront.} Etwas ist H. Kein G ist F. Alle Hs sind G. Also: Ein H ist nicht F
	\item \textbf{Cesaro.} Etwas ist H. Kein F ist G. Alle Hs sind G. Also: Ein H ist nicht F.
	\item \textbf{Camestros.} Etwas ist H. Alle Fs sind G. Kein H ist G. Also: Ein H ist nicht F.
	\item \textbf{Felapton.} Etwas ist G. Kein G ist F. Alle Gs sind H. Also: Ein H ist nicht F.
	\item \textbf{Darapti.} Etwas ist G. Alle Gs sind F. Alle Gs sind H. Also: Ein H ist F.
	\item \textbf{Calemos.} Etwas ist H. Alle Fs sind G. Kein G ist H. Also: Ein H ist nicht F.
	\item \textbf{Fesapo.} Etwas ist G. Kein F ist G. All Gs sind H. Also: Ein H ist nicht F.
	\item \textbf{Bamalip.} Etwas ist F. Alle Fs sind G. Alle Gs sind H. Also: Ein H ist F.
\end{earg}

\problempart
\label{pr.someFOLproofs}
Beweisen Sie jede der folgenden Aussagen.
\begin{earg}
\item $\proves \forall x\,\atom{F}{x} \eif \forall y(\atom{F}{y} \eand \atom{F}{y})$
\item $\forall x(\atom{A}{x}\eif \atom{B}{x}), \exists x\,\atom{A}{x} \proves \exists x\,\atom{B}{x}$
\item $\forall x(\atom{M}{x} \eiff \atom{N}{x}), \atom{M}{a} \eand \exists x\,\atom{R}{x,a} \proves \exists x\,\atom{N}{x}$
\item $\forall x\, \forall y\,\atom{G}{x,y}\proves\exists x\,\atom{G}{x,x}$
\item $\proves\forall x\,\atom{R}{x,x} \eif \exists x\, \exists y\,\atom{R}{x,y}$
\item $\proves\forall y\, \exists x (\atom{Q}{y} \eif \atom{Q}{x})$
\item $\atom{N}{a} \eif \forall x(\atom{M}{x} \eiff \atom{M}{a}), \atom{M}{a}, \enot\atom{M}{b}\proves \enot \atom{N}{a}$
\item $\forall x\, \forall y (\atom{G}{x,y} \eif \atom{G}{y,x}) \proves \forall x\forall y (\atom{G}{x,y} \eiff \atom{G}{y,x})$
\item $\forall x(\enot\atom{M}{x} \eor \atom{L}{j,x}), \forall x(\atom{B}{x}\eif \atom{L}{j,x}), \forall x(\atom{M}{x}\eor \atom{B}{x})\proves \forall x\atom{L}{j,x}$
\end{earg}

\solutions
\problempart
\label{pr.likes}
Geben Sie einen Symbolisierungsschlüssel für das folgende Argument, symbolisieren Sie es und geben Sie einen dementsprechenden Beweis:
\begin{quote}
Es gibt jemanden, der jeden mag, der jeden mag, den er mag. Es gibt also jemanden, der sich selbst mag.
\end{quote}


\problempart
Zeigen Sie, dass die folgenden Satzpaare beweisbar äquivalent sind.
\begin{earg}
\item $\forall x (\atom{A}{x}\eif \enot \atom{B}{x})$, $\enot\exists x(\atom{A}{x} \eand \atom{B}{x})$
\item $\forall x (\enot\atom{A}{x}\eif \atom{B}{d})$, $\forall x\,\atom{A}{x} \eor \atom{B}{d}$
\item $\exists x\,\atom{P}{x} \eif \atom{Q}{c}$, $\forall x (\atom{P}{x} \eif \atom{Q}{c})$
\end{earg}

\solutions
\problempart
\label{pr.FOLequivornot}
Für jedes der folgenden Satzpaare: Wenn sie beweisbar äquivalent sind, legen Sie Beweise vor, um dies zu belegen. Wenn sie nicht äquivalent sind, konstruieren Sie eine Interpretation, um zu zeigen, dass sie es nicht sind.
\begin{earg}
\item $\forall x\,\atom{P}{x} \eif \atom{Q}{c}, \forall x (\atom{P}{x} \eif \atom{Q}{c})$
\item $\forall x\,\forall y\, \forall z\,\atom{B}{x,y,z}, \forall x\,\atom{B}{x,x}x$
\item $\forall x\,\forall y\,\atom{D}{x,y}, \forall y\,\forall x\,\atom{D}{x,y}$
\item $\exists x\,\forall y\,\atom{D}{x,y}, \forall y\,\exists x\,\atom{D}{x,y}$
\item $\forall x (\atom{R}{c,a} \eiff \atom{R}{x,a}), \atom{R}{c,a} \eiff \forall x\,\atom{R}{x,a}$
\end{earg}

\solutions
\problempart
\label{pr.FOLvalidornot}
Für jedes der folgenden Argumente: Wenn es in der LEO gültig ist, geben Sie einen Beweis. Wenn es ungültig ist, konstruieren Sie eine Interpretation, um zu zeigen, dass es ungültig ist.
\begin{earg}
\item $\exists y\,\forall x\,\atom{R}{x,y} \therefore \forall x\,\exists y\,\atom{R}{x,y}$
\item $\forall x\,\exists y\,\atom{R}{x,y} \therefore  \exists y\,\forall x\,\atom{R}{x,y}$
\item $\exists x(\atom{P}{x} \eand \enot \atom{Q}{x}) \therefore \forall x(\atom{P}{x} \eif \enot \atom{Q}{x})$
\item $\forall x(\atom{S}{x} \eif \atom{T}{a}), \atom{S}{d} \therefore \atom{T}{a}$
\item $\forall x(\atom{A}{x}\eif \atom{B}{x}), \forall x(\atom{B}{x} \eif \atom{C}{x}) \therefore \forall x(\atom{A}{x} \eif \atom{C}{x})$
\item $\exists x(\atom{D}{x} \eor \atom{E}{x}), \forall x(\atom{D}{x} \eif \atom{F}{x}) \therefore \exists x(\atom{D}{x} \eand \atom{F}{x})$
\item $\forall x\,\forall y(\atom{R}{x,y} \eor \atom{R}{y,x}) \therefore \atom{R}{j,j}$
\item $\exists x\,\exists y(\atom{R}{x,y} \eor \atom{R}{y,x}) \therefore \atom{R}{j,j}$
\item $\forall x\,\atom{P}{x} \eif \forall x\,\atom{Q}{x}, \exists x\, \enot\atom{P}{x} \therefore \exists x\, \enot \atom{Q}{x}$
\item $\exists x\,\atom{M}{x} \eif \exists x\,\atom{N}{x}$, $\enot \exists x\,\atom{N}{x}\therefore  \forall x\, \enot \atom{M}{x}$
\end{earg}

\chapter{Beweise mit Quantoren}

In \S\ref{s:stratTFL} diskutierten wir Strategien für Beweise, die die Grundregeln der WFL nutzen. Die gleichen Prinzipien gelten auch für die Regeln der Quantoren. 

Wenn wir $\forall \metav{x}\, \atom{\metav{A}}{\metav{x}}$ oder $\exists \metav{x}\, \atom{\metav{A}}{\metav{x}}$ herleiten wollen, können wir rückwärts arbeiten, indem wir den relevanten Satz mit $\forall$I oder $\exists$I rechtfertigen und dann versuchen, die Prämisse der Regel herzuleiten. Wenn wir hingegen von Sätzen mit Quantoren vorwärts arbeiten wollen, dann wenden wir $\forall$E oder $\exists$E an. 

Nehmen Sie an, wir wollen $\forall \metav{x}\, \atom{\metav{A}}{\metav{x}}$ herleiten. Um das mittels $\forall$I zu tun, müssten wir $\atom{\metav{A}}{\metav{c}}$ für einen Namen $\metav{c}$, der in keiner ungetilgten Annahme vorkommt, herleiten. Von $\forall \metav{x}\, \atom{\metav{A}}{\metav{x}}$ rückwärts arbeitend schreiben wir also $\atom{\metav{A}}{\metav{c}}$ darüber auf und versuchen dann, diesen Satz herzuleiten.
\begin{fitchproof}
	\ellipsesline
	\have[n]{n}{\atom{\metav{A}}{\metav{c}}}
	\have{m}{\forall \metav{x}\, \atom{\metav{A}}{\metav{x}}}\Ai{n}
\end{fitchproof}
$\atom{\metav{A}}{\metav{c}}$ erhalten wir von $\atom{\metav{A}}{\metav{x}}$, indem wir jedes ungebundene Vorkommnis von $\metav{x}$ in $\atom{\metav{A}}{\metav{x}}$ mit $\metav{c}$ ersetzen. Damit das funktioniert, muss $\metav{c}$ eine Einschränkung erfüllen. Wir garantieren, dass es das tut, indem wir immer einen Namen auswählen, der noch nicht im bis dato konstruierten Beweis vorkommt.

Um von $\exists \metav{x}\, \atom{\metav{A}}{\metav{x}}$ aus rückwärts zu arbeiten, schreiben wir ähnlicherweise einen Satz darüber auf, der uns als Rechtfertigung für eine Anwendung der $\exists$I-Regel dienen kann: einen Satz der Form $\atom{\metav{A}}{\metav{c}}$. 
\begin{fitchproof}
	\ellipsesline
	\have[n]{n}{\atom{\metav{A}}{\metav{c}}}
	\have{m}{\exists \metav{x}\, \atom{\metav{A}}{\metav{x}}}\Ei{n}
\end{fitchproof}
Das sieht genauso aus, als ob wir von einem Satz mit Universalquantor rückwärts arbeiten. Der Unterschied ist aber, dass wir für $\forall$I einen Namen $\metav{c}$ auswählen müssen, der bis dato noch nicht im Beweis vorkam, während wir für $\exists$I einen Namen $\metav{c}$ auswählen müssen, der bereits im Beweis vorkommt. Genau wie im Fall von $\eor$I ist oft nicht sofort klar, welcher Name $\metav{c}$ letztendlich funktioniert. Daher sollten Sie nur von einem Existenzquantor rückwärts arbeiten, wenn Sie alle anderen Strategien bereits versucht haben.

\emph{Vorwärts} vom Satz $\exists \metav{x}\, \atom{\metav{A}(\metav{x}})$ zu arbeiten funktioniert hingegen so gut wie immer. Um das zu tun, betrachten wir nicht nur $\exists \metav{x}\, \atom{\metav{A}}{\metav{x}}$, sondern auch den Satz $\metav{B}$, den Sie beweisen wollen. Vorwärts arbeitend müssen Sie einen Unterbeweis über $\metav{B}$ aufschreiben, wo $\metav{B}$ die letzte Zeile ist, und eine Substitutionsinstanz $\atom{\metav{A}}{\metav{c}}$ von $\exists \metav{x}\, \atom{\metav{A}}{\metav{x}}$ als Annahme fungiert. Um sicherzustellen, dass die Einschränkung von $\metav{c}$, die im Falle von $\exists$E gilt, erfüllt ist, wählen Sie einen Namen $\metav{c}$, der bis dato noch nicht in Ihrem Beweis vorkam. 
\begin{fitchproof}
	\ellipsesline
	\have[m]{m}{\exists \metav{x}\, \atom{\metav{A}}{\metav{x}}}
	\ellipsesline
	\open
	\hypo[n]{n}{\atom{\metav{A}}{\metav{c}}}
	\ellipsesline
	\have[k]{k}{\metav{B}}
	\close
	\have{e}{\metav{B}}\Ee{m,n-k}
\end{fitchproof}
Sie fahren dann mit dem Ziel fort, $\metav{B}$ herzuleiten, aber nun innerhalb eines Unterbeweises, in welchem Sie einen weiteren Satz nutzen können ($\atom{\metav{A}}{\metav{c}}$).

Zuletzt: Vorwärts von $\forall \metav{x}\, \atom{\metav{A}}{\metav{x}}$ arbeiten hei{\ss}t, dass sie $\atom{\metav{A}}{\metav{c}}$ (für irgendeinen Namen $\metav{c}$) aufschreiben und es mit $\forall$E rechtfertigen. Sie wollen das natürlich nicht einfach so tun. Nur bestimmte Namen $\metav{c}$ werden Ihnen dabei helfen, Ihren Zielsatz zu beweisen. So wie Sie also nur unter bestimmten Umständen von $\exists \metav{x}\, \atom{\metav{A}}{\metav{x}}$ rückwärts arbeiten sollten, sollten Sie auch nur von $\forall \metav{x}\, \atom{\metav{A}}{\metav{x}}$ vorwärts arbeiten, wenn Sie schon alle anderen Strategien angewandt haben.

Lasst uns $\forall x(\atom{A}{x} \eif B) \therefore \exists x\,\atom{A}{x} \eif B$ als Beispiel nehmen. Um unseren Beweis zu beginnen, schreiben wir die Prämisse oben und die Schlussfolgerung unten auf.
\begin{fitchproof}
\hypo{1}{\forall x(\atom{A}{x} \eif B)}
\ellipsesline
\have[n]{7}{\exists x\,\atom{A}{x} \eif B}
\end{fitchproof}
Die Strategien für die Junktoren der WFL gelten nach wie vor und Sie sollten sie in der gleichen Reihenfolge anwenden: zuerst rückwärts arbeiten von Konditionalen, Negationen, Konjunktionen und jetzt auch Universalquantoren; dann vorwärts arbeiten von Disjunktionen und jetzt Existenzquantoren; erst dann versuchen Sie, $\eif$E, $\enot$E, $\lor$I, $\forall$E oder $\exists$I anzuwenden. In unserem Fall bedeutet das, dass wir von der Schlussfolgerung aus rückwärts arbeiten:
\begin{fitchproof}
	\hypo{1}{\forall x(\atom{A}{x} \eif B)}
	\open
	\hypo{2}{\exists x\,\atom{A}{x}}
	\ellipsesline
	\have[n][-1]{6}{B}
	\close
	\have[n]{7}{\exists x\,\atom{A}{x} \eif B}\ci{2-(6)}
\end{fitchproof}
Unser nächster Schritt ist, dass wir von $\exists x\,\atom{A}{x}$ in Zeile $2$ vorwärts arbeiten. Dazu wählen wir einen Namen, der noch nicht in unserem Beweis vorkommt. Weil kein Name vorkommt, können wir irgendeinen wählen, z.B.\@ `$d$':
\begin{fitchproof}
	\hypo{1}{\forall x(\atom{A}{x} \eif B)}
	\open
	\hypo{2}{\exists x\,\atom{A}{x}}
	\open
	\hypo{3}{\atom{A}{d}}
	\ellipsesline
	\have[n][-2]{5}{B}
	\close
	\have[n][-1]{6}{B}\Ee{2,3-(5)}
	\close
	\have[n]{7}{\exists x\,\atom{A}{x} \eif B}\ci{2-(6)}
\end{fitchproof}
Nun haben wir unsere ersten Strategien erschöpft und wir arbeiten von $\forall x(\atom{A}{x} \eif B)$ aus vorwärts. $\forall$E zu nutzen hei{\ss}t, dass wir jede Instanz von $A(\metav{c}) \eif B$ rechtfertigen können, egal welches $\metav{c}$ wir auswählen. Wir wählen hier natürlich $d$, denn das gibt uns $\atom{A}{d} \eif B$. Und dann können wir $\eif$E anwenden um $B$ rechtzufertigen, womit wir den Beweis beenden.
\begin{fitchproof}
	\hypo{1}{\forall x(\atom{A}{x} \eif B)}
	\open
	\hypo{2}{\exists x\,\atom{A}{x}}
	\open
	\hypo{3}{\atom{A}{d}}
\have{4}{\atom{A}{d} \eif B}\Ae{1}
	\have{5}{B}\ce{4,3}
	\close
	\have{6}{B}\Ee{2,3-5}
	\close
	\have{7}{\exists x\,\atom{A}{x} \eif B}\ci{2-6}
\end{fitchproof}

Lasst uns nun einen Beweis des Umkehrschlusses angeben. Wir beginnen mit
\begin{fitchproof}
	\hypo{1}{\exists x\,\atom{A}{x} \eif B}
	\ellipsesline
	\have[n]{7}{\forall x(\atom{A}{x} \eif B)}
\end{fitchproof}
Die Prämisse ist hier ein Konditional und kein Satz mit einem Existenzquantor. Also sollten wir nicht von ihr ausgehend vorwärts arbeiten. Von der Schlussfolgerung aus rückwärts arbeitend, versuchen wir $\atom{A}{d} \eif B$ herzuleiten:
\begin{fitchproof}
	\hypo{1}{\exists x\,\atom{A}{x} \eif B}
	\ellipsesline
	\have[n][-1]{6}{\atom{A}{d} \eif B}
	\have[n]{7}{\forall x(\atom{A}{x} \eif B)}\Ai{6}
\end{fitchproof}
Und von $\atom{A}{d} \eif B$ aus rückwärts zu arbeiten, hei{\ss}t, dass wir einen Unterbeweis mit $\atom{A}{d}$ als Annahme und $B$ als Schlusszeile brauchen:
\begin{fitchproof}
	\hypo{1}{\exists x\,\atom{A}{x} \eif B}
	\open
	\hypo{2}{\atom{A}{d}}
	\ellipsesline
	\have[n][-2]{5}{B}
	\close
	\have[n][-1]{6}{\atom{A}{d} \eif B}\ci{2-(5)}
	\have[n]{7}{\forall x(\atom{A}{x} \eif B)}\Ai{6}
\end{fitchproof}
Nun können wir von der Prämisse in Zeile $1$ aus vorwärts arbeiten. Die ist ein Konditional und sein Konsequens ist gerade der Satz $B$, den wir rechtfertigen wollen. Also brauchen wir eine Herleitung des Antezedens, $\exists x\,\atom{A}{x}$. Diese Herleitung ist netterweise einfach erhältlich, indem wir $\exists$I auf Zeile $2$ anwenden. Damit sind wir auch schon fertig:
\begin{fitchproof}
	\hypo{1}{\exists x\,\atom{A}{x} \eif B}
	\open
	\hypo{2}{\atom{A}{d}}
	\have{3}{\exists x\,\atom{A}{x}}\Ei{2}
	\have{5}{B}\ce{1,3}
	\close
	\have{6}{\atom{A}{d} \eif B}\ci{2-5}
	\have{7}{\forall x(\atom{A}{x} \eif B)}\Ai{6}
\end{fitchproof}

\practiceproblems

\problempart
Verwenden Sie die obigen Strategien, um Beweise für jedes der folgenden Argumente und Theoreme zu finden:
\begin{earg}
\item $A \eif \forall x\,\atom{B}{x} \therefore \forall x(A \eif \atom{B}{x})$
\item $\exists x(A \eif \atom{B}{x}) \therefore A \eif \exists x\, \atom{B}{x}$
\item $\therefore \forall x(\atom{A}{x} \eand \atom{B}{x}) \eiff (\forall x\,\atom{A}{x} \eand \forall x\,\atom{B}{x})$
\item $\therefore \exists x(\atom{A}{x} \eor \atom{B}{x}) \eiff (\exists x\,\atom{A}{x} \eor \exists x\,\atom{B}{x})$
\item $A \eor \forall x\,\atom{B}{x}) \therefore \forall x(A \eor \atom{B}{x})$
\item $\forall x(\atom{A}{x} \eif B) \therefore \exists x\,\atom{A}{x} \eif B$
\item $\exists x(\atom{A}{x} \eif B) \therefore \forall x\,\atom{A}{x} \eif B$
\item $\therefore \forall x(\atom{A}{x} \eif \exists y\,\atom{A}{y})$
\end{earg}
Verwenden Sie zusätzlich zu den grundlegenden Quantorenregeln nur die Grundregeln der WFL.

\problempart
Verwenden Sie die obigen Strategien, um Beweise für jedes der folgenden Argumente und Theoreme zu finden:
\begin{earg}
\item $\forall x\,\atom{R}{x,x} \therefore \forall x\,\exists y\,\atom{R}{x,y}$
\item $\forall x\,\forall y\,\forall z[(\atom{R}{x,y} \eand \atom{R}{y,z}) \eif \atom{R}{x,z}]$ \\
$\therefore \forall x\,\forall y[\atom{R}{x,y} \eif \forall z(\atom{R}{y,z} \eif \atom{R}{x,z})]$
\item $\forall x\,\forall y\,\forall z[(\atom{R}{x,y} \eand \atom{R}{y,z}) \eif \atom{R}{x,z}],$\\ $\forall x\,\forall y(\atom{R}{x,y} \eif \atom{R}{y, x})$ \\ $\therefore \forall x\,\forall y\,\forall z[(\atom{R}{x,y} \eand \atom{R}{x,z}) \eif \atom{R}{y,z}]$
\item $\forall x\,\forall y(\atom{R}{x,y} \eif \atom{R}{y, x})$ \\$\therefore \forall x\,\forall y\,\forall z[(\atom{R}{x,y} \eand \atom{R}{x,z}) \eif \exists u(\atom{R}{y,u} \eand \atom{R}{z,u})]$
\item $\therefore \enot \exists x\,\forall y (\atom{A}{x, y} \eiff \lnot\atom{A}{y, y})$
\end{earg}

\problempart
Verwenden Sie die obigen Strategien, um Beweise für jedes der folgenden Argumente und Theoreme zu finden:
\begin{earg}
\item $\forall x\,\atom{A}{x} \eif B \therefore \exists x(\atom{A}{x} \eif B)$
\item $A \eif \exists x\, \atom{B}{x} \therefore \exists x(A \eif \atom{B}{x})$
\item $\forall x(A \eor \atom{B}{x}) \therefore A \eor \forall x\,\atom{B}{x})$
\item $\therefore \exists x(\atom{A}{x} \eif \forall y\,\atom{A}{y})$
\item $\therefore \exists x(\exists y\,\atom{A}{y} \eif \atom{A}{x})$
\end{earg}
Diese erfordern den Einsatz von IB. Verwenden Sie zusätzlich zu den grundlegenden Quantorenregeln nur die Grundregeln der WFL.
%IB oder IP? War in prooftfl auch inkonsistent übersetzt...

\chapter{Umwandlung der Quantoren}\label{s:CQ}

In diesem Kapitel werden wir ein paar zusätzliche Regeln zu unserem Herleitungssystem hinzufügen. Diese befassen sich mit dem Zusammenspiel von Quantoren und der Negation.
 
In \S\ref{s:FOLBuildingBlocks} stellten wir fest, dass $\enot\exists x\metav{A}$ zu $\forall x\, \enot\metav{A}$ äquivalent ist. Wir werden dementsprechende Regeln zu unserem Herleitungssystem hinzufügen:
	\factoidbox{
	\begin{fitchproof}
		\have[m]{a}{\forall \metav{x}\, \enot\metav{A}}
		\have[\ ]{con}{\enot \exists \metav{x}\, \metav{A}}\cq{a}
	\end{fitchproof}}
und
\factoidbox{
	\begin{fitchproof}
		\have[m]{a}{ \enot \exists \metav{x}\, \metav{A}}
		\have[\ ]{con}{\forall \metav{x}\, \enot \metav{A}}\cq{a}
	\end{fitchproof}}
Ebenfalls fügen wir
\factoidbox{
	\begin{fitchproof}
		\have[m]{a}{\exists \metav{x}\, \enot \metav{A}}
		\have[\ ]{con}{\enot \forall \metav{x}\, \metav{A}}\cq{a}
	\end{fitchproof}}
und
\factoidbox{
	\begin{fitchproof}
		\have[m]{a}{\enot \forall \metav{x}\, \metav{A}}
		\have[\ ]{con}{\exists \metav{x}\, \enot \metav{A}}\cq{a}
	\end{fitchproof}}
hinzu.

\practiceproblems
\problempart
Zeigen Sie in jedem Fall, dass die Sätze beweisbar inkonsistent sind:
\begin{earg}
\item $\atom{S}{a}\eif \atom{T}{m}, \atom{T}{m} \eif \atom{S}{a}, \atom{T}{m} \eand \enot \atom{S}{a}$
\item $\enot\exists x\,\atom{R}{x,a}, \forall x\, \forall y\,\atom{R}{y,x}$
\item $\enot\exists x\, \exists y\,\atom{L}{x,y}, \atom{L}{a,a}$
\item $\forall x(\atom{P}{x} \eif \atom{Q}{x}), \forall z(\atom{P}{z} \eif \atom{R}{z}), \forall y\,\atom{P}{y}, \enot \atom{Q}{a} \eand \enot \atom{R}{b}$
\end{earg}

\problempart
Zeigen Sie, dass jedes Satzpaar beweisbar äquivalent ist:
\begin{earg}
\item $\forall x (\atom{A}{x}\eif \enot \atom{B}{x}), \enot\exists x(\atom{A}{x} \eand \atom{B}{x})$
\item $\forall x (\enot\atom{A}{x}\eif \atom{B}{d}), \forall x\,\atom{A}{x} \eor \atom{B}{d}$
\end{earg}

\problempart
In \S\ref{s:MoreMonadic} haben wir darüber nachgedacht, was passiert, wenn wir Quantoren über verschiedene Operatoren ``hinweg'' bewegen. Zeigen Sie, dass jedes der folgenden Satzpaare beweisbar äquivalent ist:
\begin{earg}
\item $\forall x(\atom{F}{x} \eand \atom{G}{a}), \forall x\,\atom{F}{x} \eand \atom{G}{a}$
\item $\exists x(\atom{F}{x} \eor \atom{G}{a}), \exists x\,\atom{F}{x} \eor \atom{G}{a}$
\item $\forall x(\atom{G}{a} \eif \atom{F}{x}), \atom{G}{a} \eif \forall x\,\atom{F}{x}$
\item $\forall x(\atom{F}{x} \eif \atom{G}{a}), \exists x\,\atom{F}{x} \eif \atom{G}{a}$
\item $\exists x(\atom{G}{a} \eif \atom{F}{x}), \atom{G}{a} \eif \exists x\,\atom{F}{x}$
\item $\exists x(\atom{F}{x} \eif \atom{G}{a}), \forall x\,\atom{F}{x} \eif \atom{G}{a}$
\end{earg}
Beachten Sie: Die Variable `$x$' kommt nicht in `$\atom{G}{a}$' vor. Wenn alle Quantoren am Anfang eines Satzes stehen, ist dieser Satz in \emph{Pränexform}. Die obigen Äquivalenzen werden manchmal als \emph{Pränexionsregeln} bezeichnet, da sie uns erlauben, jeden Satz in eine Pränexform zu gie{\ss}en.

\chapter{Identitätsregeln}
In \S\ref{s:Interpretations} erwähnten wir die kontrovers diskutierte These der Identität des Ununterscheidbaren. Laut dieser These sind Objekte, die jeder Hinsicht nach ununterscheidbar sind, identisch. Wir erwähnten auch, dass wir diese These hier nicht akzeptieren werden. Daraus folgt, dass wir, egal wie viel wir über zwei Objekte erfahren, nicht beweisen können, dass sie identisch sind; es sei denn natürlich, man erfährt, dass die beiden Objekte identisch sind, aber dann ist der Beweis nicht allzu interessant.

Allgemein gilt für uns also, dass \emph{keine Sätze}, die nicht schon das Identitätsprädikat enthalten, eine Herleitung von `$a=b$' rechtfertigen. Unsere Identitätseinführungsregel erlaubt uns als nicht, einen Identitätssatz herzuleiten, der zwei \emph{verschiedene} Namen verwendet.

Jedes Objekt ist jedoch mit sich selbst identisch. Daher sind keine Prämissen notwendig, um herzuleiten, dass etwas mit sich selbst identisch ist. Das ist also unsere Identitätseinführungsregel:
\factoidbox{
\begin{fitchproof}
	\have[\ \,\,\,]{x}{\metav{c}=\metav{c}} \by{=I}{}
\end{fitchproof}}
Diese Regel benötigt kein Zitat einer vorhergehenden Zeile. Für jeden Namen \metav{c} können wir direkt auf $\metav{c}=\metav{c}$ schlie{\ss}en; wir brauchen nur die {=}I Regel as Rechtfertigung anzugeben. 

Unsere Eliminationsregel bietet uns mehr Möglichkeiten. Wenn Sie `$a=b$' haben, dann muss alles, was auf `$a$' zutrifft, auch auf `$b$' zutreffen. Wir können also in jedem Satz, in dem `$a$' vorkommt, einige oder alle Vorkommnisse von `$a$' mit `$b$' ersetzen. Von `$\atom{R}{a,a}$' und `$a = b$', können wir z.B.\@ `$\atom{R}{a,b}$', `$\atom{R}{b,a}$' und `$\atom{R}{b,b}$' herleiten. Allgemein gesprochen:
\factoidbox{\begin{fitchproof}
	\have[m]{e}{\metav{a}=\metav{b}}
	\have[n]{a}{\metav{A}(\ldots \metav{a} \ldots \metav{a}\ldots)}
	\have[\ ]{ea1}{\metav{A}(\ldots \metav{b} \ldots \metav{a}\ldots)} \by{=E}{e,a}
\end{fitchproof}}
Die Notation hier ist die gleiche wie im Fall von $\exists$I. $\metav{A}(\ldots \metav{a} \ldots \metav{a}\ldots)$ ist eine Formel, die den Namen $\metav{a}$ enthält, und $\metav{A}(\ldots \metav{b} \ldots \metav{a}\ldots)$ eine Formel, die wir erhalten, indem wir ein Vorkommnis oder mehrere Vorkommnisse von $\metav{a}$ mit $\metav{b}$ ersetzen. Zeilen $m$ und $n$ können in beliebiger Reihenfolge auftreten, und müssen nicht nebeneinander vorkommen. Aber wir zitieren immer zuerst den Identitätssatz. Symmetrisch erlauben wir auch:
\factoidbox{\begin{fitchproof}
	\have[m]{e}{\metav{a}=\metav{b}}
	\have[n]{a}{\metav{A}(\ldots \metav{b} \ldots \metav{b}\ldots)}
	\have[\ ]{ea2}{\metav{A}(\ldots \metav{a} \ldots \metav{b}\ldots)} \by{=E}{e,a}
\end{fitchproof}}
Die Regel wird manchmal das \emph{Leibniz-Gesetz} genannt, nach Gottfried Wilhelm Leibniz. 

Lasst uns ein paar schnelle Beweise geben, um die Regeln anzuwenden. Zuerst beweisen wir, dass die Identität \emph{symmetrisch} ist:
\begin{fitchproof}
	\open
		\hypo{ab}{a = b}
		\have{aa}{a = a}\by{=I}{}
		\have{ba}{b = a}\by{=E}{ab, aa}
	\close
	\have{abba}{a = b \eif b =a}\ci{ab-ba}
	\have{ayya}{\forall y (a = y \eif y = a)}\Ai{abba}
	\have{xyyx}{\forall x\, \forall y (x = y \eif y = x)}\Ai{ayya}
\end{fitchproof}
Wir erhalten Zeile 3, indem wir ein Vorkommnis von `$a$' in Zeile 2 mit einem Vorkommnis von `$b$' ersetzen; dies dürfen wir, weil `$a= b$'. 

Zweitens beweisen wir, dass die Identität \emph{transitiv} ist:
\begin{fitchproof}
	\open
		\hypo{abc}{a = b \eand b = c}
		\have{ab}{a = b}\ae{abc}
		\have{bc}{b = c}\ae{abc}
		\have{ac}{a = c}\by{=E}{ab, bc}
	\close
	\have{con}{(a = b \eand b =c) \eif a = c}\ci{abc-ac}
	\have{conz}{\forall z((a = b \eand b = z) \eif a = z)}\Ai{con}
	\have{cony}{\forall y\,\forall z((a = y \eand y = z) \eif a = z)}\Ai{conz}
	\have{conx}{\forall x\,\forall y \forall z((x = y \eand y = z) \eif x = z)}\Ai{cony}
\end{fitchproof}
Wir erhalten Zeile 4, indem wir `$b$' in Zeile 3 mit `$a$' ersetzen; dies dürfen wir, weil `$a= b$'. 

\practiceproblems
\problempart
\label{pr.identity}
Beweisen Sie die folgenden Aussagen.
\begin{earg}
\item $\atom{P}{a} \eor \atom{Q}{b}, \atom{Q}{b} \eif b=c, \enot\atom{P}{a} \proves \atom{Q}{c}$
\item $m=n \eor n=o, \atom{A}{n} \proves \atom{A}{m} \eor \atom{A}{o}$
\item $\forall x\ x=m, \atom{R}{m,a} \proves \exists x\,\atom{R}{x,x}$
\item $\forall x\,\forall y(\atom{R}{x,y} \eif x=y)\proves \atom{R}{a,b} \eif \atom{R}{b,a}$
\item $\enot \exists x\enot x = m \proves \forall x\,\forall y (\atom{P}{x} \eif \atom{P}{y})$
\item $\exists x\,\atom{J}{x}, \exists x\, \enot\atom{J}{x}\proves \exists x\, \exists y\, \enot x = y$
\item $\forall x(x=n \eiff \atom{M}{x}), \forall x(\atom{O}{x} \eor \enot \atom{M}{x})\proves \atom{O}{n}$
\item $\exists x\,\atom{D}{x}, \forall x(x=p \eiff \atom{D}{x})\proves \atom{D}{p}$
\item $\exists x\bigl[(\atom{K}{x} \eand \forall y(\atom{K}{y} \eif x=y)) \eand \atom{B}{x}\bigr], \atom{K}{d}\proves \atom{B}{d}$
\item $\proves \atom{P}{a} \eif \forall x(\atom{P}{x} \eor \enot x = a)$
\end{earg}

\problempart
Zeigen Sie, dass die folgenden Sätze beweisbar äquivalent sind.
\begin{ebullet}
\item $\exists x \bigl([\atom{F}{x} \eand \forall y (\atom{F}{y} \eif x = y)] \eand x = n\bigr)$
\item $\atom{F}{n} \eand \forall y (\atom{F}{y} \eif n= y)$
\end{ebullet}

\problempart
In \S\ref{sec.identity} behaupteten wir, dass die folgenden Symbolisierungen von `Es gibt genau ein $F$' beweisbar äquivalent sind:
\begin{ebullet}
\item $\exists x\,\atom{F}{x} \eand \forall x\, \forall y \bigl[(\atom{F}{x} \eand \atom{F}{y}) \eif x = y\bigr]$
\item $\exists x \bigl[\atom{F}{x} \eand \forall y (\atom{F}{y} \eif x = y)\bigr]$
\item $\exists x\, \forall y (\atom{F}{y} \eiff x = y)$
\end{ebullet}
Zeigen Sie, dass dem so ist. (\emph{Hinweis}: um zu zeigen, dass die drei Sätze äquivalent sind, reicht es zu zeigen, dass der erste den zweiten zur Folge hat, der zweite den dritten, und der dritte den ersten; überlegen Sie, wieso das so ist.)


\
\problempart
Symbolisieren Sie das folgende Argument
	\begin{quote}
		Es gibt genau ein $F$. Es gibt genau ein $G$. Nichts ist sowohl $F$ als auch~$G$. Also: Es gibt genau zwei Dinge, die entweder~$F$ oder~$G$ sind.
	\end{quote}
und geben Sie einen entsprechenden Beweis.
%\begin{ebullet}
%\item  $\exists x \bigl[\atom{F}{x} \eand \forall y (\atom{F}{y} \eif x = y)\bigr], \exists x \bigl[\atom{G}{x} \eand \forall y (\atom{G}{y} \eif x = y)\bigr], \forall x (\enot\atom{F}{x} \eor \enot \atom{G}{x}) \proves \exists x\, \exists y \bigl[\enot x = y \eand \forall z ((\atom{F}{z} \eor \atom{G}{z}) \eif (x = y \eor x = z))\bigr]$
%\end{ebullet}

\chapter{Abgeleitete Regeln}\label{s:DerivedFOL}
Wie im Falle der WFL führten wir zuerst einige Grundregeln ein (in \S\ref{s:BasicFOL}) und fügten dann einige weitere Regeln hinzu (in \S\ref{s:CQ}). Tatsächlich sind die Regeln zur Umwandlung der Quantoren \emph{abgeleitete} Regeln, weil sie von den \emph{Grundregeln} in \S\ref{s:BasicFOL} hergeleitet werden können. Hier ist eine Rechtfertigung für die erste Umwandlungsregel:
\begin{fitchproof}
	\hypo{An}{\forall x\, \enot \atom{A}{x}}
	\open
		\hypo{E}{\exists x\, \atom{A}{x}}
		\open
			\hypo{c}{\atom{A}{c}}
			\have{nc}{\enot \atom{A}{c}}\Ae{An}
			\have{red}{\ered}\ne{nc,c}
		\close
		\have{red2}{\ered}\Ee{E,c-red}
	\close
	\have{dada}{\enot \exists x\, \atom{A}{x}}\ni{E-red2}
\end{fitchproof}
Und hier ist eine Rechtfertigung für die dritte Umwandlungsregel:
\begin{fitchproof}
	\hypo{nEna}{\exists x\, \enot \atom{A}{x}} 
	\open
		\hypo{Aa}{\forall x\, \atom{A}{x}}
		\open
			\hypo{nac}{\enot \atom{A}{c}}
			\have{a}{\atom{A}{c}}\Ae{Aa}
			\have{con}{\ered}\ne{nac,a}
		\close
		\have{con1}{\ered}\Ee{nEna, nac-con}
	\close
	\have{dada}{\enot \forall x\, \atom{A}{x}}\ni{Aa-con1}
\end{fitchproof}
Ähnliche Rechtfertigungen können auch für die zwei verbleibenden Umwandlungsregeln gegeben werden.

\practiceproblems

\problempart
Geben Sie Beweise, die das Hinzufügen der zweiten und vierten Umwandlungsregeln rechtfertigen.



\chapter{Beweise und Semantik}
Wir haben in diesem Lehrbuch zwei unterschiedliche Drehkreuze verwendet. Dieses hier:
$$\metav{A}_1, \metav{A}_2, \ldots, \metav{A}_n \proves \metav{C}$$
bedeutet, dass es einen Beweis gibt, der mit $\metav{C}$ beginnt und dessen ungetilgten Annahmen $\metav{A}_1, \metav{A}_2, \ldots, \metav{A}_n$ sind. Dieses Drehkreuz ist ein \emph{beweistheoretischer Begriff}. Das zweite Drehkreuz hingegen: 
$$\metav{A}_1, \metav{A}_2, \ldots, \metav{A}_n \entails \metav{C}$$
bedeutet, dass es keine Bewertung oder Interpretation gibt, die $\metav{A}_1, \metav{A}_2, \ldots, \metav{A}_n$ alle wahr, und~$\metav{C}$ falsch, macht. Dieses Drehkreuz befasst sich mit Zuordnungen von Wahrheitswerten zu Sätzen. Es ist ein \emph{semantischer Begriff}.

Es ist sehr wichtig, dass es sich hier um zwei unterschiedliche Begriffe handelt. Erst wenn Sie diesen Punkt verinnerlicht haben, lesen Sie weiter. 

Obwohl unsere semantischen und beweistheoretischen Begriffe unterschiedlich sind, sind sie eng verwandt. Um das zu erklären, werden wir zunächst die Beziehung zwischen Tautologien und Theoremen betrachten.

Um zu zeigen, dass ein Satz ein Theorem ist, brauchen Sie nur einen Beweis zu erbringen. Zugegeben, es mag schwierig sein, einen zwanzigzeiligen Beweis zu erbringen, aber es ist nicht so schwierig, jede Zeile des Beweises zu überprüfen und zu bestätigen, dass sie zulässig ist. Und wenn jede Zeile des Beweises für sich betrachtet zulässig ist, dann ist der gesamte Beweis zulässig. Um zu zeigen, dass ein Satz eine Tautologie ist, müssen wir jedoch alle möglichen Interpretationen überblicken. Müssten wir wählen, ob wir nachweisen, dass ein Satz ein Theorem ist, oder nachweisen, dass er eine Tautologie ist, wäre es einfacher zu zeigen, dass es sich um ein Theorem handelt.

Umgekehrt gilt: zu zeigen, dass ein Satz \emph{kein} Theorem ist, ist schwer. Wir müssen alle (möglichen) Beweise überblicken. Um jedoch zu zeigen, dass ein Satz \emph{keine} Tautologie ist, brauchen Sie nur eine Interpretation zu konstruieren, in der dieser Satz falsch ist. Zugegeben, es mag schwierig sein, eine solche Interpretation zu finden; aber wenn Sie das einmal getan haben, ist es relativ einfach zu überprüfen, welchen Wahrheitswert sie einem Satz zuweist. Müssten wir wählen ob wir nachweisen, dass ein Satz kein Theorem ist, oder nachweisen, dass er keine Tautologie ist, wäre es einfacher zu zeigen, dass es sich nicht um eine Tautologie handelt.

Glücklicherweise ist \emph{ein Satz genau dann ein Theorem, wenn er eine Tautologie ist}. Wenn wir also einen Beweis $\metav{A}$s ohne ungetilgte Annahmen erbringen und zeigen, dass $\metav{A}$ ein Theorem ist (${}\proves \metav{A}$), dann zeigen wir auch, dass $\metav{A}$ eine Tautologie ist ($\entails\metav{A}$). Ähnlicherweise gilt: wenn wir eine Interpretation konstruieren in der \metav{A} falsch ist und zeigen, dass es keine Tautologie ist ($\nentails \metav{A}$), dann zeigen wir auch, dass \metav{A} kein Theorem ist ($\nproves \metav{A}$).

Allgemein gesprochen haben wir das folgende wirkmächtige Resultat:
$$\metav{A}_1, \metav{A}_2, \ldots, \metav{A}_n \proves\metav{B} \textbf{genau dann, wenn}\metav{A}_1, \metav{A}_2, \ldots, \metav{A}_n \entails\metav{B}$$
Das zeigt, dass Beweisbarkeit und Folge zwar \emph{verschiedene} Begriffe sind, aber die \emph{gleiche} Extension haben:
	\begin{ebullet}
		\item Ein Argument ist \emph{gültig} genau dann, wenn \emph{die Schlussfolgerung von den Prämissen hergeleitet werden kann}.
		\item Zwei Sätze sind \emph{äquivalent} genau dann, wenn sie \emph{beweisbar äquivalent sind}.
		\item Sätze sind \emph{konsistent} genau dann, wenn sie \emph{nicht beweisbar inkonsistent} sind.
	\end{ebullet}
Aus diesem Grund können Sie sich aussuchen, wann Sie auf Beweise und auf Bewertungen/Interpretationen zurückgreifen. Je nachdem, was für eine bestimmte Aufgabe einfacher ist, können Sie den Weg des geringeren Widerstands gehen. Die Tabelle auf der nächsten Seite fasst zusammen, was (in der Regel) einfacher ist.

Es ist intuitiv, dass Beweisbarkeit und Folge übereinstimmen. Aber--lasst uns das wiederholen--wir dürfen uns nicht von der Ähnlichkeit der Symbole `$\entails$' und `$\proves$' täuschen lassen. Diese beiden Symbole haben sehr unterschiedliche Bedeutungen. Die Tatsache, dass Beweisbarkeit und Folge übereinstimmen, ist kein leicht zu erreichendes Ergebnis. Tatsächlich ist der Nachweis, dass Beweisbarkeit und Folge übereinstimmen, ein entscheidender Punkt, an dem wir die Logik für Fortgeschrittene erreichen.

\begin{sidewaystable}\small
\begin{center}
\begin{tabular*}{\textwidth}{p{.25\textheight}p{.325\textheight}p{.325\textheight}}
 & \textbf{Ja}  & \textbf{Nein}\\
\\
Ist \metav{A} eine \textbf{Tautologie}? 
& Beweis der zeigt, dass $\proves\metav{A}$ 
& Interpretation, in der \metav{A} falsch ist\\
\\
Ist \metav{A} ein \textbf{Widerspruch}? &
Beweis der zeigt, dass $\proves\enot\metav{A}$ & 
Interpretation, in der \metav{A} wahr ist\\
\\
%Ist \metav{A} kontingent? & 
%zwei Interpretationen, eine, in der \metav{A} wahr ist und eine in der \metav{A} falsch ist & give a proof which either shows $\proves\metav{A}$ or $\proves\enot\metav{A}$\\
%\\
Sind \metav{A} und \metav{B} \textbf{äquivalent}? &
zwei Beweise, einer für $\metav{A}\proves\metav{B}$, einer für $\metav{B}\proves\metav{A}$  
& Interpretation. in der \metav{A} und \metav{B} verschiedene Wahrheitswerte haben \\
\\
Sind $\metav{A}_1, \metav{A}_2, \ldots, \metav{A}_n$ \textbf{konsistent}? 
& Interpretation in der $\metav{A}_1, \metav{A}_2, \ldots, \metav{A}_n$ alle wahr sind 
& Beweis eines Widerspruchs ausgehend von $\metav{A}_1, \metav{A}_2, \ldots, \metav{A}_n$\\
\\
Ist $\metav{A}_1, \metav{A}_2, \ldots, \metav{A}_n \therefore \metav{C}$ \textbf{gültig}? 
& Beweis mit Annahmen $\metav{A}_1, \metav{A}_2, \ldots, \metav{A}_n$ und Schlussfolgerung \metav{C}
& Interpretation, in der $\metav{A}_1, \metav{A}_2, \ldots, \metav{A}_n$ wahr sind und \metav{C} falsch ist\\
\end{tabular*}
\end{center}
\end{sidewaystable}


